\documentclass[11pt, a4paper, german]{article}

\usepackage[utf8]{inputenc}
\usepackage[ngerman]{babel}
\usepackage{amsmath,amsthm,amssymb}
\usepackage[protrusion=true,expansion=true]{microtype}
\usepackage[table]{xcolor} % Für farbige Tabellenzellen
\usepackage{hhline} % mehr Kontrolle über Tabellenlinien
\usepackage{tikz}
\usepackage{enumitem} % bessere Aufzählungen
\usepackage{mathtools} % \coloneqq, \DeclarePairedDelimiter
\usepackage{stmaryrd} % \mapsfrom
\usepackage{nicefrac}
\usetikzlibrary{cd}
\usepackage{geometry}
\geometry{margin=2.5cm}
\usepackage{xfrac}
\usepackage{cutwin} % Bild neben Text
\usepackage{mathabx} % \divides
\usepackage{array} % für \setlength\extrarowheight{2pt}
\usepackage{lscape} % Für Querformat
\newcommand{\ceil}[1]{\lceil #1 \rceil} % Aufrunden

\usepackage[style=alphabetic-verb, backend=bibtex]{biblatex} % Literaturverzeichnis
\bibliography{literatur.bib}

\theoremstyle{definition}
\newtheorem{lem}{Lemma}
\newtheorem{bsp}[lem]{Beispiel}
\newtheorem{satz}[lem]{Satz}
\newtheorem{defn}[lem]{Definition}
\newtheorem{beob}[lem]{Beobachtung}
\newtheorem{bspe}[lem]{Beispiele}
\newtheorem{kor}[lem]{Korollar}
\newtheorem{axiom}[lem]{Axiom}
%\newtheorem*{nota}{Notation}

\theoremstyle{remark}
\newtheorem*{bem}{Bemerkung}

\newcommand{\TODO}[1]{\textcolor{red}{TODO: #1}} %\red{#1}} % TODO-Markierungen

% Zahlbereiche
\newcommand{\R}{\mathbb{R}} % Reelle Zahlen
\newcommand{\N}{\mathbb{N}} % Natürliche Zahlen
\newcommand{\Z}{\mathbb{Z}} % Ganze Zahlen
\newcommand{\C}{\mathbb{C}} % Komplexe Zahlen
\newcommand{\Q}{\mathbb{Q}} % Rationale Zahlen
\newcommand{\ZZ}[2]{\mathbb{Z}_{\text{\scriptsize \,$#1$}}^{\text{\scriptsize \,$#2$}}} % Ganze Zahlen mod n hoch m

% Schöne Fürall- und Existenzquantoren
\newcommand{\fa}[1]{\forall \, {#1} \,:\,}
%\newcommand{\ex}[1]{\exists \, {#1} \,:\,}

\DeclareMathOperator{\Hom}{Hom} % Homomorphisms
\DeclareMathOperator{\id}{id} % Identität
\DeclareMathOperator{\spann}{spann} % Spann
\DeclareMathOperator{\im}{im} % Image (Bild)
\DeclareMathOperator{\coker}{coker} % Kokern
\newcommand{\pt}{\mathrm{pt}}
\newcommand{\blank}{\text{--}} % Platzhalter
\newcommand{\ntimes}{\!\times\!} % schmaleres (narrower) \times
\newcommand{\angles}[1]{{\langle #1 \rangle}}
\newcommand{\const}[1]{\text{konst } #1} % konstante Funktion mit Wert #1
\newcommand{\LH}{\mathcal{H}} % Local homology
\DeclareMathOperator{\colim}{colim} % Kolimes
\newcommand{\nspace}[1]{\foreach \i in {1,...,#1}{ \! }} % Negativer Abstand
\DeclareMathOperator{\Tor}{Tor} % Tor-Funktor
\newcommand{\SC}{\mathcal{C}} % Serre-Klasse
\newcommand{\FG}{\mathcal{FG}} % Serre-Klasse der endlich erzeugten Gruppen
\newcommand{\T}{\mathcal{T}} % Serre-Klasse der endlich erzeugten Gruppen, deren Ordnung nur durch Primzahlen in einer bestimmten Menge teilbar ist
\newcommand{\F}{\mathcal{F}} % Serre-Klasse der endlichen Gruppen
\DeclarePairedDelimiter\abs{\lvert}{\rvert} % Absolutwert
\DeclarePairedDelimiter\ord{\lvert}{\rvert} % Gruppenordnung
\DeclarePairedDelimiter\norm{\lVert}{\rVert} % Norm
\DeclareMathOperator{\ggT}{ggT} % größter gemeinsamer Teiler
\DeclareMathOperator{\Aut}{Aut} % Automorphismengruppe
\newcommand{\trunc}[2]{\tau_{#1} #2} % Truncation/Postnikov-Section
\DeclareMathOperator{\sk}{sk} % Skelett
\newcommand{\V}[2]{V_{{#2},{#1}}} % $\V{k}{n}$ ist die Stiefel-Mannigfaltigkeit der orthonormalen $k$-Tupel im $\R^n$
\newcommand{\Primes}{\mathbb{P}} % Menge der Primzahlen

% Abkürzungen
\newcommand{\zB}{z.\,B.}
\renewcommand{\dh}{d.\,h.} % das heißt

% Intervalle
\newcommand{\cinterval}[2]{\left[ #1, #2 \right]} % closed interval
\newcommand{\I}{I} % Kompaktes Einheitsinterval [0,1]
%\newcommand{\I}{\cinterval{0}{1}} % Kompaktes Einheitsinterval [0,1] (alternativ)

\newcommand{\fall}[1]{\emph{#1}:\enspace}

% Schöne Mengen { #1 | #2 } (benötigt mathtools)
% siehe http://tex.stackexchange.com/questions/13634/define-pretty-sets-in-latex-esp-how-to-do-the-condition-separator
\DeclarePairedDelimiterX\Set[2]{\lbrace}{\rbrace}%
 { #1 \,\delimsize|\, #2 }

% http://tex.stackexchange.com/questions/117732/tikz-and-babel-error
% Es ist schierer Wahnsinn, welche Hacks LaTeX benötigt!
\tikzset{
  every picture/.prefix style={
    execute at begin picture=\shorthandoff{"}
  }
}

% Zentrierte kommutative Diagramme
\newenvironment{centertikzcd}
  {\begin{center}\begin{tikzcd}}
  {\end{tikzcd}\end{center}}

% TikZ-Makros
\newcommand{\zeroDot}[2]{\node[draw,circle,inner sep=0.4pt,fill] at (#1,#2) {};} % Punkt, der in Spektralsequenz für eine Null steht

\begin{document}


\begin{titlepage}
  \begin{center}
    \ \vspace{0.1cm}\\
    \Huge \textbf{\textsc{Der Endlichkeitssatz von Serre über die Homotopiegruppen der Sphären}} \vspace{0.2cm}\\
    \vspace{2cm}
    \LARGE \textbf{\textsc{Bachelorarbeit}}\\
    von\\
    \textbf{Tim Baumann}\\
    \vspace{1.1cm}
    \begin{figure}[htb]
      \centering
      %\includegraphics [scale=0.19] {D:/Studium_Mathe_Augsburg_6/Clifford-Bundel/SignUA.pdf}
    \end{figure}
    \vspace{1.1cm}
    \large Eingereicht am \\
    22. 12. 2015 \\[1cm]
    \includegraphics[scale=0.4]{uni_aux_logo.jpg} \\[1cm]
    %\large Institut für Mathematik\\ Universität Augsburg \\
    \vspace{1.2cm}
    \large \begin{tabular}{ll}
      Erstgutachter: & Prof. Dr. Bernhard Hanke \\
      Zweitgutachter: & \TODO{???}
    \end{tabular}
  \end{center}
\end{titlepage}


\tableofcontents
\newpage

%\section{Einleitung}

\section{Homotopiegruppen der Sphären}

\begin{defn}
  Es sei $(X, x_0)$ ein punktierter topologischer Raum.
  Für $n \geq 0$ sei
  \[ \pi_n(X, x_0) \coloneqq [(S^n, *), (X, x_0)] \]
  die Menge der basispunkterhaltenden Abbildungen $S^n \to X$ modulo basispunkterhaltender Homotopie.
  Für $n \geq 1$ heißt $\pi_n(X)$ die \emph{$n$-te Homotopiegruppe} von $X$ (mit Basispunkt $x_0$).
\end{defn}

Die Gruppenstruktur auf $\pi_n(X)$ wird induziert durch die Kogruppenstruktur auf $S^n$, welche durch die Abbildung $S^n \to S^n \vee S^n$ gegeben ist, die die Südkappe $D^n_+$ auf die erste $S^n$, die Nordkappe $D^n_{-}$ auf die zweite $S^n$ und den Äquator $D^n_{+} \cap D^n_{-}$ auf den Anklebepunkt von $S^n \vee S^n$ abbildet.
Das neutrale Element der Homotopiegruppen wird repräsentiert durch die konstante Abbildung auf den Basispunkt $x_0$.
Die erste Homotopiegruppe $\pi_1(X, x_0)$ kann man auffassen als die Menge der geschlossenen Wege bei $x_0$ modulo Homotopie mit der Verkettung von Wegen als Gruppenverknüpfung.
Diese Gruppe wird auch \emph{Fundamentalgruppe} genannt.
Man kann zeigen, dass $\pi_n(X, x_0)$ für $n \geq 2$ abelsch ist (vgl. \cite[340]{hatcher:at}).

Eine basispunkterhaltende Abbildung $g : (X, x_0) \to (Y, y_0)$ induziert Abbildungen $g_* = \pi_i(g) : \pi_i(X, x_0) \to \pi_i(Y, y_0)$ durch Nachkomponieren mit $g$.
Auf diese Weise wird $\pi_i$ zu einem Funktor von der Kategorie der punktierten topologischen Räume in die Kategorie der (abelschen) Gruppen.

Die Menge $\pi_0(X, x_0)$ ist die Menge der Wegzusammenhangskomponenten von $X$.
Wir schreiben $\pi_0(X, x_0) = 0$, falls die einzige Wegzusammenhangskomponente von $X$ die von $x_0$ ist.

Falls ein Weg von $x_0$ nach $x'_0$ in $X$ existiert, so sind $\pi_n(X, x_0)$ und $\pi_n(X, x'_0)$ isomorph.
Für nicht leere, wegzusammenhängende Räume $X$ kann man daher den Basispunkt in der Notation weglassen und von der Homotopiegruppe $\pi_n(X)$ von $X$ sprechen.

\TODO{Motivation für Studium der Homotopiegruppen der Sphären}

Für $i < n$ gilt $\pi_i(S^n) = 0$:
Wir können die Sphären $S^i$ und $S^n$ als CW-Komplexe mit jeweils einer Nullzelle und einer $i$-Zelle bzw. $n$-Zelle realisieren.
Aus dem zellulären Approximationstheorem (siehe \cite[\mbox{}Thm 4.8]{hatcher:at}) folgt, dass jede Abbildung $f : S^i \to S^n$ homotop relativ Basispunkt zu einer zellulären Abbildung ist, \dh{} einer Abbildung $\tilde{f} : S^i \to S^n$, die das $i$-Skelett von $S^i$ (das ist ganz $S^i$) auf das $i$-Skelett von $S^n$ (das ist $\{ * \}$) abbildet.
In anderen Worten ist jede Abbildung $S^i \to S^n$ homotop zur konstanten Abbildung und repräsentiert daher das neutrale Element in $\pi_i(S^n)$.

Mit der universellen Überlagerung $p : \R \to S^1, \enspace t \mapsto e^{it}$ kann man die Homotopiegruppen von $S^1$ bestimmen:
Die Fundamentalgruppe $\pi_1(S^1)$ ist isomorph zur Decktransformationsgruppe dieser Überlagerung, also isomorph zu~$\Z$, wie man leicht zeigen kann.
Die höheren Homotopiegruppen $\pi_i(S^1)$ mit $i > 1$ sind Null.
Dies folgt aus der Zusammenziehbarkeit von~$\R$ und aus der allgemeinen Tatsache, dass eine Überlagerung~$p : \widetilde{X} \to X$ Isomorphismen $\pi_i(p) : \pi_i(\widetilde{X}, \widetilde{x}_0) \to \pi_i(X, x_0)$ auf den höheren Homotopiegruppen induziert (siehe \cite[\mbox{}Prop 4.1]{hatcher:at}).

\begin{defn}
  Die \emph{(reduzierte) Einhängung} $(\Sigma X, *)$ eines punktierten Raumes $(X, x_0)$ ist
  \[ \Sigma X \coloneqq (X \times I) / (X \times \{ 0, 1 \} \cup \{ x_0 \} \times I). \]
  Der Basispunkt $*$ von $\Sigma X$ ist der auf einen Punkt zusammengezogene Teilraum.
\end{defn}

Einhängung ist ein Endofunktor der Kategorie der punktierten topologischen Räume: \\
Für $f : (X, x_0) \to (Y, y_0)$ ist
\[
  \Sigma f : (\Sigma X, *) \to (\Sigma Y, *), \quad
  [(x, t)] \mapsto [(f(x), t)].
\]

Man sieht leicht, dass $\Sigma S^i \approx S^{i+1}$.
Somit induziert Einhängung eine Abbildung $E : \pi_i(X, x_0) \to \pi_{i+1}(\Sigma X, *)$.
Da Einhängung mit den Kogruppenstruktur von $S^i$ und $S^{i+1}$ verträglich ist, ist $E$ sogar ein Gruppenhomomorphismus.

\begin{defn}
  Ein nicht leerer topologischer Raum $X$ heißt \emph{$n$-zusammenhängend}, falls $\pi_i(X) = 0$ für $0 \leq i \leq n$.
\end{defn}

\begin{satz}[Freudenthal'scher Einhängungssatz]
  Es sei $n \geq 0$ und $X$ ein $n$-zusammenhängender CW-Komplex.
  Dann ist $E : \pi_i(X) \to \pi_{i+1}(\Sigma X)$ bijektiv für $0 \leq i \leq 2n$ und surjektiv für $i = 2n + 1$.
\end{satz}

Ein Beweis des Einhängungssatzes findet sich in \cite[\mbox{}4.24]{hatcher:at}.

Insbesondere ist die $j$-fach iterierte Einhängung $\Sigma^j X$ eines $n$-zusammenhängenden CW-Komplexes $X$ ein $(n{+}j)$-zusammenhängender CW-Komplex. (Um den Einhängungssatz mehrfach anwenden zu können, muss man sich klar machen, dass die Einhängung eines CW-Komplexes wieder ein CW-Komplex ist.)

Für einen beliebigen CW-Komplex $X$ ist $\Sigma^j X$ ein $(j{-}1)$-zusammenhängender CW-Komplex und es gilt $\pi_{n+j}(\Sigma^j X) \cong \pi_{n+j+1}(\Sigma^{j+1} X)$ für $n + j \leq 2(j-1) \Leftrightarrow n+2 \leq j$.
Somit ist
\[ \pi_{2n+2}(\Sigma^{n+2} X) \cong \pi_{2n+3}(\Sigma^{n+3} X) \cong \pi_{2n+4}(\Sigma^{n+4} X) \cong \ldots \qquad \cong \colim \pi_{n+j}(\Sigma^j X). \]
Die Gruppe $\colim \pi_{n+j}(\Sigma^j X)$ heißt \emph{$n$-te stabile Homotopiegruppe} von $X$.
Im Fall der Sphären haben wir mit $X = S^0$:
\[ \pi_{2n+2}(S^{n+2}) \cong \pi_{2n+3}(S^{n+3}) \cong \pi_{2n+4}(S^{n+4}) \cong \ldots \qquad \cong \colim_j \pi_{n+j}(S^j) =: \pi^s_n. \]
Wir behaupten, dass $E : \pi_1(S^1) \to \pi_2(S^2)$ ein Isomorphismus ist.
Aus dem Einhängungssatz folgt, dass diese Abbildung surjektiv ist.
Somit ist $\pi_2(S^2)$ ein zyklische Gruppe mit Erzeuger $\id_{S^2}$.
Es sei nun $k \in \Z$.
Dann ist $H_2(k \cdot \id_{S^2}) : H_2(S^2) \to H_2(S^2)$ gegeben durch Multiplikation mit $k$.
Angenommen, $k \cdot \id_{S^2}$ ist nullhomotop.
Dann ist auch $H_2(k \cdot \id_{S^2}) = 0$ und somit $k = 0$.
Somit ist $E$ injektiv.
Es folgt
\[ \Z \cong \pi_1(S^1) \cong \pi_2(S^2) \cong \ldots \cong \pi_j(S^j)  \cong \ldots \qquad \cong \colim_j \pi_j(S^j) = \pi^s_0. \]
Dabei ist $[\id_{S^i}]$ ein Erzeuger von $\pi_i(S^i)$ für alle $i \geq 0$.

Im Anhang dieser Arbeit befindet sich eine Tabelle mit einigen bekannten Homotopiegruppen von Sphären.
Einige Regelmäßigkeiten sind dabei offensichtlich (z. B. die gleichen Einträge in den Spalten $n{=}2$ und $n{=}3$).
Eine globale Systematik ist aber nicht erkennbar.
Tatsächlich ist das explizite Ausrechnen von Homotopiegruppen selbst von so einfachen Räumen wie den Sphären ein schweres Problem.
So sind die stabilen Homotopiegruppen $\pi_n^s$ nur für $n \leq 64$ berechnet und unbekannt für größere $n$ (vgl. \cite[384]{hatcher:at}).

Das Ziel dieser Arbeit ist es, folgenden Satz von Serre zu beweisen:

% Hatcher, 1.21
\begin{satz}[Serre, 1951]\label{serre}
  Die Homotopiegruppen $\pi_i(S^n, *)$,~$i > n$, sind endlich bis auf die Gruppen $\pi_{4k-1}(S^{2k})$, $k \geq 1$, welche jeweils isomorph zu einer direkten Summe von $\Z$ und einer endlichen Gruppe sind.
\end{satz}

Zusammengefasst werden wir also gezeigt haben, dass

\begin{kor}\mbox{}
  \vspace{-1.4em}
  \[
    \pi_i(S^n, *) \cong \begin{cases}
      0 & \text{für $i < n$ und $i > n = 1$,} \\
      \Z & \text{für $i = n$,} \\
      \Z \oplus \text{endliche Gruppe} & \text{für $i=2n-1$ und $n$ gerade,} \\
      \text{endliche Gruppe} & \text{sonst.}
    \end{cases}
  \]
\end{kor}

Der Beweis des Satzes von Serre verwendet die Serre-Spektralsequenz.
Das ist ein algebraisches Objekt, welches für eine Faserung $p : X \to B$ eine Verbindung zwischen den Homologie- und Kohomologiegruppen von Basisraum $B$, Totalraum $X$ und Faser $F$ liefert.
Wir führen zunächst Faserungen und Spektralsequenzen und die Serre-Spektralsequenz ein.
Dann verwenden wir die Serre-Spektralsequenz, um zu zeigen, dass die Homotopiegruppen von einfach zusammenhängenden topologischen Raumes genau dann endlich (erzeugt) sind, wenn die Homologiegruppen endlich (erzeugt) sind.
Dies wollen wir anwenden, um den Satz von Serre zu zeigen.
Dazu müssen wir allerdings zunächst den Raum $S^n$ so abändern, sodass die $n$-te Homotopiegruppe verschwindet, die höheren Homotopiegruppen aber erhalten bleiben.
Die Homologiegruppen dieses neuen Raumes hängen über die Serre-Spektralsequenz mit denen eines Eilenberg-MacLane-Raumes vom Typ $K(\Z, n{-}1)$ zusammen, einem Raum mit nur einer nichtverschwindenen Homotopiegruppe.
Deshalb studieren wir zunächst dessen Homologiegruppen und zeigen insbesondere, dass sie endlich erzeugt sind.

Der Aufbau dieser Arbeit orientiert sich an dem ersten Kapitel von \cite{hatcher:ss}.

\section{Faserungen}

\begin{defn}\label{def:fibration}
  Eine \emph{(Hurewicz-)\,Faserung} ist eine stetige Abbildung $p : E \to B$, welche die \emph{Homotopieliftungseigenschaft} (HLE) für die topologischen Räume $X$ besitzt, \dh{}
  für alle $n \geq 0$ und für alle stetigen Abbildungen $H$, $H_0$ wie unten, sodass das äußere Quadrat kommutiert, gibt es eine stetige Abbildung $\tilde{H}$, sodass die beiden inneren Dreiecke kommutieren:
  \begin{centertikzcd}[row sep=1.2cm, column sep=1.6cm]
    X \arrow[r, "H_0"] \arrow[d, hook, "i_0", swap] &
    E \arrow[d, "p"] \\
    X \times \I \arrow[r, "H"] \arrow[ur, "\exists\, \tilde{H}", dashed] &
    B
  \end{centertikzcd}
  Dabei ist $i_0$ die Inklusion von $X$ in $X \times \I$ als $X \times \{ 0 \}$.
  Eindeutigkeit von $\tilde{H}$ wird nicht gefordert. \\
  Der Raum $E$ wird \emph{Totalraum} und $B$ \emph{Basisraum} genannt.
\end{defn}

\begin{bem}
  Allgemeiner ist eine \emph{Serre-Faserung} ist eine stetige Abbildung $p : E \to B$, welche die Homotopieliftungseigenschaft bezüglich der Scheiben $\Set{D^n}{n \geq 0}$ oder äquivalent allen CW-Komplexen, aber nicht unbedingt allen topologischen Räume besitzt.
  Man kann viele Resultate über Hurewicz-Faserungen, wie beispielsweise die Existenz der Serre-Spektralsequenz, auch für Serre-Faserungen zeigen.
  Es bedarf aber etwas mehr an Arbeit, die grundlegenden Eigenschaften dieser Faserungen zu zeigen.
  Deshalb beschränke ich mich in dieser Arbeit auf Hurewicz-Faserungen.
\end{bem}

\begin{bsp}
  Das einfachste Beispiel einer Faserung ist die \emph{Produktfaserung}, die Projektion $p_0 : B \times F \to B$ eines Produktraumes auf einen Faktor.
  
  Seien zur Nachweis der Homotopieliftungseigenschaft ein Raum $X$ sowie $H : X \times I \to B$ und $H_0 : X \to B \times F$ wie in Definition~\ref{def:fibration} gegeben.
  Dann bringt folgende Abbildung das Diagramm zum Kommutieren:
  \[
    \widetilde{H} : X \times I \to B \times F, \quad
    (x, t) (H(x, t), \pi_2(H_0(x))).
  \]
\end{bsp}

Bei dieser Produktfaserung sind die \emph{Fasern}, das heißt die Urbilder $\Set{p^{-1}(\{b\})}{b \in B}$ von Einpunktmengen mit der Teilraumtopologie, alle homöomorph zu dem festen Raum $F$.
Allgemein sind bei Faserungen über wegzusammenhängenden Basisräumen alle Fasern zwar nicht immer homöomorph, aber homotopieäquivalent, wie das folgende Lemma zeigt.

% TODO: 4.61 in hatcher:at
\begin{lem}
  Es sei nun $p : E \to B$ eine Faserung.
  Für jeden stetigen Weg $\gamma : I \to B$ von $a \coloneqq \gamma(0)$ nach $b \coloneqq \gamma(1)$ definieren wir nun eine Abbildung $L_\gamma : p^{-1}(\{a\}) \to p^{-1}(\{b\})$.
  Wegen der HLE gibt es ein $\widetilde{H}_\gamma$, sodass
  \begin{centertikzcd}[row sep=1.2cm, column sep=1.6cm]
    p^{-1}(\{a\}) \arrow[r, hook] \arrow[d, hook, "i_0", swap] &
    E \arrow[d, "p"] \\
    p^{-1}(\{a\}) \times \I \arrow{r}{}{(f, t) \mapsto \gamma(t)} \arrow[ur, "\exists\, \tilde{H}_\gamma", dashed] &
    B
  \end{centertikzcd}
  kommutiert.
  Wir setzen
  \[
    L_\gamma : p^{-1}(\{a\}) \to p^{-1}(\{b\}), \quad
    f \mapsto \widetilde{H}_\gamma(f, 1).
  \]
  Dann gilt für stetige Wege $\gamma, \eta, \epsilon$ mit und $\gamma(1) = \epsilon(0)$:
  \begin{enumerate}[label=(\roman*)]
    \item $L_\gamma$ hängt bis auf Homotopie nicht von der Wahl des Lifts $\widetilde{H}_\gamma$ ab.
    \item Falls $\gamma \simeq \eta$ rel $\{0, 1\}$, so gilt $L_\gamma \simeq L_\eta$.
    \item $L_{\gamma \cdot \epsilon} \simeq L_\epsilon \circ L_\gamma$ (dabei ist $\gamma \cdot \epsilon$ die Verkettung von $\gamma$ und $\epsilon$)
    \item $L_\gamma$ ist eine Homotopieäquivalenz.
  \end{enumerate}
\end{lem}

\begin{proof}
  Wir halten zunächst fest, dass Faserungen auch eine relative Homotopieliftungseigenschaft erfüllen: Sei dazu das äußere Quadrat im Diagramm
  \begin{centertikzcd}[row sep=1.2cm, column sep=1.6cm]
    (X \times \I \times \{0\}) \cup (X \times \partial \I \times \I) \arrow[r, "H_0"] \arrow[d, hook, "i_0", swap] &
    E \arrow[d, "p"] \\
    X \times \I \times \I \arrow{r}{}{H} \arrow[ur, "\exists\, \tilde{H}", dashed] &
    B
  \end{centertikzcd}
  gegeben.
  Dann gibt es eine stetige Abbildung $\tilde{H}$, sodass das Diagramm kommutiert.
  Der Unterschied zu Definition~\ref{def:fibration} besteht darin, dass der Lift der Homotopie nicht nur zum Zeitpunkt $t=0$, sondern auch auf dem Rand $X \times \partial \I \times \I$ vorgegeben ist.
  Um dies zu zeigen, verwenden wir die Existenz eines Homöomorphismus $\psi : (\I \times \I, \I \times \{0\}) \to (\I \times \I, \I \times \{0\} \cup \partial \I \times \I)$ von Raumpaaren (\dh{} $\psi$ schränkt ein zu einem Homöomorphismus $\psi|_{\I \times \{0\}} : \I \times \{0\} \to \I \times \{0\} \cup \partial \I \times \I$).
  Wir benutzen nun die gewöhnliche Homotopieliftungseigenschaft mit $H' \coloneqq H \circ (\id_X \times \psi)$ und $H_0' \coloneqq H_0 \circ (\id_X \times \psi|_{\psi|_{\I \times \{0\}}})$ und erhalten einen Lift $\tilde{H}'$.
  Dann kann man leicht nachrechnen, dass $\tilde{H} \coloneqq \tilde{H}' \circ (\id_X \times \psi^{-1})$ ein geeigner Lift im obigen Diagramm ist. \\[2pt]
  \fall{Zu (i) und (ii)} Es sei $K : I \times I \to B$ eine Homotopie relativ der Endpunkte zwischen $\gamma = K|_{\{0\} \times \I}$ und $\eta = K|_{\{1\} \times \I}$ sowie $\tilde{H}_\gamma, \tilde{H}_\eta : p^{-1}(\{a\}) \times \I \to E$ gegeben.
  Wir verwenden nun die relative Homotopieliftungseigenschaft mit
  \[
    H : p^{-1}(\{a\}) \times \I \times \I \to B, \quad
    (f, t, s) \mapsto K(t, s)
  \]
  sowie
  \begin{align*}
    & H_0 : (p^{-1}(\{a\}) \times \I \times \{0\}) \cup (p^{-1}(\{a\}) \times \partial\I \times \I) \to E, \\
    & H_0(f, t, 0) \coloneqq f, \enspace
    H_0(f, 0, s) \coloneqq \tilde{H}_\gamma(f, s), \enspace
    H_0(f, 1, s) \coloneqq \tilde{H}_\eta(f, s).
  \end{align*}
  und erhalten einen Lift $\tilde{H} : (p^{-1}(\{a\}) \times \I \times \I \to E$. Die gesuchte Homotopie zwischen $L_\gamma$ und $L_\eta$ ist
  $\tilde{H}(\blank, \blank, 1) : p^{-1}(\{a\}) \times \I \to p^{-1}(\{b\})$. \\[2pt]
  \fall{Zu (iii)} Die Behauptung folgt daraus, dass man
  \[
    \tilde{H}_{\gamma \cdot \epsilon}(f, t) \coloneqq \begin{cases}
      \tilde{H}_\gamma(f, 2t) & \text{falls } 0 \leq t \leq \tfrac{1}{2}, \\
      \tilde{H}_\epsilon(\tilde{H}_\gamma(f, 1), 2t - 1) & \text{falls } \tfrac{1}{2} \leq t \leq 1
    \end{cases}
  \]
  wählen kann. \\[2pt]
  \fall{Zu (iv)} Aus (ii) und (iii) folgt, dass $L_{\gamma^{-1}}$ homotopieinvers zu $L_\gamma$ ist.
\end{proof}

\begin{bem}
  Man notiert eine Faserung $p : E \to B$ häufig als $F \to E \to B$, wobei $F$ eine typische Faser ist, also ein Raum, zu dem die Fasern $\Set{p^{-1}(\{b\})}{b \in B}$ homotopieäquivalent sind.
  Da homotopieäquivalente Räume diesselben (Ko-) Homologiegruppen besitzen, kann man von der Homologie $H_*(F; G)$ bzw. Kohomologie $H^*(F; R)$ der Faser sprechen.
\end{bem}

\begin{kor}
  Es sei $p : E \to B$ eine Faserung, $b \in B$.
  Dann ist
  \[
    \pi_1(B, b) \to \Aut(H_m(p^{-1}(\{b\}); G)), \quad
    [\gamma] \mapsto L_{\gamma*}
  \]
  eine Wirkung der Fundamentalgruppe von $B$ auf der $m$-ten Homologiegruppe der Faser.
\end{kor}

Diese Wirkung ist \emph{trivial}, falls $L_{\gamma*} = \id$ für alle geschlossenen Wege $[\gamma] \in \pi_1(B, b)$.
Zum Beispiel ist die Wirkung bei Produktfaserungen immer trivial.

\begin{bsp}
  Die Kleinsche Flasche $K$ ist der topologische Raum
  \[
    K \coloneqq (\I \times \I)/{\sim}, \quad
    (x, 0) \sim (x, 1), \enspace
    (0, y) \sim (1, 1-y).
  \]
  Sie ist Teil einer Faserung $S^1 \to K \xrightarrow{p} S^1$ mit $p[(x,y)] \coloneqq [x] \in I/_{0 \sim 1}$.
  Die Wirkung von $\pi_1(S^1)$ auf der Homologie $H_1(S^1; \Z) \cong \Z$ der Faser ist \emph{nicht} trivial, denn für $\gamma = \id : S^1 \to S^1$ ist $L_\gamma$ homotop zu einer Spiegelung von $S^1$ entlang einer Achse und induziert damit die Negationsabbildung auf $H_1(S^1; \Z)$.
\end{bsp}

% Lemma 4.5 in http://www.math.washington.edu/~mitchell/Notes/serre.pdf
% Theorem 4.41 in hatcher:at
\begin{lem}
  Es sei $p : E \to B$ eine Faserung, $b_0 \in B$, $F \coloneqq p^{-1}(b_0)$ die Faser über $b_0$ und $f_0 \in F$.
  Dann gibt es eine lange exakte Sequenz
  \[ \ldots \to \pi_n(F, f_0) \xrightarrow{i_*} \pi_n(E, f_0) \xrightarrow{p_*} \pi_n(B, b_0) \xrightarrow{\partial} \pi_{n-1}(F, f_0) \to \ldots \to \pi_1(B, b_0) \]
  von Homotopiegruppen.
  Dabei ist $i : F \hookrightarrow E$ die Inklusion.
\end{lem}

\begin{proof}
  Die gesuchte exakte Sequenz ist die lange exakte Homotopiesequenz
  \[ \ldots \to \pi_n(F, f_0) \xrightarrow{i_*} \pi_n(E, f_0) \to \pi_n(E, F, f_0) \xrightarrow{\partial} \pi_{n-1}(F, f_0) \to \ldots \to \pi_1(E, F, f_0) \]
  des Raumpaares $(E, F)$.
  Es bleibt zu zeigen: $\pi_n(E, F, f_0) \cong \pi_n(B, b_0)$ als Gruppe für $n > 1$ und als punktierte Menge für $n = 1$.
  Der Isomorphismus muss außerdem so gewählt werden, dass
  \[ p_* = \left( \pi_n(E, f_0) \to \pi_n(E, F, f_0) \xrightarrow{\cong} \pi_n(B, b_0) \right). \]
  Wir zeigen: $p_* : \pi_n(E, F, f_0) \to \pi_n(B, b_0)$ ist der gesuchte Isomorphismus (damit ist obige Gleichung erfüllt). \\[2pt]
  \emph{Surjektivität}: Sei $[g : (I^{n+1}, \partial I^{n+1}, b_0) \to (B, \{ b_0 \}, b_0)] \in \pi_{n+1}(B, b_0)$, $n \geq 0$.
  Sei $\tilde{g}$ der Lift im folgenden relativen HLE-Diagramm:
  \begin{centertikzcd}[row sep=1.2cm, column sep=1.4cm]
    U \arrow[r, "\const{f_0}"] \arrow[d, hook, swap] &
    E \arrow[d, "p"] \\
    I^n \times I \arrow[r, "g"] \arrow[ur, "\exists\, \tilde{g}", dashed] \arrow[r, "g"] &
    B
  \end{centertikzcd}
  wobei $U \coloneqq I^n \!\times\! \{ 0 \} \cup (\partial I^n) \!\times\! I \subset I^{n+1}$.
  Dann kann man $\tilde{g}$ als eine Abbildung $(I^{n+1}, \partial I^{n+1}, U) \to (E, F, \{ f_0 \})$ von Raumtripeln auffassen, welche ein Element von $\pi_{n+1}(E, F, f_0)$ repräsentiert.
  Es gilt $p_*[\tilde{g}] = [p \circ \tilde{g}] = [g]$. \\[2pt]
  \emph{Injektivität}: Seien $[h_0], [h_1] \in \pi_{n+1}(E, F, f_0)$ mit $p_*[h_0] = p_*[h_1]$.
  Sei
  \[ H : I \times I^{n+1}, \quad (t, x) \mapsto H_t(x) \]
  eine Homotopie mit $H_0 = p \circ h_0$, $H_1 = p \circ h_1$, welche zu jedem Zeitpunkt $t \in \I$ eine Abbildung $H_t : (I^{n+1}, \partial I^{n+1}) \to (B, \{ b_0 \})$ von Raumpaaren ist.
  Betrachte folgendes HLE-Diagramm:
  \begin{centertikzcd}[row sep=1.2cm, column sep=1.8cm]
    V \arrow[r, "h"] \arrow[d, hook, swap] &
    E \arrow[d, "p"] \\
    I^{n+1} \times I \arrow[r, "H"] \arrow[ur, "\exists\, \tilde{H}", dashed] &
    B
  \end{centertikzcd}
  mit $V \coloneqq I^{n+1} \!\times\! \{ 0 \} \cup (\partial I^{n+1}) \!\times\! I \subset I^{n+2}$ und
  \[
    h|_{\{0\} \times I^{n+1}} \coloneqq h_0, \quad
    h|_{\{1\} \times I^{n+1}} \coloneqq h_1, \quad
    h|_{I \times U} \coloneqq \const{f_0}.
  \]
  Nun ist $\tilde{H}$ eine Homotopie von $h_0$ nach $h_1$, welche zu jedem Zeitpunkt $t$ eine Abbildung $\tilde{H}_t : (I^{n+1}, \partial I^{n+1}, U) \to (E, F, \{ b_0 \})$ von Raumtripeln ist.
\end{proof}

\begin{bsp}
  Die \emph{Hopf-Faserung} ist eine Abbildung $\eta : S^3 \to S^2$, welche ein nichttriviales Element in $\pi_3(S^2)$ repräsentiert.
  Wenn man $S^3$ mit $\Set{(x, y) \in \C^2}{\abs{x} + \abs{y} = 1}$ identifiert und $S^2$ als Riemannsche Zahlenkugel $\C \cup \{ \infty \}$ auffasst, so ist $\eta$ gegeben durch $\eta(x, y) \coloneqq x/y$ für $y \neq 0$ und $\eta(x, 0) \coloneqq \infty$.
  Man sieht leicht, dass $\eta$ eine Faserung ist und dass die Faser
  \[ \eta^{-1}(c) = \Set{s / (\abs{c} + 1) \cdot (c, 1)}{s \in \C, \abs{s} = 1} \approx S^1 \]
  für alle $c \in \C$ ist.
  In der langen exakte Homotopiesequenz
  \[ \ldots \to \pi_n(S^1) \to \pi_n(S^3) \xrightarrow{\eta_*} \pi_n(S^2) \to \pi_{n-1}(S^1) \to \ldots \]
  dieser Faserung verschwindet für $n \geq 3$ die erste und die letzte Gruppe und $\eta$ liefert einen Isomorphismus zwischen $\pi_n(S^3)$ und $\pi_n(S^2)$.
  Das ist der Grund für die Ähnlichkeit der Spalten $n{=}2$ und $n{=}3$ der Tabelle der Homotopiegruppen im Anhang.
  Die \emph{Hopf-Invariante} $h(p) \in \Z$ ist eine Homotopieinvariante von Abbildungen $p : S^{2n-1} \to S^n$.
  Man kann zeigen, dass $h : \pi_{2n-1}(S^n) \to \Z$ ein Homomorphismus ist.
  Da $h(\eta) = 1$, besitzt $\eta$ Ordnung unendlich.
  Allgemeiner kann man zeigen, dass $2 \Z \subseteq \im(h : \pi_{2n-1}(S^n) \to \Z)$ für gerade $n$.
  Somit ist $\Z$ ein direkter Sumand von $\pi_{2n-1}(S^n)$ für gerade $n$, was einen Teil der Aussage des Satzes von Serre zeigt.
  Die Konstruktion von $\eta$ kann man auch mit den Quaternionen bzw. Oktonionen statt den komplexen Zahlen durchführen.
  Dies ergibt Faserungen
  $S^3 \to S^7 \xrightarrow{\nu} S^4$
  und
  $S^7 \to S^{15} \xrightarrow{\sigma} S^8$,
  die ebenfalls Hopf-Faserungen genannt werden.
  In der langen exakten Homotopiesequenz
  \begin{centertikzcd}
    \ldots \arrow[r] &
    \pi_7(S^3) \arrow[r] &
    \Z \cong \pi_7(S^7) \arrow[r, "\nu_*"] &
    \pi_7(S^4) \arrow[r] \arrow[l, bend left, "h"] &
    \pi_6(S^3) \arrow[r] &
    \pi_6(S^7) = 0 \arrow[r] &
    \ldots
  \end{centertikzcd}
  von $\nu$ ist $h$ eine Retraktion von $\nu_*$.
  Somit gilt $\pi_7(S^4) \cong \Z \oplus \pi_6(S^3)$.
  Die Faserung $\sigma$ induziert analog $\pi_{15}(S^8) \cong \Z \oplus \pi_{14}(S^7)$.
  Für Details siehe \cite[4.44-4.47 und §4.B]{hatcher:at}.
\end{bsp}

% Prop 4.64 in hatcher:at
\begin{lem}\label{convert-to-fibration}
  Man kann jede stetige Abbildung $f : X \to Y$ schreiben als Komposition
  \[ X \xrightarrow{i} E_f \xrightarrow{p} Y \]
  einer Homotopieäquivalenz $i$ und einer Faserung $p$.
  Genauer gilt
  \begin{align*}
    E_f & \coloneqq \Set{(x, \gamma) \in X \times Y^\I}{f(x) = \gamma(0)} \subset X \times Y^\I, \\
    i(x) & \coloneqq (x, t \mapsto f(x)), \\
    p(x, \gamma) & \coloneqq \gamma(1),
  \end{align*}
  wobei $Y^\I$ die Kompakt-Offen-Topologie trägt.
\end{lem}

% Notation wie in https://en.wikipedia.org/wiki/Homotopy_fiber
\begin{proof}
  Offensichtlich sind $i$ und $p$ stetig und es gilt $p \circ i = f$.
  Das Homotopieinverse von $i$ ist $j : E_f \to X, \enspace (x, \gamma) \mapsto x$.
  Es gilt $j \circ i = \id_X$ und eine Homotopie zwischen $i \circ j$ und $\id_{E_f}$ ist gegeben durch
  \[
    H : \I \times E_f \to E_f, \quad
    (s, (x, \gamma)) \mapsto (x, \gamma(s \cdot \blank)).
  \]
  Es bleibt zu zeigen, dass $p$ eine Faserung ist.
  Es sei dazu ein topologischer Raum $A$ und Abbildungen $H_0 : A \to E_f$ und $H : \I \times A \to Y$ mit $H \circ i_0 = p \circ H_0$ gegeben.
  Dann ist folgende Abbildung eine Homotopieliftung:
  \begin{align*}
    \tilde{H} : \I \times A & \to E_f, \\
    (s, a) & \mapsto \gamma_{s, a}, \quad
    \gamma_{s, a}(t) \coloneqq \begin{cases}
      H_0(a)(t \cdot (1+s)) & \text{falls $t \cdot (1+s) \leq 1$,} \\
      H(t \cdot (1+s) - 1, a) & \text{falls $t \cdot (1+s) \geq 1$.}
      \qedhere
    \end{cases}
  \end{align*}
\end{proof}

Wir betrachten die Inklusion $i : \{ x_0 \} \to X$ des Basispunktes.
Durch Lemma~\ref{convert-to-fibration} erhalten wir eine Faserung $F \to E_i \xrightarrow{p} X$ mit $E_i \simeq \{ x_0 \}$ zusammenziehbar.

\begin{defn}
  Der Totalraum $PX \coloneqq E_i = \Set{\gamma \in X^\I}{\gamma(0) = x_0}$ heißt \emph{Pfadraum} von $(X, x_0)$.
  Der Basispunkt von $PX$ ist der konstante Weg $\gamma_0 : \I \to X, \enspace t \mapsto x_0$.
  Die Faser von $p$ über $x_0$ ist der \emph{Schleifenraum}
  $\Omega X \coloneqq \Set{\gamma \in X^\I}{\gamma(0) = \gamma(1) = x_0}$.
  Die Faserung $\Omega X \to PX \xrightarrow{p} X$ heißt \emph{Pfadfaserung}.
\end{defn}

Die Randoperatoren aus der langen exakten $F \to PX \xrightarrow{p} X$ sind Isomorphismen $\pi_{i+1}(Y, y_0) \cong \pi_i(\Omega Y, \gamma_0)$ für $i \geq 0$, da die Homotopiegruppen von $PX$ null sind.
Die Einhängung bewirkt also eine Gradverschiebung der Homotopiegruppen.

%\begin{bem}
%  Es seien $(X, x_0)$ und $(Y, y_0)$ punktierte Räume.
%  Dann gibt es eine in $X$ und $Y$ natürliche Bijektion
%  \begin{alignat*}{4}
%    \Hom((\Sigma X, x_0), (Y, y_0)) & \enspace\cong\enspace && \Hom((X, x_0), (\Omega Y, \gamma_0)), \\
%    f & \enspace\mapsto\enspace && (x \mapsto t \mapsto f([(x, t)])), \\
%    ([(x, t)] \mapsto g(x)(t)) & \enspace\mapsfrom\enspace && g.
%  \end{alignat*}
%  Diese Bijektion stiftet auch eine Bijektion zwischen den Mengen von Abbildungen modulo basispunkterhaltender Homotopie.
%\end{bem}

\section{Spektralsequenzen}

Ein häufig verwendetes Hilfsmittel in der algebraischen Topologie sind lange exakte Sequenzen.
Sie liefern einen Zusammenhang zwischen Homologie- oder Homotopiegruppen von verschiedenen Räumen.
Wenn man genügend viele dieser Gruppen kennt, so kann man oft rein algebraisch die anderen Gruppen erschließen.
Ein Beispiel ist die lange exakte Homotopiesequenz einer Faserung.
Nun kann man sich fragen, ob es bei Faserungen auch einen Zusammenhang zwischen den Homologie- und Kohomologiegruppen von Basisraum, Totalraum und Faser gibt.
Im Jahr 1951 hat Jean-Pierre Serre in seiner Dissertation \cite{serre:thesis} gezeigt, dass es einen solchen tatsächlich gibt.
Dieser hat jedoch nicht die Form einer langen exakten Sequenz, sondern ist kodiert in einem komplexerem algebraischen Objekt, einer Spektralsequenz.

Es sei $A$ im Folgenden ein kommutativer Ring mit Eins.

\begin{defn}
  Eine (homologische) \emph{Spektralsequenz} $E$ besteht aus
  \begin{itemize}
    \item $A$-Moduln $E^r_{p,q}$ für alle $p, q \in \Z$ und $r \geq 1$,
    \item $A$-Modul-Homomorphismen $d^r_{p,q} : E^r_{p,q} \to E^r_{p-r,q+r-1}$ mit $d^r_{p-r,q+r-1} \circ d^r_{p,q} = 0$
    \item und Isomorphismen $\alpha^r_{p,q} : H_{p,q}(E^r) \!\coloneqq\! \ker(d ^r_{p,q}) / \im(d^r_{p+r,q-r+1}) \xrightarrow{\cong} E^{r+1}_{p,q}$.
  \end{itemize}
\end{defn}

\begin{bem}
  \begin{itemize}
    \item Die Homomorphismen $d^r_{p,q}$ heißen \emph{Differentiale}.
    \item Die Gesamtheit der Moduln $E^r_{p,q}$ und Differentiale $d^r_{p,q}$ mit $r \in \N$ fest heißt \emph{$r$-te Seite} $E^r$.
    \item Die Isomorphismen $\alpha^r_{p,q}$ werden in der Notation  unterdrückt und so gerechnet, als wäre $E^{r+1}_{p,q}$ gleich $H_{p,q}(E^r)$.
    \item Wenn man eine Seite einer Spektralsequenz kennt, kann man also die Einträge auf der nächsten Seite berechnen, die Differentiale jedoch im Allgemeinen nicht.
    \item Der Eintrag $E^r_{p,q}$ ist ein \emph{Subquotient} (\dh{} ein Quotientenmodul eines Untermoduls) von $E^s_{p,q}$ falls $s \leq r$.
  \end{itemize}
  Man stellt Seiten für gewöhnlich in einem 2-dimensionalen Raster dar:
  \begin{center}
    \begin{tikzpicture}[x=12pt,y=12pt]
      \begin{scope}[shift={(0,0)}]
        \foreach \x in {-1,...,4}{
          \foreach \y in {-2,-1,...,3}{
            \zeroDot{\x}{\y}
          }
        }
        \foreach \x in {-1,...,5}{
          \foreach \y in {-2,-1,...,3}{
            \draw[->,gray] (\x-0.2,\y) -- (\x-0.8,\y);
          }
        }
        \draw[->] (-0.35,-0.35) -- (5.35,-0.35) node[below] {p};
        \draw[->] (-0.35,-0.35) -- (-0.35,4) node[left] {q};
        \node at (5,4.5) {$E^1$};
      \end{scope}
      \begin{scope}[shift={(10,0)}]
        \foreach \x in {-1,...,4}{
          \foreach \y in {-2,-1,...,3}{
            \zeroDot{\x}{\y}
          }
        }
        \foreach \x in {0,...,5}{
          \foreach \y in {-2,-1,...,3}{
            \draw[->,gray] (\x-0.2,\y+0.1) -- (\x-1.8,\y+0.9);
          }
        }
        \draw[->] (-0.35,-0.35) -- (5.35,-0.35) node[below] {p};
        \draw[->] (-0.35,-0.35) -- (-0.35,4) node[left] {q};
        \node at (5,4.5) {$E^2$};
      \end{scope}
      \begin{scope}[shift={(20,0)}]
        \foreach \x in {-1,...,4}{
          \foreach \y in {-2,-1,...,3}{
            \node[draw,circle,inner sep=0.5pt,fill] at (\x,\y) {};
          }
        }
        \foreach \x in {1,...,5}{
          \foreach \y in {-2,-1,...,2}{
            \draw[->,gray] (\x-0.2,\y+0.1) -- (\x-2.8,\y+1.9);
          }
        }
        \draw[->] (-0.35,-0.35) -- (5.35,-0.35) node[below] {p};
        \draw[->] (-0.35,-0.35) -- (-0.35,4) node[left] {q};
        \node at (5,4.5) {$E^3$};
      \end{scope}
    \end{tikzpicture}
  \end{center}
\end{bem}

\begin{defn}
  Eine Spektralsequenz \emph{konvergiert}, falls für alle $p, q \in \Z$ ein $R \in \N$ existiert, sodass für alle $r \geq R$ die Differentiale von und nach $E^r_{p,q}$ verschwinden und damit
  \[ E^\infty_{p,q} \coloneqq E^R_{p,q} \cong E^{R+1}_{p,q} \cong E^{R+2}_{p,q} \cong \ldots \]
  Der Grenzwert der Spektralsequenz ist die Unendlich-Seite $E^\infty \coloneqq \{ E^\infty_{p,q} \}_{p,q \in \Z}$.
\end{defn}

Viele Spektralsequenzen sind im ersten Quadranten konzentriert, \dh{} $E^r_{p,q}$ ist nur für $p, q \geq 0$ ungleich null.
Solche Spektralsequenzen konvergieren immer, denn für alle $p, q \in \Z$ führen für $r \geq \max(p+1, q+2)$ alle Differentiale von $E^r_{p,q}$ aus dem ersten Quadranten heraus und alle dort eintreffenden Differentiale kommen von außerhalb des ersten Quadranten und sind daher null.

\begin{defn}
  Eine \emph{Filtrierung} eines $A$-Moduls $M$ ist eine aufsteigende Folge
  \[ 0 \subseteq \ldots \subseteq F^{p-1} M \subseteq F^p M \subseteq F^{p+1} M \subseteq \ldots \subseteq M \]
  von Untermoduln von $M$, $p \!\in\! \Z$.
  Eine Filtrierung heißt
  \begin{itemize}
    \item \emph{ausschöpfend}, falls $M = \bigcup_p F^p M$,
    \item \emph{Hausdorffsch}, wenn $0 = \bigcap_p F^p M$ und
    \item \emph{regulär}, wenn sie ausschöpfend und Hausdorffsch ist.
  \end{itemize}
\end{defn}

\begin{defn}
  Eine Spektralsequenz $E$ \emph{konvergiert gegen} einen graduierten $A$-Modul $M = \oplus_{n \in \Z} M_n$ (notiert $E^r_{p,q} \Rightarrow M_{p+q}$), falls $E$ überhaupt konvergiert und reguläre Filtrierungen
  \[ 0 \subseteq \ldots \subseteq F^{p-1} M_n \subseteq F^p M_n \subseteq F^{p+1} M_n \subseteq \ldots \subseteq M_n \]
  existieren, sodass $E^\infty_{p,q} \cong F^p M_{p+q} / F^{p-1} M_{p+q}$ für alle $p, q \in \Z$.
\end{defn}

In der Anwendung von Spektralsequenzen kennt man oft die Einträge der $E^1$ oder $E^2$-Seite.
Man versucht dann, die Einträge der $E^\infty$-Seite zu bestimmen, indem man sukzessive die Seiten $E^r$ ausrechnet.
Das Problem dabei ist, dass die dazu benötigten Differentiale auf diesen Seiten im Allgemeinen nicht bekannt sind.
In der Praxis hofft man darauf, dass Differentiale allein deswegen verschwinden, da ihre Quell- bzw. Zielgruppe die Nullgruppe ist.
Man kann auch bereits bekannte Information über die $E^\infty$-Seite verwenden, um Aussagen über Differentiale zu treffen, z.\,B. dass ein Differential ein Isomorphismus ist.

Angenommen, man hat auf diesem Weg die Seite $E^\infty$ einer Spektralsequenz, welche gegen $M$ konvergiert, berechnet.
Dann kennt man immer noch nicht die Moduln $M_n$, sondern lediglich die Quotienten in einer Filtrierung von $M_n$.
Im Fall, dass $A$ ein Körper ist, ist $M_n$ isomorph zur direkten Summe dieser Quotienten.
Im Allgemeinen ist das aber nicht der Fall, wie das Beispiel der Filtrierung $0 \subset 2 \Z \subset \Z$ der abelschen Gruppe $\Z$ zeigt.

Anders herum kann man Spektralsequenzen auch verwenden, um von $E^\infty$ (oder den Moduln $M_n$) auf die Seite $E^2$ bzw. $E^1$ zu schließen.
Dies werden wir tun, um die Homologie von Eilenberg-MacLane-Räumen mit Koeffizienten in $\Q$ zu bestimmen.

\section{Die Serre-Spektralsequenz}

\begin{satz}
  Es sei $F \to E \xrightarrow{p} B$ eine Faserung mit wegzusammenhängendem Basisraum $B$ und $G$ eine abelsche Gruppe.
  Angenommen, $\pi_1(B)$ wirkt trivial auf $H_*(F; G)$.
  Dann existiert eine (homologische) Spektralsequenz mit
  \[ E^2_{p,q} = H_p(B; H_q(F_p; G)), \]
  welche gegen $H_*(E; G)$ konvergiert.
\end{satz}

% Quellen:
% Spanier, Algebraic Topology: SS zum filtrierten Komplex; nur triviale Wirkung; etwas genauer als Hatcher
% Hatcher: nur triviale Wirkung
% Switzer, Homology and Homotopy: nur triviale Wirkung, nur Homologie genau beschrieben, p. 350
% May/Ponto, More Concise AT: Allgemeine Serre-Spektralsequenz (mit lokalen Koeff'en); nur ein "sketch proof"
% McCleary, User's Guide to SS: Lokale Koeffizienten, statement: p. 139, proof: p. 167ff

Die Existenz dieser Spektralsequenz wurde von Jean-Pierre Serre in seiner Doktorarbeit~\cite{serre:thesis} gezeigt.
Sie wird daher ihm zu Ehren \emph{Serre-Spektralsequenz} mit Koeffizienten in $G$ genannt.
Serre verwendet für den Beweis eine Definition von singulärer Homologie mittels Kuben anstatt von Simplizes.
Ein Beweis, der ohne dieses technische Hilfsmittel auskommt, findet sich in \cite[Thm 1.3]{hatcher:ss}.

Im Falle einer nichttrivialen Wirkung der Fundamentalgruppe existiert ebenfalls eine Serre-Spektralsequenz.
Die Einträge auf der $E^2$-Seite haben dann die Form $E^2_{p,q} = H_p(B; \LH_q(F; G))$.
Dabei ist $\LH_q(F; G)$ keine feste Gruppe, sondern ein sogenanntes \emph{lokales Koeffizientensystem}.
Dies ist eine Familie von Gruppen $(G_b)_{b \in B}$, welche in einem gewissen Sinn stetig vom Punkt $b \in B$ abhängen können.
Am Punkt $b \in B$ ist $\LH_q(F; G)_b = H_q(p^{-1}(b))$.
Ein stetiger Weg $\gamma$ zwischen Punkten $b_0, b_1 \in B$ induziert einen Isomorphismus $\gamma_* : \LH_q(F; G)_{b_0} \to \LH_q(F; G)_{b_1}$.
Dieser Isomorphismus hängt nur von der Homotopieklasse von $\gamma$ ab.
Wenn $\pi_1(B)$ trivial auf der Homologie der Faser wirkt, so hängt der Isomorphismus überhaupt nicht vom gewählten Weg ab.
Dann ist $H_p(B; \LH_q(F; G)) \cong H_p(B; H_q(F; G))$.
Für Details siehe \cite[133-185]{mccleary:ss}

Es gibt eine Version des Satzes für Serre für Kohomologie anstatt Homologie.
In Kohomologie wird eine andere Indizierung für Spektralsequenzen verwendet:

\begin{defn}
  Eine kohomologische \emph{Spektralsequenz} $E$ besteht aus
  \begin{itemize}
    \item $A$-Moduln $E_r^{p,q}$ für alle $p, q \in \Z$ und $r \geq 1$,
    \item $A$-Modul-Homomorphismen $d_r^{p,q} : E_r^{p,q} \to E_r^{p+r,q-r+1}$ mit $d_r^{p+r,q-r+1} \circ d_r^{p,q} = 0$
    \item und Isomorphismen $\alpha_r^{p,q} : H^{p,q}(E_r) \!\coloneqq\! \ker(d_r^{p,q}) / \im(d_r^{p-r,q+r-1}) \xrightarrow{\cong} E_{r+1}^{p,q}$.
  \end{itemize}
\end{defn}

Jede homologische Spektralsequenz $E$ liefert eine kohomologische Spektralsequenz, wenn man $E_r^{p,q} \coloneqq E^r_{-p,-q}$ setzt.

\begin{defn}
  Eine kohomologische Spektralsequenz $E$ \emph{konvergiert gegen} einen graduierten $A$-Modul $M = \oplus_{n \in \Z} M^n$ (notiert $E_r^{p,q} \Rightarrow M^{p+q}$), falls $E$ überhaupt konvergiert und reguläre Filtrierungen
  \[ M^n \supseteq \ldots \supseteq F^{p-1} M^n \supseteq F^p M^n \supseteq F^{p+1} M^n \supseteq \ldots \supset 0 \]
  existieren, sodass $E_\infty^{p,q} \cong F^p M^{p+q} / F^{p+1} M^{p+q}$ für alle $p, q \in \Z$.
\end{defn}

\begin{satz}
  Es sei $F \to E \xrightarrow{p} B$ eine Faserung mit wegzusammenhängendem Basisraum $B$ und $G$ eine abelsche Gruppe.
  Angenommen, $\pi_1(B)$ wirkt trivial auf $H^*(F; G)$.
  Dann existiert eine kohomologische Spektralsequenz mit
  \[ E_2^{p,q} = H^p(B; H^q(F; G)), \]
  welche gegen $H^*(E; G)$ konvergiert.
\end{satz}

Dieser Satz wird bewiesen in \cite[Thm 1.14]{hatcher:ss}.
Eine Version mit lokalen Koeffizienten findet sich in \cite[Thm 5.2]{mccleary:ss}.

Kohomologie besitzt gegenüber Homologie den Vorteil, dass die Kohomologiegruppen mit dem Cup-Produkt einen graduierten Ring bilden.
Diese zusätzliche Struktur ermöglicht es, Räume zu unterscheiden, die die gleichen Homologie- und damit auch Kohomologiegruppen besitzen.
Solch eine multiplikative Struktur existiert auch auf der Serre-Spektralsequenz in Kohomologie:

\begin{satz}
  Es sei $F \to E \xrightarrow{p} B$ eine Faserung mit wegzusammenhängendem Basisraum $B$ und $R$ ein Ring.
  Angenommen, $\pi_1(B)$ wirkt trivial auf $H^*(F; R)$.
  Es sei $E$ die Serre-Spektralsequenz der Faserung in Kohomologie mit Koeffizienten in $R$.
  Dann gibt es bilineare Abbildungen
  \[
    m_r = m_r^{p,q,s,t} \enspace:\enspace E^{p,q}_r \times E^{s,t}_r \to E^{p+s,q+t}_r, \quad
    (x, y) \mapsto m_r(x, y) =: xy
  \]
  mit folgenden Eigenschaften:
  \begin{enumerate}[label=(\roman*)]
    \item $d_r$ ist derivativ: $d_r^{p+s,q+t}(xy) = (d_r^{p,q} x) y + (-1)^{p+q} x (d_r^{s,t} y)$ für alle $x \in E^{p,q}_r$ und $y \in E^{s,t}_r$
    \item Es gilt $m_{r+1}([x], [y]) = [m_r(x, y)]$ für alle $x \in \ker(d_r^{p,q})$ und $y \in \ker(d_r^{s,t})$. \\
    Dabei ist $m_r(x, y) \in \ker(d_r^{p+s,q+t})$ wegen (i).
    \item Auf der $E_2$-Seite ist $m_2 : E_2^{p,q} \!\times\! E_2^{s,t} \!\to\! E_2^{p+s,q+t}$ das $(-1)^{qs}$-fache des Cup-Produkts
    \[
      H^p(B; H^q(F; R)) \times H^s(B; H^t(F; R)) \to H^{p+s}(B; \LH^{q+t}(F; R)),
    \]
    welches für $a = [(a_\sigma)_{\sigma \in \Delta_p(B)}] \in H^p(B; H^q(F; R))$ und $b = [(b_\sigma)_{\sigma \in \Delta_s(B)}] \in H^s(B; H^t(F; R))$ definiert ist durch
    \[
      (a \cup b)_{\sigma \in \Delta_{p+s}(B)} \coloneqq
      a_{\sigma_{\angles{e_0, \ldots, e_p}}} \cup b_{\sigma_{\angles{e_p, \ldots, e_{p+s}}}}.
    \]
    \item Das Cup-Produkt auf $H^*(B; R)$ respektiert die Filtrierungen
    \[ H^n \coloneqq H^n(B; R) \supset \ldots \supset F^{p-1} H^n \supset F^p H^n \supset F^{p+1} H^n \supset \ldots \supset 0 \]
    und schränkt daher ein zu Abbildungen $F^p H^m \times F^s H^n \to F^{p+s} H^{m+n}$.
    Die induzierte Abbildung auf dem Quotienten $F^p H^m/F^{p+1} H^m \times F^s H^n/F^{s+1} H^n \to F^{p+s} H^{m+n} / F^{p+s+1} H^{m+n}$ entspricht dem Grenzwert $m_\infty^{p,m-p,s,n-s} : E_\infty^{p,m-p} \times E_\infty^{s,n-s} \to E_\infty^{p+s,m+n-p-s}$ der Multiplikationen $m_r^{p,m-p,s,n-s}$.
    Dieser Grenzwert existiert aufgrund von (ii).
  \end{enumerate}
\end{satz}

Ein Beweis dieses Satzes wird in \cite[Abschnitt 5.3]{mccleary:ss} geführt.

\section{Töten von Homotopiegruppen und Eilenberg-MacLane-Räume}

Beim Studium von topologischen Räumen ist es oft sinnvoll, den betrachteten Raum etwas abzuändern, sodass er leichter zu untersuchen sind, indem man z.\,B. bestimmte Sätze anwendet, und dann die Ergebnisse auf den ursprünglichen Raum überträgt.
Eine Möglichkeit einen Raum abzuändern ist es, bestimmte Homotopiegruppen zu null zu machen, zu "`töten"'.
Der Raum wird damit aus Sicht der Homotopietheorie einfacher, aber geometrisch und aus Sicht der Homologietheorie komplexer.

Es sei dazu $X$ ein zusammenhängender CW-Komplex.
Wähle Erzeuger $\Set{\phi_\alpha : S^n \to X}{\alpha \in A}$ von $\pi_n(X)$.
Benutze diese Abbildungen, um einen neuen CW-Komplex $X'$ aus $X$ durch Ankleben von $n{+}1$-Zellen $e_\alpha^{n+1}$ zu konstruieren.
Mit anderen Worten ist folgendes Diagramm ist ein Pushout:

\begin{centertikzcd}
  S^n \times A \arrow[d, swap, hook, "i \times \id_A"] \arrow[r, "\amalg \, \phi_\alpha"] &
  X \arrow[d, hook, "i"] \\
  D^{n+1} \times A \arrow[r] &
  X' \arrow[ul, phantom, "\ulcorner", very near start]
\end{centertikzcd}

\begin{lem}
  Die Abbildungen $\pi_j(X \xhookrightarrow{i} X')$ sind Isomorphismen für $j < n$ und $\pi_n(X') = 0$.
\end{lem}

\begin{proof}
  Aus zellulärer Approximation (siehe \cite[\mbox{}4.8]{hatcher:at}) folgt:
  \begin{itemize}
    \item Für alle $x \in \pi_j(X')$, $j \leq n$ gibt es eine Abbildung $f : S^n \to X$ mit $x = [i \circ f]$.
    \item Falls für zwei Abbildungen $f, g : S^j \to X$, $j < n$ eine basispunkterhaltende Homotopie $H$ zwischen $i \circ f : S^j \to X'$ und $i \circ g$ existiert, so auch zwischen $f$ und $g$.
  \end{itemize}
  Somit ist $\pi_j(X \xhookrightarrow{i} X')$ bijektiv für $j < n$ und surjektiv für $j = n$.
  Um zu sehen, dass $\pi_n(X') = 0$, reicht es zu zeigen, dass $\pi_n(i)[\phi_\alpha] = [i \circ \phi_\alpha] = 0$ für alle $\alpha \in A$.
  Das gilt nach Konstruktion, da die charakteristische Abbildung $\Phi : e^{n+1}_\alpha \to X'$ eine Fortsetzung von $i \circ \phi_\alpha$ auf $D^{n+1}$ ist.
\end{proof}

Wir haben jedoch keine Kontrolle über die höheren Homotopiegruppen $\pi_j(X)$, $j > n$.
Diese können von der Wahl der Erzeuger $\phi_\alpha$ abhängen.
Wir können aber mit dem Verfahren fortfahren, und diese Gruppen ebenfalls zu Null machen.
Dazu führen wir obige Konstruktion durch und erhalten eine Folge
\[ \ldots = X_{n-2} = X_{n-1} \coloneqq X \subseteq X' =: X_n \subseteq X_{n+1} \subseteq X_{n+2} \subseteq \ldots \]
von CW-Komplexen, für die gilt:
\begin{itemize}
  \item $X_m$ entsteht aus $X_{m-1}$ durch Ankleben von $(m{+}1)$-Zellen.
  \item $\pi_j(X_m \hookrightarrow X_k) : \pi_j(X_m) \to \pi_j(X_k)$ ist ein Isomorphismus für $j \leq m \leq k$.
  \item $\pi_j(X_m) = 0$ für $n \leq j \leq m$
\end{itemize}
Wir setzen $\trunc{< n}{X} \coloneqq \trunc{\leq n+1}{X} \bigcup_{j=n}^\infty X_j$.

\begin{lem}
  Die Inklusion $i : X \to \trunc{< n}{X}$ induziert Isomorphismen $\pi_j(X) \to \pi_j(\trunc{< n}{X})$ für $j < n$.
  Es gilt $\pi_j(\trunc{< n}{X}) = 0$ für $j \geq n$.
\end{lem}

\begin{proof}
  Der Satz über zelluläre Approximation impliziert:
  \begin{itemize}
    \item Jedes Element $x \in \pi_j(\trunc{< n}{X})$ wird repräsentiert durch eine Abbildung $f : S^j \to X_j$.
    \item Falls $i \circ f \sim i \circ g$ für $f, g : S^j \to X_j$, so gilt auch $f \sim g$.
  \end{itemize}
  Das zeigt die Behauptung, da $\pi_j(X_j) = 0$ für $j \geq n$.
\end{proof}

Der Raum $\trunc{< n}{X}$ hängt auch nicht (bis auf Homotopie) von den Wahlen der Erzeuger ab:
Es seien dazu $Y, Y' \supset X$ zwei durch obiges Verfahren aus $X$ konstruierte Räume.
Dann kann man die Inklusionsabbildung $i : X \hookrightarrow Y'$ fortsetzen zu einer Abbildung $f : Y \to Y'$, indem man induktiv $f$ auf das $j$-Skelett von $Y$, $j > n$ fortsetzt.
Dies ist möglich, da für eine $j$-Zelle $e_\beta^j \subset Y$ die Einschränkung $f|_{\partial e_\beta^j}$ nullhomotop ist.
Die Abbildung $f$ ist dann eine Homotopieäquivalenz nach dem Whitehead-Theorem (siehe \cite[Thm 4.5]{hatcher:at}), da sie Isomorphismen auf allen Homotopiegruppen induziert.

Nun können wir mit Lemma \ref{convert-to-fibration} die Abbildung $i : X \to \trunc{< n}{X}$ in eine Faserung $F \to E_i \to \trunc{< n}{X}$ mit $E_i \simeq X$ umwandeln.
Anhand der langen exakten Sequenz von Homotopiegruppen sieht man, dass $\pi_j(F \to E_i)$ für $j \geq n$ ein Isomorphismus ist und dass $\pi_j(F) = 0$ für $j < n$.
Wir haben es also geschafft, einen $(n{-}1)$-zusammenhängenden Raum $\trunc{\geq n}{X} \coloneqq \trunc{> n-1}{X} \coloneqq F$ zu konstruieren, der im Grad $j \geq n$ diesselben Homotopiegruppen wie $X$ besitzt.
Mit anderen Worten: Wir haben die Homtopiegruppen unterhalb Grad $n$ getötet.
Die Faserung $\trunc{\geq n}{X} \to X' \to \trunc{< n}{X}$ mit $X' \coloneqq E_i \simeq X$ wird uns im Folgenden erlauben, $X$ zu untersuchen, indem wir $\trunc{\geq n}{X}$ und $\trunc{< n}{X}$ getrennt studieren.
Wir werden insbesondere die Faserung $\trunc{> n}{S^n} \to S^n \to \trunc{\leq n}{S^n}$ verwenden, um die Homotopiegruppen von $S^n$ zu untersuchen.
Der Raum $\trunc{\leq n}{S^n}$ besitzt nur eine nicht verschwindende Homotopiegruppe, nämlich $\pi_n(\trunc{\leq n}{S^n}) \cong \pi_n(S^n) \cong \Z$.
Solche Räume spielen eine wichtige Rolle in der Homotopietheorie und haben deshalb einen eigenen Namen:

\begin{defn}
  Es sei $G$ eine Gruppe und $n \geq 1$.
  Ein \emph{Eilenberg-MacLane-Raum} vom Typ $K(G, n)$ ist ein punktierter, zusammenhängender topologischer Raum $(X, x_0)$ mit
  \[
    \pi_q(X, x_0) = \begin{cases}
      G & \text{falls $q = n$,} \\
      0 & \text{falls $q \neq n$.}
    \end{cases}
  \]
\end{defn}

Mit anderen Worten ist $\trunc{\leq n}{S^n}$ ein $K(\Z, n)$.
Man kann zeigen (siehe \cite[365-366]{hatcher:at}):

\begin{satz}
  Sei $n \geq 1$ und $G$ eine Gruppe, abelsch für $n \geq 2$.
  Dann existiert ein CW-Komplex $(X, x_0)$ vom Typ $K(G, n)$.
  Der CW-Komplex $X$ ist eindeutig bis auf Homotopieäquivalenz, \dh{} ist $X'$ ein weiterer CW-Komplex vom Typ $K(G, n)$, so gilt $X \simeq X'$.
\end{satz}

\begin{bem}
  Es sei $(X, x_0)$ ein $K(G, n)$. Dann ist $\Omega X$ ein $K(G, n-1)$, denn
  \[
    \pi_q(\Omega X, \gamma_0) \cong
    \pi_{q+1}(X, x_0) \cong
    \begin{cases}
      G & \text{falls $q+1 = n \iff q = n-1$,} \\
      0 & \text{falls $q+1 \neq n \iff q \neq n-1$.}
    \end{cases}
  \]
  Die Homologiegruppen von Räumen vom Typ $K(G, n)$ kann man über die Pfadfaserung $K(G, n{-}1) \to P K(G, n) \to K(G, n)$ untersuchen.
\end{bem}

\section{Das Mod-$\SC$-Hurewicz-Theorem}

Es sei $(X, x_0)$ ein punktierter topologischer Raum.
Der Hurewicz-Homomorphismus $h_n : \pi_n(X, x_0) \to H_n(X; \Z)$ verläuft zwischen der $n$-ten Homotopiegruppe und der $n$-ten Homologiegruppe von $X$.
Er ist definiert durch $h_n([f]) \coloneqq H_n(f)(\alpha)$ für einen fest gewählten Erzeuger $\alpha \in H_n(S^n; \Z)$.
In diesem Abschnitt werden wir folgenden klassischen Satz verallgemeinern:

\begin{satz}[Hurewicz]
  Sei $(X, x_0)$ ein $(n{-}1)$-zusammenhängender topologischer Raum, \dh{} $\pi_i(X, x_0) = 0$ für $i < n$.
  Dann ist $h_i : \pi_i(X, x_0) \to H_i(X; \Z)$ ein Isomorphismus für $0 < i \leq n$.
  Insbesondere gilt $H_i(X; \Z) = 0$ für $0 < i < n$.
\end{satz}

Ein Beweis dieses Satzes wird in~\cite[\mbox{}4.32]{hatcher:at} geführt.

\begin{defn}\label{serre-class}
  Eine Klasse $\SC$ von abelschen Gruppen heißt \emph{Serre-Klasse}, falls
  \begin{enumerate}[label=(\Roman*)]
    \item Für jede kurze exakte Seq. $0 \to A \to B \to C \to 0$ von ab. Gruppen gilt: $B \in \SC \Leftrightarrow A, C \in \SC$.
    \item Für $A, B \in \SC$ sind auch $A \otimes B \in \SC$ und $\Tor(A, B) \in \SC$.
  \end{enumerate}
\end{defn}

\begin{bem}
  Aus der ersten Eigenschaft folgt, dass Bilder, Untergruppen und Quotienten einer Gruppe aus $\SC$ wieder in $\SC$ sind.
  Genauer gilt für eine ab. Gruppe $B$ und eine Untergruppe $A < B$: $B \in \SC \iff A, B/A \in \SC$.
  Durch Induktion kann man zeigen, dass für eine Gruppe $A$ mit endlicher Filtrierung
  $A = F^0 A \supseteq F^1 A \supseteq \ldots \supseteq F^k A = 0$
  gilt: $A \in \SC \iff F^0 A / F^1 A, \ldots, F^{k-1} A / F^k A \in \SC$.
  Außerdem liegt die direkte Summe zweier Gruppen aus $\SC$ wieder in $\SC$.
\end{bem}

\begin{defn}
  Es sei $\SC$ eine Serre-Klasse.
  Ein Morphismus $f : A \to B$ zwischen abelschen Gruppen heißt \emph{Isomorphimus modulo $\SC$}, falls $\ker(f), \coker(f) \in \SC$. \\
\end{defn}

\vspace{-1.5em}
\begin{bem}
  Dies ist äquivalent zur Existenz einer exakten Sequenz $K \to A \xrightarrow{f} B \to C$ mit $K, C \in \SC$.
  Eine Gruppe, welche modulo-$\SC$-isomorph zu einer Gruppe aus $\SC$ ist, ist selbst in $\SC$.
\end{bem}

\begin{lem}
  Folgende Klassen sind Serre-Klassen:
  \begin{enumerate}[label=\alph*)]
    \item $\T_P \coloneqq \{ \text{endliche abelsche Gruppen, deren Ordnung nur durch Primzahlen in $P \subseteq \Primes$ teilbar ist} \}$, wobei $\Primes$ die Menge aller Primzahlen bezeichnet,
    \item $\F \coloneqq \T_\Primes = \{\, \text{endliche abelsche Gruppen} \,\}$
    \item $\FG \coloneqq \{\, \text{endlich erzeugte abelsche Gruppen} \,\}$
  \end{enumerate}
\end{lem}

\begin{proof}
  \emph{a)} \enspace Es sei $0 \to A \xrightarrow{f} B \xrightarrow{g} C \to 0$ eine kurze exakte Sequenz von abelschen Gruppen.
  Wir behaupten, dass $\abs{g^{-1}(c)} = \abs{A}$ für alle $c \in C$.
  Für $c = 0$ gilt die Aussage wegen Exaktheit.
  Sei nun $c \in C$ beliebig.
  Wähle ein Urbild $a \in g^{-1}(c)$.
  Dann ist Rechtsmultiplikation mit $a^{-1}$ eine Bijektion $x \mapsto x \cdot a^{-1} : g^{-1}(c) \to g^{-1}(0)$ mit Umkehrabbildung $x \mapsto x \cdot a$.
  Somit ist $\abs{g^{-1}(c)} = \abs{g^{-1}(0)} = \abs{A}$.
  
  Also gilt $\ord{A} \cdot \ord{C} = \ord{B} \in \N \cup \{ \infty \}$.
  Daraus folgt Axiom (I).
  
  Es bleibt, (II) zu zeigen.
  Für $\Z_n, \Z_m \in \T_P$ (\dh{} $n$ und $m$ sind Produkte von Primzahlen aus $P$) gilt $\Z_n \otimes \Z_m \cong \Tor(\Z_n, \Z_m) \cong \Z_{\ggT(n, m)} \in \T_P$.
  Wegen des Hauptsatzes über endlich erzeugte abelsche Gruppen kann jede endliche abelsche Gruppe als endliche direkte Summe von zyklischen Gruppen schreiben.
  Es seien daher nun $A = \bigoplus_{i=1}^N \Z_{n_i}$ und $B = \bigoplus_{j=1}^M \Z_{m_j}$.
  Dann ist $\Tor(A, B) \cong A \otimes B \cong \bigoplus_{i=1}^N \bigoplus_{j=1}^M \Z_{\ggT(n_i, m_j)}$ ebenfalls endlich.

  \emph{c)} \enspace Wir betrachten nun wieder die kurze exakten Sequenz $0 \to A \xrightarrow{f} B \xrightarrow{g} C \to 0$ von abelschen Gruppen.
  Falls $B$ endlich erzeugt ist, so ist $C$ offensichtlich endlich erzeugt und $A$ ist ebenfalls endlich erzeugt als Untergruppe einer endlich erzeugten abelschen Gruppe.
  Es seien nun $A$ und $C$ endlich erzeugt, $A = \angles{a_1, \ldots, a_N}$ und $C = \angles{c_1, \ldots, c_M}$.
  Dann ist $\ker(g) = \angles{f(a_1), \ldots, f(a_N)}$.
  Wähle für $j = 1, \ldots, M$ ein Element $b_j \in g^{-1}(c_j)$.
  Wir behaupten, dass $B = \angles{f(a_1), \ldots, f(a_N), b_1, \ldots, b_M}$.
  Sei dazu $b \in B$ beliebig.
  Schreibe $g(b) = m_1 \cdot c_1 + \ldots + m_M \cdot c_M$ mit $m_1, \ldots, m_M \in \Z$.
  Dann ist $\tilde{b} \coloneqq b - (m_1 \cdot b_1 + \ldots + m_M \cdot b_M) \in \ker(g)$, \dh{} es gibt $n_1, \ldots, n_N \in \Z$ mit $\tilde{b} = n_1 \cdot f(a_1) + \cdots + n_N \cdot f(a_N)$.
  Somit $b = m_1 \cdot c_1 + \ldots + m_M \cdot c_M + n_1 \cdot a_1 + \ldots n_N \cdot a_N$.
  Damit ist der erste Teil der Definition gezeigt.
  Für den zweiten Teil seien $A$ und $B$ endlich erzeugte abelsche Gruppen und $n, n_1, \ldots, n_N \in \N$ und $m, m_1, \ldots, m_M \in \N$ mit $A \cong \Z^n \oplus \Z_{n_1} \oplus \ldots \oplus \Z_{n_N}$ und $B \cong \Z^m \oplus \Z_{m_1} \oplus \ldots \oplus \Z_{m_M}$.
  Es gilt $A \otimes B = \Z^{nm} \oplus \left( \Z_{m_1} \oplus \ldots \oplus \Z_{m_M} \right)^n \oplus \left( \Z_{n_1} \oplus \ldots \oplus \Z_{n_N} \right)^M \oplus \bigoplus_{i=1}^N \bigoplus_{j=1}^M \Z_{\ggT(n_i, m_j)}$
  und $\Tor(A, B) = \bigoplus_{i=1}^N \bigoplus_{j=1}^M \Z_{\ggT(n_i, m_j)}$, es sind also $A \otimes B$ und $\Tor(A, B)$ ebenfalls endlich erzeugt.
\end{proof}

\begin{defn}
  Es sei $\SC$ eine Serre-Klasse.
  Ein topologischer Raum~$X$ heißt \emph{$\SC$-azyklisch}, falls $\widetilde{H}_n(X) \in \SC$ für alle $n \geq 0$.
\end{defn}

\begin{axiom}
  Es sei $\SC$ eine Serre-Klasse.
  \begin{enumerate}[label=(\Roman*)]
    \setcounter{enumi}{2}
    \item Es sei $G \in \SC$.
    Dann ist $K(G, m)$ $\SC$-azyklisch für alle $m \geq 1$.
  \end{enumerate}
\end{axiom}

% Hatcher, 1.8
\begin{satz}[Mod-$\SC$-Hurewicz-Theorem]\label{hurewicz-mod-c}
  Es sei $\SC$ eine Serre-Klasse, die (III) erfüllt, und
  $(X, x_0)$ ein einfach zusammenhängender topologischer Raum.
  Angenommen, $\pi_i(X, x_0) \in \SC$ für $0 < i < n$.
  Dann ist $h_i : \pi_i(X, x_0) \to H_i(X; \Z)$ ein Isomorphismus modulo $\SC$ für $i \leq n$.
  Insbesondere gilt $H_i(X; \Z) \in \SC$ für $0 < i < n$.
\end{satz}

Mit $\SC \coloneqq \{ 0 \}$ ist die Aussage des Satzes gerade die des gewöhnlichen Hurewicz-Theorems.
Durch Induktion kann man zeigen:

% Hatcher, 1.7 (etwas allgemeiner)
\begin{kor}\label{homotopy-in-c-iff-homology-in-c}
  Es sei $\SC$ eine Serre-Klasse, die (III) erfüllt, und
  $(X, x_0)$ ein einfach zusammenhängender topologischer Raum.
  Dann gilt für alle $N \in \N \cup \{ \infty \}$:
  \[
    \pushQED{\qed} 
    \fa{0 \leq n < N} \pi_n(X, x_0) \in \SC
    \quad \iff \quad
    \fa{1 \leq n < N} H_n(X; \Z) \in \SC.
    \qedhere
    \popQED
  \]
\end{kor}

Aus dem Korollar folgt, dass die Homotopiegruppen der Sphären alle endlich erzeugt sind, da die Homologiegruppen der Sphären endlich erzeugt sind.

\begin{lem}\label{two-of-three}
  Es sei $F \to X \to B$ eine Faserung von wegzusammenhängenden Räumen $F$, $X$ und $B$.
  Es wirke $\pi_1(B)$ trivial auf $H_*(F)$.
  Dann gilt folgende $2$-aus-$3$-Eigenschaft: Falls zwei der Räume $F$, $X$ und $B$ $\SC$-azyklisch sind, so auch der dritte.
\end{lem}

\begin{proof}
  Wir betrachten die Serre-Spektralsequenz zu der Faserung mit Koeffizienten in $\Z$.
  Zunächst gilt $H_n(X) \in \SC$ genau dann, wenn die Gruppen $E^\infty_{i,n-i}$ für $i = 0, \ldots, n$ alle in $\SC$ liegen, denn diese Gruppen sind die Quotienten einer endlichen Filtrierung von $H_n(X)$ (vgl. Bemerkung nach Definition~\ref{serre-class}).

  \fall{Fall 1: $F$ und $B$ sind $\SC$-azyklisch}
  Die universelle Koeffizientenformel liefert
  \[ E^2_{p,q} \cong H_p(B; H_q(F; \Z)) \cong \left( H_p(B; \Z) \otimes H_q(F; \Z) \right) \oplus \Tor(H_{p-1}(B; \Z), H_q(F; \Z)). \]
  Man sieht durch Unterscheidung der Fälle $p=0$, $p=1$ und $p>1$ sowie $q=0$ und $q > 0$, dass $E^2_{p,q} \in \SC$ für $(p, q) \neq (0, 0)$.
  Als Subquotient von $E^2_{p,q}$ liegt nun auch $E^\infty_{p,q}$ in $\SC$ für $(p, q) \neq (0, 0)$.
  Dies zeigt die Behauptung nach der Bemerkung am Anfang des Beweises.

  \fall{Fall 2: $F$ und $X$ sind $\SC$-azyklisch}
  Wir zeigen nun durch Induktion über $k$, dass $H_p(B) \in \SC$ für $0 < p < k$.
  %Dies gilt trivialerweise für $k = 1$.
  Gelte dies für $k \geq 1$.
  Wir wollen zeigen, dass dann auch $H_k(B)$ in $\SC$ liegt.
  %Zunächst sieht man wie im Fall 1, dass $E^2_{p,q}$ und somit auch $E^r_{p,q}$ in $\SC$ für $(p, q) \neq (0, 0)$ und $p < k$.
  Für alle $r \geq 2$ gibt es eine kurze exakte Sequenz
  \begin{centertikzcd}[column sep=0.4cm, row sep=0.35cm]
    0 \arrow[r] &
    \ker(d^r_{k,0}) \arrow[d, equal] \arrow[r] &
    E^r_{k,0} \arrow[r, "d^r_{k,0}"] &
    \im(d^r_{k,0}) \arrow[d, "\subseteq"] \arrow[r] &
    0 \\
    & E^{r+1}_{k,0} &&
    E^r_{k-r,r-1}
  \end{centertikzcd}
  Man sieht unter Verwendung der Induktionsannahme und der universellen Koeffizientenformel wie in Fall~1, dass $E^2_{k-r,r-1}$ und somit auch $E^r_{k-r,r-1}$ in $\SC$ liegen.
  Damit gilt auch $\im(d^r_{k,0}) \in \SC$.
  Aus der ersten Eigenschaft von Serre-Klassen folgt $E^r_{k,0} \in \SC \iff E^{r+1}_{k,0} \in \SC$ für alle $r \geq 1$.
  Da aber $E^R_{k,0} \cong E^\infty_{k,0} \in \SC$ für $R$ groß genug, gilt $E^r_{k,0} \in \SC$ für alle $r \geq 2$.
  Insbesondere $H_k(B; \Z) \cong H_k(B; H_0(F; \Z)) \cong E^2_{k,0} \in \SC$.

  \fall{Fall 3: $B$ und $X$ sind $\SC$-azyklisch}
  Analog zum vorherigen Fall zeigt man induktiv, dass $H_q(F) \in \SC$ für $0 < q < k$.
  Dazu verwendet man die kurze exakte Sequenz $0 \to \im(d^r_{r,k-r+1}) \hookrightarrow E^r_{0,k} \to E^{r+1}_{0,k}$.
\end{proof}

% 1.10 im Hatcher
\begin{lem}\label{homology-kgn-in-c}
  $\FG$ und $\F$ erfüllen Axiom (III).
\end{lem}

\begin{proof}
  Es sei zunächst $n=1$.
  \begin{itemize}
    \item Falls $G = \Z$, so stimmt die Aussage, denn der Kreis $S^1$ ist ein $K(\Z, 1)$ und $H_*(S^1; \Z) \in \FG$.
    \item Falls $G = \Z_m$, so kann man einen $K(\Z_m, 1)$ als unendlich-dimensionalen Linsenraum $L_m$ konstruieren.\footnote{Sei dazu $S^\infty \subset \C^\infty$ die unendlich-dimensionale Sphäre. Es wirke $\Z_m$ auf jedem Faktor $\C$ von $\C^\infty$ durch Rotation $(k, x) \mapsto x \cdot e^{\nicefrac{2 \pi i k}{m}}$. Dies induziert eine eigentlich diskontinuierliche Gruppenwirkung $\Z_m \to \Aut(S^\infty)$. Dann ist $L_m \coloneqq S^\infty / \Z_m$ ein $K(\Z_m, 1)$: Die Überlagerung $S^\infty \to S^\infty / \Z_m$ hat Decktransformationsgruppe~$\Z_m$ und ist universell, da $S^\infty$ zusammenziehbar ist (vgl. \cite[\mbox{}1B.3-1B.4]{hatcher:at}). Somit ist $\pi_1(L_m) \cong \Z_m$. Für die höheren Homotopiegruppen gilt $\pi_n(L_m) \cong \pi_n(S^\infty) = 0$.}
    Dieser besitzt Homologiegruppen $\tilde{H}_i(L_m) \cong \Z_m$ für $i > 0$ ungerade und $\tilde{H}_i(L_m) \cong 0$ sonst (vgl. \cite[\mbox{}2.43]{hatcher:at}).
    \item Falls $G = G_1 \oplus G_2$, dann ist $K(G_1, 1) \times K(G_2, 1)$ ein $K(G, 1)$.
    Wenn die Aussage für $G_1$ und $G_2$ stimmt, so folgt aus dem letzten Lemma, angewendet auf die Produktfaserung $K(G_1, 1) \to K(G_1, 1) \times K(G_2, 1) \to K(G_2, 1)$, dass sie auch für $G$ gilt.
  \end{itemize}
  Da man jede endlich erzeugte abelsche Gruppe als direkte Summe von endlich vielen Summanden der Form $\Z$ und $\Z_m$ schreiben kann, gilt die Aussage für $n=1$.

  Induktiv zeigen wir nun, dass die Aussage für beliebiges $n$ gilt.
  Dazu verwenden wir die Pfadfaserung $K(G, n) \to P \to K(G, n{+}1)$.
  Es gilt $H_k(P) = 0 \in \SC$ und $H_k(K(G, n)) \in \SC$ für $k \geq 1$ nach Induktionshypothese, also $H_k(K(G, n{+}1)) \in \SC$ für alle $k \geq 1$ nach dem vorhergehenden Lemma.
\end{proof}

Für den Beweis des nächsten Lemmas benötigen wir einen weiteren Begriff.

\renewcommand\windowpagestuff{
  \vspace{1cm}
  \begin{centertikzcd}[ampersand replacement=\&, column sep=0.2cm, row sep=0.5cm]
    \&\&\& X \arrow[dll] \arrow[dl] \arrow[d] \\
    \dots \arrow[r] \&
    X_3 \arrow[r] \&
    X_2 \arrow[r] \&
    X_1
  \end{centertikzcd}
}
\opencutright
\begin{cutout}{0}{\dimexpr\linewidth-4.5cm\relax}{0pt}{2}
  \begin{defn}
    Ein \emph{Postnikov-Turm} eines wegzusammenhängenden Raumes $X$ ist ein kommutatives Diagramm wie rechts, für das gilt: \\[3pt]
    %\begin{itemize}
    \quad$\bullet$\enspace $\pi_i(X \to X_n)$ ist ein Isomorphismus für $i \leq n$ und \\
    \quad$\bullet$\enspace $\pi_i(X_n) = 0$ für $i > n$.
    %\end{itemize}
  \end{defn}
\end{cutout}

Postnikov-Türme kann man folgendermaßen konstruieren:
Setze $X_n \coloneqq \trunc{\leq n}{X}$.
Dabei wird $\trunc{\leq n}{X}$ wie im letzten Abschnitt beschrieben aus $X$ durch Ankleben von Zellen der Dimension $\geq n+2$ konstruiert, sodass die Homotopiegruppen in Grad $> n$ verschwinden.
Die Abbildung $i : X \to X_n$ ist die Inklusion.
Diese lässt sich nun fortsetzen zu einer Abbildung $f : X_{n+1} \to X_n$:
Sei dazu induktiv eine Fortsetzung $f^k : \sk_k X_{n+1} \cup X \to X_n$, $k \geq n+1$ von $i$ gegeben.
Dann lässt sich die Abbildung auch auf $(k{+}1)$-Zellen $e^{k+1}_\alpha$ ausdehnen, denn $f^k|_{\partial e^{k+1}_\alpha}$ ist nullhomotop, da $\pi_k(X_n) = 0$ für $k > n$.

\begin{bem}
  Es sei ein Postnikov-Turm $\ldots \to X_2 \to X_1$ gegeben.
  Dann kann man durch wiederholtes Anwenden der in Lemma~\ref{convert-to-fibration} beschriebenen Konstruktion einen neuen Postnikov-Turm $\ldots \to X_2' \to X_1'$ und Homotopieäquivalenzen $X_i \simeq X_i'$ konstruieren, sodass die Abbildungen $X_{i+1}' \to X_i'$ Faserungen sind.
  Man sieht anhand der langen exakten Sequenz von Homotopiegruppen, dass die Faser von $X_{i+1}' \to X_i'$ ein $K(\pi_{n+1}(X), n{+}1)$ ist.
\end{bem}

Man kann daher einen Postnikov-Turm als eine Zerlegung eines topologischen Raumes als Eilenberg-MacLane-Räume betrachten.
Dadurch wird es möglich, induktiv über den Raum $X$ zu argumentieren oder seine Homologie zu bestimmen, wenn man Kenntnis über die Bestandteile $K(\pi(n), n)$ besitzt.
Das wird im Beweis des nächsten Lemmas deutlich.

\begin{lem}\label{homotopy-in-c-implies-homology-in-c}
  Es sei $\SC$ eine Serre-Klasse, die (III) erfüllt, und
  $X$ einfach zusammenhängend mit $\pi_i(X, x_0) \in \SC$ für alle $i \geq 0$.
  Dann ist $X$ $\SC$-azyklisch, \dh{} es gilt $H_i(X) \in \SC$ für $i \geq 1$.
\end{lem}

\begin{proof}
  Es sei $\ldots \to X_{i+1} \to X_i \to \ldots \to X_1$ ein Postnikov-Turm von $X$, dessen Abbildungen $X_{i+1} \to X_i$ Faserungen sind.
  Wir zeigen, dass $H_i(X_k; \Z) \in \SC$ für alle $i, k > 0$.
  Die Aussage stimmt für $k = 1$, da alle Homotopiegruppen und nach dem Satz von Hurewicz auch Homologiegruppen von $X_1$ null sind.
  Gelte die Aussage nun für $k \geq 1$.
  Wir verwenden die Faserung $K(\pi_{k+1}(X), k{+}1) \to X_{k+1} \to X_k$.
  Nach Voraussetzung liegen die Homologiegruppen der Faser in~$\SC$.
  Gleiches gilt für den Basisraum $X_k$ nach Induktionsvoraussetzung,
  und somit auch für $X_{k+1}$ nach Lemma \ref{two-of-three}.

  Die Abbildung $\pi_i(X \to X_k)$ ist ein Isomorphismus für $i \leq k$ und ein Epimorphismus für $i = k+1$.
  Somit ist $H_i(X \to X_k)$ ein Isomorphismus für $i \leq k$ und ein Epimorphismus für $i = k+1$ ist.
  Dies folgt aus einer relativen Version des gewöhnlichen Hurewicz-Theorems (siehe \cite[\mbox{} Thm 4.32]{hatcher:at}) durch Betrachtung des Abbildungszylinders wie im Beweis von Korollar~\ref{homotopy-homology-equivalence-mod-c}.

  Somit gilt $H_k(X; \Z) \cong H_k(X_k; \Z) \in \SC$.
\end{proof}


\begin{proof}[Beweis von Satz \ref{hurewicz-mod-c}]
  Es sei wieder $\ldots \to X_{i+1} \to X_i \to \ldots \to X_1$ ein Postnikov-Turm von $X$, dessen Abbildungen $X_{i+1} \to X_i$ Faserungen sind.
  Aufgrund der Natürlichkeit des Hurewicz-Homomorphismus kommutiert folgendes Diagramm:
  \begin{centertikzcd}
    \pi_n(X) \arrow[r, "\cong"] \arrow[d, "h_n"] &
    \pi_n(X_n) \arrow[d, "h_n"] \\
    H_n(X) \arrow[r, "\cong"] &
    H_n(X_n)
  \end{centertikzcd}
  Es genügt daher zu zeigen, dass $h_n : \pi_n(X_n) \to H_n(X_n; \Z)$ ein Isomorphismus modulo $\SC$ ist.
  Wir verwenden die Serre-Spektralsequenz zur Faserung $F_n \to X_n \to X_{n-1}$. Es gilt
  \[
    E^2_{p,q} \cong
    H_p(X_{n-1}; \underbrace{H_q(K(\pi_n(X), n); \Z)}_{\mathclap{\substack{
      = 0 \text{ für } 0 < q < n, \\
      \text{da } \pi_q(K(\pi_n(X), n)) = 0 \text{ für } 0 < q < n
    }}}),
  \]
  es verschwinden also alle Einträge zwischen der $0$-ten und der $n$-ten Zeile.
  Wir betrachten die exakte Sequenz
  \begin{centertikzcd}[row sep=0.1cm, column sep=0.5cm]
    H_{n+1}(X_{n-1}) \arrow[d, equal] &&
    H_n(F_n) \arrow[d, equal] &
    0 \arrow[rd] &&
    0 &&&
    H_n(X_{n-1}) \arrow[d, equal] \\
    E^2_{n+1,0} \arrow[d, equal] && E^2_{0,n} \arrow[d, equal] && E^\infty_{0,n} \arrow[rrd] \arrow[ru] &&&&
    E^2_{n,0} \arrow[d, equal] \\
    %\underbrace{E^n_{n+1,0}}_{\mathrlap{\substack{= E^2_{n+1,0} \\ = H_{n+1}(X_{n-1})}}} \arrow[r, "d^n_{n+1,0}"] &
    E^n_{n+1,0} \arrow[rr, "d^n_{n+1,0}"] &&
    E^n_{0,n} \arrow[rrrr] \arrow[rru] &&&&
    H_n(X_n) \arrow[rr] &&
    E^\infty_{n,0} \arrow[r] &
    0
  \end{centertikzcd}
  welche sich aus zwei kurzen Sequenzen zusammensetzt:
  \begin{itemize}
    \item Die linke Sequenz ist exakt, da $E^\infty_{0,n} \cong E^{n+1}_{0,n} \cong \ker(d^n_{0,n}) / \im(d^n_{n+1,0}) = E^n_{0,n} / \im(d^n_{n+1,0})$.
    \item Die rechte Sequenz ist exakt, da $E^\infty_{0,n} = F^0 H_n(X_n)$ und $E^\infty_{n,0} = F^n H_n(X_n) / F^{n-1} H_n(X_n)$ für eine Filtrierung $0 = F^{-1} H_n(X_n) \subseteq F^0 H_n(X_n) \subseteq \ldots \subseteq F^n H_n(X_n) = H_n(X_n)$.
    Da $F^p H_n(X_n) / F^{p-1} H_n(X_n) = E^\infty_{p,n-p} = 0$ für $p = 1, \ldots, n-1$, gilt $F^0 H_n(X_n) = \ldots = F^{n-1} H_n(X_n)$.
  \end{itemize}
  Aus Lemma \ref{homotopy-in-c-implies-homology-in-c} folgt, dass $E^n_{n+1,0}, E^\infty_{n,0} \in \SC$.
  Somit ist der mittlere Morphismus $H_n(F_n) = E^n_{0,n} \to H_n(X_n)$ ein Isomorphismus modulo~$\SC$.
  Dieser Morphismus ist aber gerade der von der Inklusion $F_n \hookrightarrow X_n$ induzierte Morphismus in Homologie\footnote{Dies sieht man durch nähere Betrachtung des Beweises über die Existenz der Serre-Spektralsequenz zu einer Faserung $F \to X \to B$ in \cite{hatcher:ss}: Durch CW-Approximation kann man davon ausgehen, dass $B$ ein CW-Komplex ist. Die Filtrierung von $H_n(X)$ wird dort definiert durch $F^p H_n(X) \cong \im(H_n(p^{-1}(B^p) \hookrightarrow X)) \subseteq H_n(X)$, wobei~$B^p$ das $p$-Skelett von $B$ bezeichnet. Die Abbildung $H_n(F) \cong H_0(B; H_n(F)) \cong E^2_{0,n} \to E^\infty_{0,n} \cong F^0 H_n(X)$ wird durch die Inklusion von $F \approx p^{-1}(B^0)$ in $X$ induziert.}.

  Betrachte nun das kommutative Diagramm
  \begin{centertikzcd}
    \pi_n(F_n) \arrow{d}{h_n}[swap]{\cong} \arrow[r, "\cong"] &
    \pi_n(X_n) \arrow[d, "h_n"] \\
    H_n(F_n) \arrow[r] &
    H_n(X_n)
  \end{centertikzcd}
  Dabei sind die vertikalen Morphismen die Hurewicz-Homomorphismen und die horizontalen Morphismen werden durch die Inklusion induziert.
  Der linke Hurewicz-Homomorphismus ist nach dem Hurewicz-Theorem ein Isomorphismus, da $F_n$ $(n{-}1)$-zusammenhängend ist.
  Der obere Morphismus ist ein Isomorphismus, wie man anhand der langen exakten Sequenz der Faserung $F_n \to X_n \to X_{n-1}$ sehen kann.
  Der untere Morphismus ist ein Isomorphismus modulo $\SC$, wie wir gerade eben gesehen haben.
  Somit ist auch der rechte Morphismus im Diagramm ein Isomorphismus modulo $\SC$.
\end{proof}

\begin{satz}[Relatives Mod-$\SC$-Hurewicz-Theorem]\label{relative-hurewicz-mod-c}
  Es sei $\SC$ eine Serre-Klasse, die (III) erfüllt, $(X, A)$ ein $1$-zusammenhängendes Raumpaar, wobei $A$ nichtleer und einfach zusammenhängend ist.
  Angenommen, $\pi_i(X, A) \in \SC$ für $0 < i < n$.
  Dann ist $h_i : \pi_i(X, A) \to H_i(X, A; \Z)$ ein Isomorphismus modulo $\SC$ für $i \leq n$.
  Insbesondere gilt $H_i(X, A; \Z) \in \SC$ für $0 < i < n$.
\end{satz}

% nach Vorlage des Beweises des relativen gewöhnlichen Hurewicz-Theorems aus dem absoluten gewöhnlichen Hurewicz-Theorem, siehe hatcher:at, Seite 367
\begin{proof}
  Durch CW-Approximation erhält man ein CW-Paar $(X', A')$ und eine Abbildung $f : (X', A') \to (X, A)$, welche Isomorphismen in relativen und absoluten Homotopiegruppen induziert (vgl. \cite[\mbox{} Ex 4.14]{hatcher:at}).
  Insbesondere ist auch $A'$ einfach zusammenhängend.
  Da~$(X', A')$ ein gutes Raumpaar ist, induziert $p : (X', A') \to (X'/A', *)$ Isomorphismen in allen positiven Homologiegruppen.
  In Homotopie ist $p_* : \pi_i(X', A') \to \pi_i(X'/A')$ für $i \leq n$ ein Isomorphismus (siehe \cite[\mbox{} Prop 4.28]{hatcher:at}).
  Somit ist $X'/A'$ $(n {-} 1)$-zusammenhängend.
  Betrachte nun
  \begin{center}
    \begin{tikzcd}[row sep=1.2cm, column sep=1.6cm]
      \pi_i(X'/A') \arrow{d}{}{h_i} &
      \pi_i(X', A') \arrow{d}{}{h_i} \arrow{l}{\cong}[swap]{p_*} \arrow{r}{f_*}[swap]{\cong} &
      \pi_i(X, A) \arrow{d}{h_i} \\
      H_i(X'/A') &
      H_i(X', A') \arrow{l}{\cong}[swap]{p_*} \arrow{r}{f_*}[swap]{\cong} &
      H_i(X, A)
    \end{tikzcd}
  \end{center}
  Aufgrund der absoluten Version des Theorems (Satz~\ref{hurewicz-mod-c}) ist also der linke Pfeil im folgenden Diagramm ein Isomorphismus modulo~$\SC$ für $i \leq n$.
  Gleiches gilt somit für den rechten Pfeil.
\end{proof}

Wir erhalten eine relative Version von Korollar~\ref{homotopy-in-c-iff-homology-in-c}:

\begin{kor}\label{homotopy-in-c-iff-homology-in-c-relative}
  Es sei $\SC$ eine Serre-Klasse, die (III) erfüllt, $(X, A)$ ein $1$-zusammenhängendes Raumpaar, wobei $A$ nichtleer und einfach zusammenhängend ist.
  Dann gilt für alle $N \in \N \cup \{ \infty \}$:
  \[
    \pushQED{\qed} 
    \fa{0 \leq n < N} \pi_n(X, A, x_0) \in \SC
    \quad \iff \quad
    \fa{1 \leq n < N} H_n(X, A; \Z) \in \SC.
    \qedhere
    \popQED
  \]
\end{kor}

\begin{kor}\label{homotopy-homology-equivalence-mod-c}
  Es sei $\SC$ eine Serre-Klasse, die (III) erfüllt, $f : A \to B$ stetig, $f_* : \pi_1(A) \to \pi_1(B)$ surjektiv und $A$ einfach zusammenhängend.
  Dann sind äquivalent:
  \begin{enumerate}[label=(\alph*)]
    \item $f_* : \pi_i(A) \to \pi_i(B)$ ist ein Isomorphismus modulo~$\SC$ für $1 \leq i < n$ und ein Epimorphismus modulo~$\SC$ für $i = n$.
    \item $f_* : H_i(A) \to H_i(B)$ ein Isomorphismus modulo~$\SC$ für $1 \leq i < n$ und ein Epimorphismus modulo~$\SC$ für $i = n$.
  \end{enumerate}
\end{kor}

\begin{proof}
  Betrachte den Abbildungszylinder~$M_f$ von~$f$.
  Die Abbildung~$f$ entspricht der Inklusion~$A \hookrightarrow M_f$ von $A$ als Deckel des Abbildungzylinder, wenn man $X$ als Deformationsretrakt von $M_f$ auffasst.
  Es gilt nun
  \[
    \text{(a)}
    \stackrel{\text{l.\,e.\,S.}}{\iff}
    \pi_i(M_f, A) \in \SC \text{ für } 0 \leq i \leq n
    \stackrel{\text{Kor \ref{homotopy-in-c-iff-homology-in-c-relative}}}{\iff}
    H_i(M_f, A) \in \SC \text{ für } 1 \leq i \leq n
    \stackrel{\text{l.\,e.\,S.}}{\iff}
    \text{(b)}.
    \qedhere
  \]
\end{proof}

\begin{bem}
  Man kann in diesem Abschnitt die Voraussetzung, dass $X$ einfach zusammenhängend ist, ersetzen durch die Forderung, dass $X$ wegzusammenhängend und abelsch ist, \dh{} die Wirkung der Fundamentalgruppe $\pi_1(X)$ auf den höheren Homotopiegruppen $\pi_n(X)$ trivial ist.
\end{bem}

\section{Rationale Kohomologie von Räumen vom Typ $K(\Z, n)$}

Wir wollen zeigen, dass die Homotopiegruppen von gewissen Räumen $X$ alle endlich sind.
Nach Korollar \ref{homotopy-in-c-iff-homology-in-c} können wir dazu zeigen, dass die Homologiegruppen in positiven Graden endlich sind.
Angenommen, wir wissen bereits, dass die Homologiegruppen von $X$ endlich erzeugt sind.
Dann ist $H_i(X; \Z)$ genau dann endlich, wenn $H_i(X; \Z) \otimes \Q = 0$.
Allgemeiner ist die $i$-te Bettizahl $b_i \coloneqq \dim_\Q(H_i(X; \Z) \otimes \Q)$ die Dimension des freien Anteils von $H_i(X; \Z)$.
Da $\Q$ flach ist, gilt $H_i(X; \Z) \otimes \Q \cong H_i(X; \Q)$.
Außerdem gilt $H_i(X; \Q) \cong \Hom(H_i(X; \Q), \Q) \cong H^i(X; \Q)$, wenn die Homologiegruppen von $X$ endlich erzeugt sind.
Das ist der Grund, warum wir uns für Homologie- und Kohomologiegruppen mit Koeffizienten in $\Q$ interessieren.
Der Vorteil gegenüber Homologie- und Kohomologiegruppen mit Koeffizienten in $\Z$ besteht darin, dass diese Gruppen leichter zu bestimmen sind.
Konkret haben wir für $X = K(\Z, n)$:

% Hatcher, 1.20
\begin{satz}\label{rational-homology-kzn}
  Für $n \geq 1$ gilt
  \vspace{-1em}
  \[
    H^*(K(\Z, n); \Q) \cong \begin{cases}
      \Q[x] & \text{falls $n$ gerade}, \\
      \Lambda_\Q[x] & \text{falls $n$ ungerade}
    \end{cases}
  \]
  als graduierte Ringe mit Erzeuger $x \in H^n(K(\Z, n); \Q)$.
  Dabei bezeichnet $\Lambda_\Q[x]$ die äußere Algebra mit Erzeuger $x$.
\end{satz}

\begin{proof}
  Durch Induktion über $n$.
  Der Satz gilt für $n = 1$, denn der Kreis $S^1$ ist ein $K(\Z, 1)$ und es gilt bekanntermaßen $H^*(S^1; R) \cong \Lambda_R[x]$ für $R = \Z$ und somit auch für $R = \Q$. \\
  Im Induktionsschritt nutzen wir die Pfadfaserung $F \coloneqq K(\Z, n{-}1) \to P \to B \coloneqq K(\Z, n)$. Da $K(\Z, n)$ für $n \geq 2$ einfach zusammenhängend ist, gilt für deren zugehörige Serre-Spektralsequenz
  $E_2^{p,q} \cong H^p(B; H^q(F))$.

  \vspace{0.5cm}

  \renewcommand\windowpagestuff{
    \begin{center}\begin{tikzpicture}[x=16pt,y=16pt]\begin{scope}[shift={(0,0)}]
      \foreach \x in {1,2,3,5,6,7}{
        \foreach \y in {0,1,...,3}{
          \zeroDot{\x}{\y}
        }
      }
      \foreach \x in {0,...,8}{
        \zeroDot{\x}{4}
      }
      \foreach \y in {0,...,4}{
        \node at (9.5,\y) {$\cdots$};
      }
      \node at (0,0) {$\Q$\small $1$};
      \zeroDot{0}{1}
      \zeroDot{0}{2}
      \node at (0,3) {$\Q$\small $a$};
      \node at (4,0) {$\Q$\small $x$};
      \zeroDot{4}{1}
      \zeroDot{4}{2}
      \node at (4,3) {$\Q$\small $ax$};
      \draw[->,gray] (0.6,2.6) -- (3.4,0.4);
      \node at (8,0) {$\Q$\small $x^2$};
      \zeroDot{8}{1}
      \zeroDot{8}{2}
      \node at (8,3) {$\Q$\small $ax^2$};
      \draw[->,gray] (4.6,2.6) -- (7.4,0.4);
      \draw[->] (-0.7,-0.6) -- (10.3,-0.6); %node[below] {p};
      \draw[->] (-0.7,-0.6) -- (-0.7,4.5); %node[left] {q};
      \node[left] at (-0.7,0) {$0$};
      \node[left] at (-0.9,1.7) {\vdots};
      \node[left] at (-0.7,3) {$n{-}1$};
      \node[below] at (0,-0.6) {$0$};
      \node[below] at (2,-0.7) {$\cdots$};
      \node[below] at (4,-0.7) {$n$};
      \node[below] at (6,-0.7) {$\cdots$};
      \node[below] at (8,-0.6) {$2n$};
      \node[below] at (9.5,-0.7) {$\cdots$};
    \end{scope}\end{tikzpicture}\end{center}
  }
  \opencutright
  \begin{cutout}{0}{\dimexpr\linewidth-8cm\relax}{0pt}{8}
    \emph{Falls $n$ gerade:} \enspace
    Dann sieht die Spektralsequenz auf der Seite $E_r$, $r \leq n$ aus wie rechts skizziert (dabei stehen die Gitterpunkte für die Nullgruppe).
    Das sieht man folgendermaßen: Zunächst ist $E_2^{p,q} = 0$ und somit $E_r^{p,q} = 0$ außer für $q \in \{ 0, n{-}1 \}$, denn nach Induktionsvoraussetzung gilt $H^*(F; \Q) \cong \Lambda_\Q[x]$.
    Es folgt, dass nur auf der $n$-ten Seite $E_n$ nicht verschwindende Differentiale existieren können und $E_2 \cong E_n$ und $E_{n+1} \cong E_\infty$ gilt.
    Außerdem ist $E_n^{0,0} \cong E_2^{0,0} \cong \Q$ und $E_n^{0,n-1} \cong E_2^{0,n-1} \cong \Q$, da $B$ zusammenhängend ist.
    Die Spektralsequenz konvergiert gegen $H^*(P; \Q)$.
    Da $P$ zusammenziehbar ist, gilt $H^0(P; \Q) = \Q$ und $H^n(P; \Q) = 0$ für $n > 0$.
    Folglich ist $E_{n+1}^{p,q} \cong E_\infty^{p,q} = 0$ außer für $p = q = 0$.
    Insbesondere gilt $E_{n+1}^{0,n-1} \cong E_\infty^{0,n-1} = 0$.
    %Es gilt $E_n^{0,n-1} \cong E_2^{0,n-1} \cong \Q$, da die von $E_r^{0,n-1}$ ausgehenden Differentiale $d_r^{0,n-1}$ für $r < n$ Null und die dort eintreffenden Differentiale für alle $r$ Null sind.
    Das eingezeichnete Differential $d_n^{0,n-1} : E_n^{0,n-1} \to E_n^{n,0}$ ist nun injektiv, denn $\ker(d_n^{0,n-1}) \cong E_{n+1}^{0,n-1} = 0$.
    Dieses Differential ist auch surjektiv, denn $\coker(d_n^{0,n-1}) \cong E_{n+1}^{n,0} \cong E_\infty^{n,0} = 0$, also ein Isomorphismus.
    Somit $H^n(B; \Q) \cong E_2^{n,0} \cong E_n^{n,0} \cong E_n^{0,n-1} \cong \Q$.
    Der zweite Isomorphismus kommt daher, da für $r \leq n-1$ alle Differentiale von und nach $E_r^{n,0}$ Null sind. % XXX: besser formulieren?
    Damit ist $E_2^{n,n-1} \cong H^{n}(B; H^{n-1}(F; \Q)) \cong H^{n}(B; \Q) \cong \Q$.
    Induktiv sieht man nun, dass die Abbildungen $d_n^{kn,n-1}$ Isomorphismen sind und dass $H^{kn}(B; \Q) \cong E_2^{kn,0} \cong  E_2^{kn,n-1} \cong \Q$ für alle $k \geq 0$.
    Damit haben wir gezeigt, dass die graduierte, additive Struktur von $H^*(B; \Q)$ wie behauptet ist.
  \end{cutout}
  Es sei nun $a \in E_n^{0,n-1} \cong H^0(B; H^{n-1}(F; \Q))$ ungleich Null und $x \coloneqq d_n^{0,n-1}(a) \in E_n^{n,0} \cong H^n(B; H^0(F; \Q)) \cong H^n(B; \Q)$.
  Dann gilt auch $x \neq 0$ und $ax \coloneqq m_r(a, x) \neq 0$, da wegen (ii) und (iii) das Produkt $m_r$ gerade dem kanonischen Produkt $H^n(B; H^0(F; \Q)) \times H^0(B; H^{n-1}(F; \Q)) \to H^n(B; H^{n-1}(F; \Q))$ entspricht. % XXX: kommutatives Diagramm dazu?
  Es gilt
  $0 \neq d_n^{n,n-1}(ax) = d_n^{0,n-1}(a)x - a d_n^{n,0}(x) = xx$.
  Da das Produkt $xx \in E_n^{2n,0}$ gerade dem Cup-Produkt $x \cup x \in H^{2n}(B; \Q)$ entspricht, ist $x \cup x \neq 0$, also ein Erzeuger von $H^{2n}(B; \Q)$.
  Induktiv ist nun $0 \neq d_n^{kn,n-1}(a x^k) = x^{k+1} \in E_n^{kn,0}$ da ja $0 \neq a x^k$.
  Somit ist für alle $k$ das $k$-fache Cup-Produkt $x^k \in H^{kn}(B; \Q)$ ein Erzeuger.

  \vspace{1.5cm}

  \renewcommand\windowpagestuff{
    \begin{center}\begin{tikzpicture}[x=16pt,y=16pt]\begin{scope}[shift={(0,0)}]
      \foreach \x in {1,2,3}{
        \foreach \y in {0,1,...,6}{
          \zeroDot{\x}{\y}
        }
      }
      \foreach \y in {0,...,6}{
        \zeroDot{5}{\y}
      }
      \foreach \x in {0,...,5}{
        \node at (\x,7.3) {\vdots};
      }
      \node at (0,0) {$\Q$\small $1$};
      \zeroDot{0}{1}
      \zeroDot{0}{2}
      \node at (0,3) {$\Q$\small $a$};
      \zeroDot{0}{4}
      \zeroDot{0}{5}
      \node at (0,6) {$\Q$\small $a^2$};
      \node at (4,0) {$\Q$\small $x$};
      \zeroDot{4}{1}
      \zeroDot{4}{2}
      \node at (4,3) {$\Q$\small $ax$};
      \zeroDot{4}{4}
      \zeroDot{4}{5}
      \node at (4,6) {$\Q$\small $a^2 x$};
      \draw[->,gray] (0.6,2.6) -- (3.4,0.4);
      \draw[->,gray] (0.6,5.6) -- (3.4,3.4);
      \draw[->] (-0.7,-0.6) -- (5.6,-0.6); %node[below] {p};
      \draw[->] (-0.7,-0.6) -- (-0.7,7.8); %node[left] {q};
      \node[left] at (-0.7,0) {$0$};
      \node[left] at (-0.9,1.7) {\vdots};
      \node[left] at (-0.7,3) {$n{-}1$};
      \node[left] at (-0.9,4.7) {\vdots};
      \node[left] at (-0.7,6) {$2n{-}2$};
      \node[below] at (0,-0.6) {$0$};
      \node[below] at (2,-0.7) {$\cdots$};
      \node[below] at (4,-0.7) {$n$};
    \end{scope}\end{tikzpicture}\end{center}
  }
  \opencutleft
  \begin{cutout}{0}{0pt}{\dimexpr\linewidth-5.5cm\relax}{11}
    \emph{Falls $n$ ungerade:} \enspace
    Dann ist $E_r^{p,q} = 0$ für alle $q$, die kein Vielfaches von $n-1$ sind.
    Somit verschwinden alle Differentiale auf $E_r$ für $r < n$ und $E_2 \cong E_n$.
    Für $0 < m < n$ verschwinden alle Differentiale von und nach $E_r^{m,0}$ und daher ist $H^m(B; \Q) \cong E_2^{m,0} \cong E_\infty^{m,0} = 0$ und  folglich $E_2^{m,k} = 0$ für alle $k \geq 0$.
    Selbiges gilt folglich auch für $n < m < 2n$ und allgemeiner für solche $m$, die kein Vielfaches von $n$ sind.
    Analog wie im vorherigen Fall sieht man, dass das eingezeichnete Differential $d_n^{0,n-1}$ ein Isomorphismus ist.
    Somit $H^n(B; \Q) \cong H^n(B; H^0(F; \Q)) \cong \Q$ und $E_2^{n,k(n-1)} \cong \Q$ für alle $k \geq 0$.
    Sei $a \in H^0(B; H^{n-1}(F; \Q))$ ungleich Null und $x \coloneqq d_n^{0,n-1}$.
    Dann ist auch $a^2 \neq 0 \in E_n^{0,2n-2}$ und $d_n^{0,2n-2}(a^2) = d_n^{0,n-1}(a)a + d_n^{0,n-1}(a)a = xa + ax = (-1)^{0 \cdot n + (n-1) \cdot 0} ax + ax = 2ax \neq 0$.
    Also ist $d_n^{0,2n-2}$ ein Isomorphismus.
    Analog sieht man, dass $d_n^{0,k(n-1)}$ für alle $k \geq 1$ ein Isomorphismus ist.
    Es bleibt zu zeigen, dass $H^{kn}(B; \Q) = 0$ für $k > 1$.
    Das einzige potentiell nichttriviale Differential, das bei $E_r^{2n,0}$ ankommt, ist $d_n^{n,n-1}$.
    Dieses ist aber Null, da $\ker(d_n^{n,n-1}) = \im(d_n^{0,2n-2}) = E_n^{n,n-1}$.
    Also $H^{2n}(B; \Q) \cong E_2^{2n,0} \cong E_\infty^{2n,0} = 0$ und $E_2^{2n,k} = 0$ für alle $k \geq 0$.
    Für $k > 2$ sieht man durch Induktion, dass alle Differentiale von und nach $E_r^{2n,0}$ verschwinden und daher $H^{kn}(B; \Q) = 0$. \qedhere
  \end{cutout}
\end{proof}

\section{Homologie der Stiefel-Mannigfaltigkeit $\V{2}{m}$}

% Die Homologie dieser Stiefel-Mannigfaltigkeit wird auch in Hatcher, §3.D berechnet
Die Stiefel-Mannigfaltigkeit $\V{2}{m}$ ist die aller orthonormalen $2$-Tupel $(u, v)$ im $\R^m$ mit dem euklidischen Skalarprodukt.
Die Dimension dieser Mannigfaltigkeit ist $(m-1) + (m-2) = 2m - 3$.
Man kann einen Punkt $(u, v)$ auf dieser Mannigfaltigkeit auch als Einheitstangentialvektor an die Sphäre~$S^{m-1}$ (mit der kanonischen Metrik) am Punkt~$u \in S^{m-1}$ ansehen, wenn man sich den Startpunkt von~$v$ an den Punkt~$u$ verschoben vorstellt.
Die Abbildung $p : \V{2}{m} \to S^{m-1}, \enspace (u, v) \mapsto u$ ist die Projektionsabbildung eines \emph{Faser-Bündels}, d.\,h. sie erfüllt eine lokale Trivialitätsbedingung.
Die Menge aller Einheitstangentialvektoren in einem Punkt~$u$ ist homöomorph zur $S^{m-2}$.
Die Faser dieses Bündels ist daher eine $S^{m-2}$.
Da jedes Faser-Bündel auch eine Serre-Faserung ist (vgl. \cite[\mbox{} Prop 4.48]{hatcher:at}), gibt es eine Faserung $S^{m-2} \to \V{2}{m} \xrightarrow{p} S^{m-1}$.

Wir wollen nun die Homologie von $\V{2}{m+1}$ berechnen.
Dazu konstruieren wir eine CW-Struktur auf $\V{2}{m+1}$.
Wir fassen nun $\V{2}{m+1}$ als Mannigfaltigkeit $UT S^m$ der Einheitstangentialvektoren an $S^m = \Set{v \in R^{m+1}}{\norm{v} = 1}$ auf.
Wir notieren $UT_p S^m \coloneqq \Set{v \in T_p S^m}{\norm{v} = 1}$ für $p \in S^m$.
Es seien $p_N, p_S \in S^m$ der Nord- bzw. Südpol von $S^m$, $u_S \in UT_{p_S}$ und $u_N \in UT_{p_N} S^m$.

Wir verwenden vier Zellen:

\begin{itemize}
  \item einer $0$-Zelle $e^0$ für $u_S$,
  \item einer $(m-1)$-Zelle $e^{m-1}$ für $UT_{p_S}$, die Einheitstangentialvektoren am Punkt $p_S$,
  \item einer $m$-Zelle $e^m$ für alle Einheitstangentialvektoren auf $S^m \setminus \{ p_S \}$, die man durch Paralleltransport von $u_N$ entlang von Meridianen erhält und
  \item einer $(2m - 1)$-Zelle $e^{2m-1}$ für alle restlichen Einheitstangentialvektoren.
\end{itemize}

Wir fassen das $(m-1)$-Skelett $(\V{2}{m+1})^{m-1} \approx S^{m-1}$ als Raum der Einheitstangentialvektoren an~$p_S$ auf.
Den Rand $\partial e^m$ können wir mit dem Raum der Einheitstangentialvektoren an~$p_N$ identifizieren.

Die Anklebeabbildung $\phi : \partial e^m \to (\V{2}{m+1})^{m-1}$ schickt einen Vektor $v \in T_{p_N} S^m$ auf $P_{\gamma_v(0),\gamma_v(\pi)} u_N$, dem entlang dem Meridian $\gamma : [0, \pi] \to S^m$ mit $\dot{\gamma}(0) = v$ parallel verschobenen Vektor $u_N$.

\TODO{letzte Anklebeabbildung}

Um die Homologie von $\V{2}{m+1}$ zu bestimmen, müssen wir den Abbildungsgrad von $\phi : \partial e^m \to (\V{2}{m+1})^{m-1}$ ausrechnen.
Dieser Abbildungsgrad ist (bis auf evtl. ein Vorzeichen) gleich dem Abbildungsgrad der Abbildung
\[
  g : UT_{p_N} \to UT_{p_N}, \quad
  v \mapsto P_{\gamma_{u_N}(0),\gamma_{u_N}(\pi)}^{-1} \circ P_{\gamma_v(0),\gamma_v(\pi)} u_N,
\]
welche $v$ abbildet auf den Vektor, den man erhält, wenn man $u_N$ entlang des $v$-Meridians an den Südpol verschiebt und dann entlang des $u_N$-Meridians wieder zurück an den Nordpol verschiebt.
Einfache geometrische Überlegungen zeigen, dass diese Abbildung gegeben ist durch $g(v) = r_{u_N} \circ r_{v} u_N$, wobei $r_w : UT_{p_N} S^m \to UT_{p_N} S^m$ die Spiegelung an der zu $w \in UT_{p_N} S^m$ senkrechten Hyperebene ist.
Sei $W \subset UT_{p_N} S^m$ die Menge aller auf $u_N$ senkrechten Einheitstangentialvektoren.
Die Abbildung $g$ ist auf den beiden Komponente von $UT_{p_N} S^m \setminus W$ ein Homöomorphismus auf $UT_{p_N} S^m \setminus \{ -u_N \}$.
Diese beiden Homöomorphismen gehen durch Komposition mit der Antipodenabbildung $w \mapsto -w$ auf $UT_{p_N} S^m \approx S^{m-1}$ auseinander hervor.
Die Antipodenabbildung besitzt den Abbildungsgrad $(-1)^m$.

\TODO{weiter}


\TODO{Zeigen oder zitieren: Es gilt $H_i(\V{2}{2k+1}) \cong \begin{cases}
  \Z & \text{für } i = 0, 2k-1, \\
  \Z_2 & \text{für } i = k-1, \\
  0 & \text{sonst.}
\end{cases}$.
Es gibt eine stetige Abbildung $q : \V{2}{m} \to X^{2m-3} / X^{2m-4} \approx S^{2m-3}$, die in $H_{2m-3}$ einen Isomorphismus induziert.}

\section{Beweis des Satzes von Serre}

\begin{proof}[Beweis des Satzes von Serre]
  Ist $n$ ungerade, so ist die Abbildung $i : S^n \to \tau_{\leq n} S^n = K(\Z, n)$ nach Satz~\ref{rational-homology-kzn} eine Homologieäquivalenz modulo~$\F$.
  Aus Korollar~\ref{homotopy-homology-equivalence-mod-c} folgt, dass~$i$ auch auf allen Homotopiegruppen Isomorphismen modulo~$\F$ induziert.
  Somit sind alle Homotopiegruppen von~$S^n$ im Grad $> n$ endlich.

  Es sei nun $n$ gerade.
  Die oben beschriebene Abbildung~$q : \V{2}{n+1} \to S^{2n-1}$ induziert Isomorphismen in Homologie modulo~$\F$.
  Somit induziert $q$ auch Isomorphismen auf den Homotopiegruppen.
  Aus der langen exakten Homotopiesequenz
  \[
    \ldots
    \enspace\to\enspace
    \pi_{i+1}(S^n)
    \enspace\to\enspace
    \underbrace{\pi_i(S^{n-1})}_{\substack{\text{endlich für} \\ i \neq n-1}}
    \enspace\to\enspace \underbrace{\pi_i(\V{2}{n+1})}_{\mathclap{\substack{\cong \pi_i(S^{2n-1}) \bmod{\F}, \\ \text{endlich für } i \neq 2n-1}}}
    \enspace\to\enspace
    \pi_i(S^n)
    \enspace\to\enspace
    \pi_{i-1}(S^{n-1})
    \enspace\to\enspace
    \ldots
  \]
  zur Faserung $S^{n-1} \to \V{2}{n+1} \xrightarrow{p} S^n$ folgt nun, dass $\pi_i(S^n)$ endlich ist für $i \not\in \{ n, 2n-1 \}$ und dass $\pi_{2n-1}(S^n) \cong \pi_{2n-1}(\V{2}{n+1}) \cong \Z$ modulo~$\F$.
\end{proof}

\section{Die erste Homotopiegruppe von $S^n$ mit $p$-Torsion}

% TODO: kurze Einleitung

Es sei $G$ eine abelsche Gruppe, $p \in \N$ eine Primzahl.
Die \emph{$p$-primäre Torsionskomponente} von $G$ ist die Menge aller $g \in G$, deren Ordnung eine $p$-Potenz ist.
Sei $\Z_{(p)}$ die Lokalisierung von $\Z$ nach dem Primideal $(p)$ oder äquivalent der Unterring von $\Q$ bestehend aus den Brüchen, deren Nenner nicht durch $p$ teilbar ist.
Wir können die $p$-Torsionsgruppe einer endlich erzeugten abelschen Gruppe $G$ an $G \otimes \Z_{(p)}$ ablesen, denn es gilt:
\begin{itemize}
  \item $\Z \otimes \Z_{(p)} \cong \Z_{(p)}$,
  \item $\Z_{p^n} \otimes \Z_{(p)} \cong \Z_{p^n}$ und
  \item $\Z_m \otimes \Z_{(p)} = 0$, falls $m$ nicht durch $p$ teilbar ist.
\end{itemize}

Es sei $X$ nun ein topologischer Raum.
Da $\Z_{(p)}$ torsionsfrei ist, gilt
$H_i(X; \Z) \otimes \Z_{(p)} \cong H_i(X; \Z_{(p)})$ nach dem universellen Koeffiziententheorem.
Falls die Kohomologiegruppen von $X$ endlich erzeugt sind, so gilt
$H^i(X; \Z) \otimes \Z_{(p)} \cong H^i(X; \Z_{(p)})$
nach \cite[\mbox{} Thm 5.5.10]{spanier:at}.
Wir können somit die $p$-Torsion in (ganzzahliger) Homologie und Kohomologie bestimmen, indem wir die Homologie- und Kohomologiegruppen mit Koeffizienten mit $\Z_{(p)}$ berechnen.

% 1.29 in Hatcher SS1
\begin{lem}\label{mod-p-cohomology-kzn}
  Es sei $n \geq 3$ und $p$ prim.
  Dann ist die p-Torsionsuntergruppe von $H^i(K(\Z, n); \Z)$ null für $i < n + 2p - 1$ und isomorph zu $\Z_p$ für $i = n + 2p - 1$.
\end{lem}

\begin{proof}
  Durch Induktion über $n$.
  Wir verwenden wieder die Pfadfaserung $K(\Z, n-1) \to P \to K(\Z, n)$ und die dazugehörige Serre-Spektralsequenz, diesmal mit Koeffizienten in $\Z_{(p)}$.
  
  \emph{Falls $n = 3$:} \enspace
  Wir können die Faser, einen $K(\Z, 2)$-Raum, als unendlich-dimensionalen komplexen projektiven Raum~$\C P^\infty$ realisieren.\footnote{Sei $S^\infty \subset \C^\infty$ die unendlich-dimensionale Sphäre mit der Äquivalenzrelation $a \sim b :\!\iff \exists \, \lambda \in S^1 \subset \C \,:\, a = \lambda \cdot b$. Setze $\C P^\infty \coloneqq S^\infty / {\sim}$. Anhand der langen exakten Sequenz zur Faserung $S^1 \to S^\infty \to \C P^\infty$ erkennt man, dass $\C P^\infty$ ein $K(\Z, 2)$ ist.}
  Dieser besitzt eine CW-Struktur bestehend aus je einer Zelle in geraden Dimensionen und keiner Zelle in ungeraden Dimensionen (siehe \cite[\mbox{} Bsp 0.6]{hatcher:at}).
  Daher ist $H^{2i}(\C P^\infty) \cong \Z$ und $H^{2i+1}(\C P^\infty) = 0$.
  Genauer gilt $H^*(\C P^\infty; \Z) \cong \Z[a]$ mit Erzeuger $a \in H^2(\C P^\infty; \Z)$ \cite[\mbox{} Thm 3.12]{hatcher:at}.

  \begin{center}
    \begin{tikzpicture}[x=32pt,y=16pt]
      \foreach \x in {1,2,4,5,7,8}{
        \foreach \y in {0,1,...,8}{
          \zeroDot{\x}{\y}
        }
      }
      \foreach \y in {0,...,8}{
        %\node at (6,\y) {$\cdots$};
        \zeroDot{6}{\y}
      }
      \node at (0,0) {\small $\Z_{(p)} 1$};
      \zeroDot{0}{1}
      \node at (0,2) {\small $\Z_{(p)} a$};
      \zeroDot{0}{3}
      \node at (0,4.15) {\vdots};
      \zeroDot{0}{5}
      \node at (0,6) {\small $\Z_{(p)} a^{p-1}$};
      \zeroDot{0}{7}
      \node at (0,8) {\small $\Z_{(p)} a^p$};
      \node at (3,0) {\small $\Z_{(p)} x$};
      \zeroDot{3}{1}
      \node at (3,2) {\small $\Z_{(p)} ax$};
      \zeroDot{3}{3}
      \node at (3,4.15) {\vdots};
      \zeroDot{3}{5}
      \node at (3,6) {\small $\Z_{(p)} a^{p-1} x$};
      \zeroDot{3}{7}
      \node at (3,8) {*};
      \node at (9,0) {\small $\Z_{(p)} y$};
      \zeroDot{9}{1}
      \node at (9,2) {*};
      \zeroDot{9}{3}
      \node at (9,4.15) {\vdots};
      \zeroDot{9}{5}
      \node at (9,6) {*};
      \zeroDot{9}{7}
      \node at (9,8) {*};
      \draw[->,gray] (0.6,1.6) -- (2.4,0.4);
      \draw[->,gray] (0.6,3.6) -- (2.4,2.4);
      \draw[->,gray] (0.6,5.6) -- (2.4,4.4);
      \draw[->,gray] (0.6,7.6) -- (2.3,6.4);
      %\draw[->,gray] (0.6,5.6) -- (3.4,3.4);
      \draw[->,gray] (3.6,5.6) -- (8.4,0.4);
      \draw[->] (-0.7,-0.6) -- (9.5,-0.6); %node[below] {p};
      \draw[->] (-0.7,-0.6) -- (-0.7,8.5); %node[left] {q};
      \node[left] at (-0.7,0) {$0$};
      \node[left] at (-0.7,1) {$1$};
      \node[left] at (-0.7,2) {$2$};
      \node[left] at (-0.9,4.15) {\vdots};
      \node[left] at (-0.7,6) {$2p{-}2$};
      \node[left] at (-0.7,7) {$2p{-}1$};
      \node[left] at (-0.7,8) {$2p$};
      \node[below] at (0,-0.6) {$0$};
      \node[below] at (1,-0.6) {$1$};
      \node[below] at (2,-0.6) {$2$};
      \node[below] at (3,-0.6) {$3$};
      \node[below] at (6,-0.7) {$\cdots$};
      \node[below] at (9,-0.6) {$2p+2$};
    \end{tikzpicture}
  \end{center}

  Die $0$-Spalte der $E_2$-Seite ist daher wie abgebildet.
  Die nächsten beiden Spalten verschwinden, da $H^1(K(\Z, 3)) = H^2(K(\Z, 3)) = 0$. Für die $3$-Spalte gilt $E_2^{3,2i} \cong \Z_{(p)}$ erzeugt von~$a^i x$, wobei~$x$ ein Erzeuger von $E_2^{3,0} \cong H^3(K(\Z, 3); \Z_{(p)}) \cong \Z_{(p)}$ ist.
  Der Einträge an Position~$(0, 2)$ und~$(3, 0)$ überleben bis auf die $E_3$-Seite, müssen aber auf der $E_4$-Seite verschwinden.
  Daher ist $d_3^{0,2} : E_3^{0,2} \to E_3^{3,0}$ ein Isomorphismus.
  Wir können ohne Einschränkung davon ausgehen, dass $d_3(a) = x$.
  Aus der Cup-Produkt-Struktur der Spektralsequenz folgt nun $d_3^{0,2k}(a^k) = k a^{k-1} x$.
  Somit ist~$d_3^{0,2k}$ ein Isomorphismus für $k < p$.
  Für~$k = p$ ist~$d_3^{0,2p}$ immer noch injektiv, besitzt aber einen nichttrivialen Kokern $\Z_{(p)} / p \Z_{(p)} \cong \Z_p$.
  Die Spalten mit Dimension $m = 4, \ldots, 2p+1$ verschwinden, denn alle Differentiale mit Ziel $E_r^{m,0}$ sind null.
  Damit muss bereits $E_2^{m,0}$ und somit die ganze $m$-te Spalte verschwinden.
  Folglich ist $E_{2p-1}^{0,2p} \cong E_4^{0,2p} \cong \Z_p$ und $E_{2p-1}^{2p+2,0} \cong E_2^{2p+2,0}$.
  Das Differential $d_{2p-1}^{0,2p}$ ist ein Isomorphismus zwischen diesen beiden Einträgen, da sonst mindestens ein Eintrag die die Spektralsequenz überleben würde.
  Somit ist
  \begin{align*}
    H^{2p+2}(K(\Z, 3)) \otimes \Z_{(p)} & \cong H^{2p+2}(K(\Z, 3); \Z_{(p)}) \cong H^{2p+2}(K(\Z, 3); H^0(K(\Z, 2); \Z_{(p)})) \\
    & \cong E_2^{2p+2,0} \cong E_{2p-1}^{2p+2,0} \cong \Z_p.
  \end{align*}
  
  \emph{Falls $n > 3$, $n$ gerade:} \enspace
  
  \begin{center}
    \begin{tikzpicture}[x=28pt,y=16pt]
      \foreach \x in {1,2,4,7}{
        \foreach \y in {0,1,...,7}{
          \zeroDot{\x}{\y}
        }
      }
      \foreach \y in {0,...,7}{
        %\node at (5,\y) {$\cdots$};
        \zeroDot{5}{\y}
      }
      \node at (0,0) {\small $\Z_{(p)} 1$};
      \zeroDot{0}{1}
      \node at (0,2) {\small $\Z_{(p)} a$};
      \zeroDot{0}{3}
      \zeroDot{0}{4}
      \zeroDot{0}{5}
      \zeroDot{0}{6}
      \node at (0,7) {$\Z_p b$};
      \node at (3,0) {\small $\Z_{(p)} x$};
      \zeroDot{3}{1}
      \node at (3,2) {\small $\Z_{(p)} ax$};
      \zeroDot{3}{3}
      \zeroDot{3}{4}
      \zeroDot{3}{5}
      \zeroDot{3}{6}
      \node at (3,7) {\small $\Z_{p} bx$};
      \node at (6,0) {$\cdots$};
      \zeroDot{6}{1}
      \node at (6,2) {$\cdots$};
      \zeroDot{6}{3}
      \zeroDot{6}{4}
      \zeroDot{6}{5}
      \zeroDot{6}{6}
      \node at (6,7) {$\cdots$};
      \node at (8,0) {\small $\Z_{p} y$};
      \zeroDot{8}{1}
      \node at (8,2) {\small $\Z_p a y$};
      \zeroDot{8}{3}
      \zeroDot{8}{4}
      \zeroDot{8}{5}
      \zeroDot{8}{6}
      \node at (8,7) {$*$};
      \draw[->,gray] (0.6,1.6) -- (2.4,0.4);
      \draw[->,gray] (3.6,1.6) -- (5.4,0.4);
      \draw[->,gray] (6.5,1.7) -- (8.6,0.3);
      \draw[->,gray] (0.5,6.7) -- (7.5,0.4);
      \draw[->] (-0.7,-0.6) -- (8.5,-0.6); %node[below] {p};
      \draw[->] (-0.7,-0.6) -- (-0.7,7.5); %node[left] {q};
      \node[left] at (-0.7,0) {$0$};
      \node[left] at (-0.8,1.2) {$\vdots$};
      \node[left] at (-0.7,2) {$n-1$};
      %\node[left] at (-0.7,4) {$2(n-1)$};
      \node[left] at (-0.8,4.7) {$\vdots$};
      \node[left] at (-0.7,7) {$2p + n - 2$};
      \node[below] at (0,-0.6) {$0$};
      \node[below] at (1.5,-0.6) {$\cdots$};
      \node[below] at (3,-0.6) {$n$};
      \node[below] at (4.5,-0.7) {$\cdots$};
      \node[below] at (6,-0.7) {$qn$};
      \node[below] at (8,-0.6) {$2p+n-1$};
    \end{tikzpicture}
  \end{center}

  Aus der Induktionsannahme sowie Satz~\ref{rational-homology-kzn} folgt, dass die $0$-te Spalte wie abgebildet ist.
  Das Muster in der $0$-ten und der $(n-1)$-ten Zeile begründet sich wie im Beweis von Satz~\ref{rational-homology-kzn}.
  Insbesondere ist der Kohomologiering $H^{< 2p + n - 1}(K(\Z, n); \Z_{(p)}) \coloneqq \oplus_{i=0}^{2p + n - 2} H^i(K(\Z, n); \Z_{(p)})$ von $K(\Z, n)$ im Grad kleiner $2p + n - 2$ isomorph zu $(\Z_{(p)}[a]) / (a^l)$ mit Generator $a \in H^n(K(\Z, n); \Z_{(p)})$ und $l \coloneqq \ceil{\tfrac{2p + n - 1}{n}}$ ist.
  Dies werden wir im nächsten Fall benötigen.
  Wir behaupten, dass $E_{2p + n - 1}^{0, 2p + n - 2} \cong E_2^{0, 2p + n - 2} \cong \Z_p$. %das eingezeichnete Differential $d_{2p + n - 1}^{0, 2p + n - 2} : E_{2p + n - 1}^{0, 2p + n - 2} \to E_{2p + n - 1}^{2p + n - 1, 0}$ ein Isomorphismus ist.
  Dies folgt daraus, dass das Ziel der Differentiale $E_r^{0, 2p + n - 2}$ für $r < 2p + n - 1$ die Nullgruppe ist, mit einer Ausnahme:
  Für $n = 2p$ könnte $d_n^{0, 2p + n - 2} : E_n^{0, 2p + n - 2} \to E_n^{n, n - 1}$ nichttrivial sein.
  Das ist aber nicht der Fall, denn $\im(d_n^{0, 2p + n - 2}) \subseteq \ker(d_n^{n, n-1}) = 0$.
  Man sieht außerdem leicht, dass $E_{2p + n - 1}^{2p + n - 1, 0} \cong E_2^{2p + n - 1, 0}$.
  Das Differential $d_{2p + n - 1}^{0, 2p + n - 2}$ ist wie eben ein Isomorphismus.
  Somit gilt
  \begin{align*}
    H^{2p + n - 1}(K(\Z, n)) \otimes \Z_{(p)} & \cong H^{2p + n - 1}(K(\Z, n); \Z_{(p)}) \cong H^{2p + n - 1}(K(\Z, n); H^0(K(\Z, n-1); \Z_{(p)})) \\
    & \cong E_2^{2p + n - 1, 0} \cong E_{2p + n - 1}^{2p + n - 1, 0} \cong E_{2p + n - 1}^{0, 2p + n - 2} \cong E_2^{0, 2p + n - 2} \cong \Z_p.
  \end{align*}

  \emph{Falls $n > 3$, $n$ ungerade:} \enspace

  \begin{center}
    \begin{tikzpicture}[x=32pt,y=16pt]
      \foreach \x in {1,2,4,6}{
        \foreach \y in {0,1,...,7}{
          \zeroDot{\x}{\y}
        }
      }
      \foreach \y in {0,...,7}{
        %\node at (5,\y) {$\cdots$};
        \zeroDot{5}{\y}
      }
      \node at (0,0) {\small $\Z_{(p)} 1$};
      \zeroDot{0}{1}
      \node at (0,2) {\small $\Z_{(p)} a$};
      \zeroDot{0}{3}
      \node at (0,4) {\small $\Z_{(p)} a^2$};
      \zeroDot{0}{5}
      \node at (0,6.2) {$\vdots$};
      \node at (0,7) {$\Z_p b$};
      \node at (3,0) {\small $\Z_{(p)} x$};
      \zeroDot{3}{1}
      \node at (3,2) {\small $\Z_{(p)} ax$};
      \zeroDot{3}{3}
      \node at (3,4.2) {$\vdots$};
      \zeroDot{3}{5}
      \node at (3,6.2) {$\vdots$};
      \node at (3,7) {\small $\Z_{p} bx$};
      \node at (7,0) {\small $\Z_{p} y$};
      \zeroDot{7}{1}
      \node at (7,2) {\small $\Z_p a y$};
      \zeroDot{7}{3}
      \node at (7,4.2) {$\vdots$};
      \zeroDot{7}{5}
      \node at (7,6.2) {$\vdots$};
      \node at (7,7) {$*$};
      \draw[->,gray] (0.6,1.6) -- (2.4,0.4);
      \draw[->,gray] (0.6,3.6) -- (2.4,2.4);
      \draw[->,gray] (0.6,5.6) -- (2.4,4.4);
      \draw[->,gray] (0.6,7.6) -- (2.4,6.4);
      \draw[->,gray] (0.5,6.7) -- (6.5,0.4);
      \draw[->] (-0.7,-0.6) -- (7.5,-0.6); %node[below] {p};
      \draw[->] (-0.7,-0.6) -- (-0.7,7.5); %node[left] {q};
      \node[left] at (-0.7,0) {$0$};
      \node[left] at (-0.8,1.2) {$\vdots$};
      \node[left] at (-0.7,2) {$n-1$};
      \node[left] at (-0.7,4) {$2(n-1)$};
      \node[left] at (-0.8,6.2) {$\vdots$};
      \node[left] at (-0.7,7) {$2p + n - 2$};
      \node[below] at (0,-0.6) {$0$};
      \node[below] at (1.5,-0.6) {$\cdots$};
      \node[below] at (3,-0.6) {$n$};
      \node[below] at (5,-0.7) {$\cdots$};
      \node[below] at (7,-0.6) {$2p+n-1$};
    \end{tikzpicture}
  \end{center}

  In diesem Fall argumentieren wir ähnlich wie im Fall $n = 3$.
  Wir haben eben im geraden Fall gezeigt, dass die erste Spalte wie abgebildet ist.
  Wie im Fall $n = 3$ sieht man, dass $H^n(K(\Z, n); \Z_{(p)}) \cong E_2^{n,0} \cong \Z_{(p)}$.
  Die Morphismen $d_n^{0,k (n-1)} : E_n^{0, k (n-1)} \to E_n^{n, (k-1) (n-1)}$ sind Isomorphismen falls erstens $k (n-1) < 2p + n - 2$ und zweitens $k < p$ gilt.
  Die zweite Bedingung folgt dabei aus der ersten, denn dann ist
  \[ k < \tfrac{2p}{n-1} + \tfrac{n - 2}{n-1} < \tfrac{2p}{4} + 1 \leq p. \]
  Wir sehen somit, dass auf der $E_{n+1}$-Seite alle Einträge in der $0$-ten Spalte unterhalb der $(2p + n - 2)$-ten Zeile und alle Einträge in der $n$-ten Spalte unterhalb der $(2p - 1)$-ten Zeile verschwinden.
  Die Spalten mit Dimension $m = n+1, \ldots, 2p + n - 2$ verschwinden bereits auf der $E_2$-Seite, da es kein nichttriviales Differential gibt, welches ihren Eintrag in der $0$-ten Zeile trifft.
  Folglich ist $E_{2p + n - 1}^{0, 2p + n - 1} \cong E_{2}^{0, 2p + n - 1}$ und das eingezeichnete Differential $d_{2p + n - 1}^{0, 2p + n - 1} : E_{2p + n - 1}^{0, 2p + n - 1} \to E_{2p + n - 1}^{2p + n - 1, 0}$ ein Isomorphismus.
  Somit ist
  \begin{align*}
    H^{2p + n - 1}(K(\Z, n)) \otimes \Z_{(p)} & \cong H^{2p + n - 1}(K(\Z, n); \Z_{(p)}) \cong H^{2p + n - 1}(K(\Z, n); H^0(K(\Z, n-1); \Z_{(p)})) \\
    & \cong E_2^{2p + n - 1, 0} \cong E_{2p + n - 1}^{2p + n - 1, 0} \cong \Z_p.
  \end{align*}
\end{proof}

Mit dem universellen Koeffiziententheorem folgt:

\begin{kor}\label{mod-p-homology-kzn}
  Es sei $n \geq 3$ und $p$ prim.
  Dann ist die p-Torsionsuntergruppe von $H_i(K(\Z, n); \Z)$ null für $i < n + 2p - 2$ und isomorph zu $\Z_p$ für $i = n + 2p - 2$. \qed
\end{kor}

\begin{satz}
  Sei $n \geq 3$ und $p$ eine Primzahl.
  Dann ist die $p$-Torsionskomponente von $\pi_i(S^n)$ null falls $i < n + 2p - 3$ und isomorph zu $\Z_p$, falls $i = n + 2p - 3$.
\end{satz}

\begin{proof}
  Es sei $\T_p \coloneqq \T_{\Primes \setminus \{ p \}}$ die Serre-Klasse der endlichen abelschen Gruppen, deren Ordnung nicht durch $p$ teilbar ist.
  Wir nehmen zunächst an, dass $p \neq 2$.

  Es sei $n$ ungerade.
  Wir betrachten den Abbildungszylinder~$M_i$ der Abbildung $i : S^n \to \tau_{\leq n} S^n = K(\Z, n)$.
  Die lange exakte Homologiesequenz des Paares $(M_i, S^n)$ und Lemma~\ref{mod-p-homology-kzn} induzieren, dass
  \[
   H_i(M_i, S^n) \cong \begin{cases}
     0 & i < n + 2p - 2, \\
     \Z_p & i = n + 2p - 2
   \end{cases}
   \qquad \pmod{\T_p}.
  \]
  Mit Satz~\ref{relative-hurewicz-mod-c} folgt
  \[
   \pi_i(M_i, S^n) \cong \begin{cases}
     0 & i < n + 2p - 2, \\
     \Z_p & i = n + 2p - 2
   \end{cases}
   \qquad \pmod{\T_p}.
  \]
  Dies zeigt die Aussage für ungerade~$n$, da $\pi_{n+2p-3}(S^n) \cong \pi_{n+2p-3}(M_i, S^n)$, wie man an der langen exakten Homotopiesequenz von $(M_i, S^n)$ sieht.

  Es sei nun~$n$ gerade.
  Wir betrachten wieder die Faserung $S^{n-1} \to \V{2}{n+1} \to S^n$.
  Die Abbildung $q : \V{2}{n+1} \to S^{2n-1}$ induziert Isomorphismen in Homologie modulo~$\T_p$ und somit auch in Homotopie modulo~$\T_p$.
  Aus der langen exakten Homotopiesequenz
  \[
    \ldots
    \enspace\to\enspace
    \pi_{i+1}(S^n)
    \enspace\to\enspace
    \underbrace{\pi_i(S^{n-1})}_{\mathclap{\substack{\text{$p$-torsionsfrei} \\ \text{für } i < n + 2p - 4, \\ \text{endlich für} \\ i \neq n-1}}}
    \enspace\to\enspace \underbrace{\pi_i(\V{2}{n+1})}_{\mathclap{\substack{\cong \pi_i(S^{2n-1}) \bmod{\T_p}, \\ \text{$p$-torsionsfrei} \\ \text{für } i < 2n + 2p - 4, \\ \text{endlich für } i \neq 2n-1}}}
    \enspace\to\enspace
    \pi_i(S^n)
    \enspace\to\enspace
    \pi_{i-1}(S^{n-1})
    \enspace\to\enspace
    \ldots
  \]
  folgt, dass $\pi_i(S^n)$ für $i < n + 2p - 3$ keine $p$-Torsion besitzt.
  Aus dieser Sequenz folgt außerdem, dass die $p$-Torsionsuntergruppe von $\pi_i(S^n)$ für $i = n + 2p - 3$ isomorph ist zu $\Z_p$, sofern $n + 2p - 3 \neq 2n - 1$ gilt\footnote{Im Fall $i = n + 2p - 3 = 2n - 1$ könnte nämlich die exakte Sequenz $\pi_i(\V{2}{n+1}) \to \pi_i(S^n) \to \pi_{i-1}(S^{n-1})$ isomorph zu $\Z \xrightarrow{\cdot p} \Z \to \Z_p$ sein, die $p$-Torsionkomponente von $\pi_i(S^n)$ könnte also auch null sein.}.
  
  Für $i = n + 2p - 3 = 2n - 1$ (und allgemeiner $i = n + 2p - 3 < 2n$) geben wir daher ein alternatives Argument an:
  Es sei~$M_i$ der Abbildungzylinder von $i : S^n \to \tau_{\leq n} S^n = K(\Z, n)$.
  Die lange exakte Homologiesequenz des Paars $(M_i, S^n)$ impliziert, dass $H_j(M_i, S^n) \cong H_j(K(\Z, n))$ für $j \neq n$ und dass $H_n(M_i, S^n) = 0$.
  Insbesondere gilt $H_j(M_i, S^n) \in \T_p$ für $j \leq i$ und die p-Torsion von $H_{i+1}(M_i, S^n) \cong H_{i+1}(K(\Z, n)) \bmod{\T_p}$ ist isomorph zu $\Z_p$  nach Korollar~\ref{mod-p-homology-kzn} und Lemma~\ref{rational-homology-kzn}.
  Somit gilt $\pi_j(M_i, S^n) \in \T_p$ für $j \leq i$ und die p-Torsion von $\pi_{i+1}(M_i, S^n)$ ist ebenfalls isomorph zu $\Z_p$.
  Die Aussage folgt nun mit der langen exakten Homotopiesequenz von $(M_i, S^n)$.
  
  Dieses alternative Argument liefert außerdem die Behauptung im Fall $p=2$ und $n \geq 3$ beliebig.
\end{proof}

\begin{bem}
  Für $n=2$ stimmt die Aussage des Satzes nicht, denn
  $\pi_i(S^n) \cong \pi_{i-1}(S^3)$ besitzt für $i < 2 + 2p - 2$ keine $p$-Torsion und die $p$-Torsion von $\pi_{2 + 2p - 2}(S^n)$ ist $\Z_p$.
\end{bem}

\nocite{*}
\addcontentsline{toc}{section}{Literatur}
\printbibliography

\newpage
\addcontentsline{toc}{section}{Anhang: Tabelle der Homotopiegruppen der Sphären}
\begin{landscape}
  \begin{center}
    \mbox{}\vspace{8em}
    \begin{table}
      \footnotesize \setlength\extrarowheight{5pt}
      \begin{tabular}{r | c c c c c c c c c c c | l}
  $\pi_{n+k}(S^n)$ & $n{=}1$ & $n{=}2$ & $n{=}3$ & $n{=}4$ & $n{=}5$ & $n{=}6$ & $n{=}7$ & $n{=}8$ & $n{=}9$ & $n{=}10$ & $n{=}11$ & $\pi^s_k$ \\[0.2em] \hline
  %$k < 0$ & $0$ & $\cdots$ \\[2pt] \hhline{~-}
  $k = 0$ & \multicolumn{1}{c|}{\cellcolor{gray!15} $\Z$} & \cellcolor{gray!15} $\Z$ & \cellcolor{gray!15} $\cdots$ & \cellcolor{gray!15} $\cdots$ & \cellcolor{gray!15} $\cdots$ & \cellcolor{gray!15} $\cdots$ & \cellcolor{gray!15} $\cdots$ & \cellcolor{gray!15} $\cdots$ & \cellcolor{gray!15} $\cdots$ & \cellcolor{gray!15} $\cdots$ & \cellcolor{gray!15} $\cdots$ & \cellcolor{gray!15} $\Z$ \\[2pt] \hhline{~~-}
  $k = 1$ & $0$ & \multicolumn{1}{c|}{\cellcolor{gray!15} $\Z$} & $\ZZ{2}{}$ & $\cdots$ & $\cdots$ & $\cdots$ & $\cdots$ & $\cdots$ & $\cdots$ & $\cdots$ & $\cdots$ & $\ZZ{2}{}$ \\[2pt] \hhline{~~~-}
  $k = 2$ & $0$ & $\ZZ{2}{}$ & \multicolumn{1}{c|}{$\ZZ{2}{}$} & $\ZZ{2}{}$ & $\cdots$ & $\cdots$ & $\cdots$ & $\cdots$ & $\cdots$ & $\cdots$ & $\cdots$ & $\ZZ{2}{}$ \\[2pt] \hhline{~~~~-}
  $k = 3$ & $0$ & $\ZZ{2}{}$ & $\ZZ{4}{} {\times} \ZZ{3}{}$ & \multicolumn{1}{c|}{\cellcolor{gray!15} $\Z {\times} \ZZ{4}{} {\times} \ZZ{3}{}$} & $\ZZ{8}{} {\times} \ZZ{3}{}$ & $\cdots$ & $\cdots$ & $\cdots$ & $\cdots$ & $\cdots$ & $\cdots$ & $\ZZ{8}{} {\times} \ZZ{3}{}$ \\[2pt] \hhline{~~~~~-}
  $k = 4$ & $0$ & $\ZZ{4}{} {\times} \ZZ{3}{}$ & $\ZZ{2}{}$ & $\ZZ{2}{2}$ & \multicolumn{1}{c|}{$\ZZ{2}{}$} & $0$ & $\cdots$ & $\cdots$ & $\cdots$ & $\cdots$ & $\cdots$ & $0$ \\[2pt] \hhline{~~~~~~-}
  $k = 5$ & $0$ & $\ZZ{2}{}$ & $\ZZ{2}{}$ & $\ZZ{2}{2}$ & $\ZZ{2}{}$ & \multicolumn{1}{c|}{\cellcolor{gray!15} $\Z$} & $0$ & $\cdots$ & $\cdots$ & $\cdots$ & $\cdots$ & $0$ \\[2pt] \hhline{~~~~~~~-}
  $k = 6$ & $0$ & $\ZZ{2}{}$ & $\ZZ{3}{}$ & $\ZZ{8}{} {\times} \ZZ{3}{2}$ & $\ZZ{2}{}$ & $\ZZ{2}{}$ & \multicolumn{1}{c|}{$\ZZ{2}{}$} & $\ZZ{2}{}$ & $\cdots$ & $\cdots$ & $\cdots$ & $\ZZ{2}{}$ \\[2pt] \hhline{~~~~~~~~-}
  $k = 7$ & $0$ & $\ZZ{3}{}$ & $\ZZ{3}{} {\times} \ZZ{5}{}$ & $\ZZ{3}{} {\times} \ZZ{5}{}$ & $\ZZ{2}{} {\times} \ZZ{3}{} {\times} \ZZ{5}{}$ & $\ZZ{4}{} {\times} \ZZ{3}{} {\times} \ZZ{5}{}$ & $\ZZ{8}{} {\times} \ZZ{3}{} {\times} \ZZ{5}{}$ & \multicolumn{1}{c|}{\cellcolor{gray!15} $\Z {\times} \ZZ{8}{} {\times} \ZZ{3}{} {\times} \ZZ{5}{}$} & $\ZZ{16}{} {\times} \ZZ{3}{} {\times} \ZZ{5}{}$ & $\cdots$ & $\cdots$ & $\ZZ{16}{} {\times} \ZZ{3}{} {\times} \ZZ{5}{}$ \\[2pt] \hhline{~~~~~~~~~-}
  $k = 8$ & $0$ & $\ZZ{3}{} {\times} \ZZ{5}{}$ & $\ZZ{2}{}$ & $\ZZ{2}{}$ & $\ZZ{2}{}$ & $\ZZ{8}{} {\times} \ZZ{2}{} {\times} \ZZ{3}{}$ & $\ZZ{2}{3}$ & $\ZZ{2}{4}$ & \multicolumn{1}{c|}{$\ZZ{2}{3}$} & $\ZZ{2}{2}$ & $\cdots$ & $\ZZ{2}{2}$ \\[2pt] \hhline{~~~~~~~~~~-}
  $k = 9$ & $0$ & $\ZZ{2}{}$ & $\ZZ{2}{2}$ & $\ZZ{2}{3}$ & $\ZZ{2}{3}$ & $\ZZ{2}{3}$ & $\ZZ{2}{4}$ & $\ZZ{2}{5}$ & $\ZZ{2}{4}$ & \multicolumn{1}{c|}{\cellcolor{gray!15} $\Z {\times} \ZZ{2}{3}$} & $\ZZ{2}{3}$ & $\ZZ{2}{3}$ \\[2pt]
  %$k = 10$ & $0$ & $\ZZ{2}{2}$ & $\ZZ{4}{} {\times} \ZZ{2}{} {\times} \ZZ{3}{}$ & $\ZZ{8}{} {\times} \ZZ{4}{} {\times} \ZZ{2}{} {\times} \ZZ{3}{2} {\times} \ZZ{5}{}$ & $\ZZ{8}{} {\times} \ZZ{2}{} {\times} \ZZ{9}{}$ & $\ZZ{8}{} {\times} \ZZ{2}{} {\times} \ZZ{9}{}$ & $\ZZ{8}{} {\times} \ZZ{3}{} {\times} \ZZ{2}{}$ & $\ZZ{8}{2} {\times} \ZZ{2}{} {\times} \ZZ{3}{2}$ & $\ZZ{8}{} {\times} \ZZ{3}{} {\times} \ZZ{2}{}$ & $\ZZ{4}{} {\times} \ZZ{3}{} {\times} \ZZ{2}{}$ & $\ZZ{2}{2} {\times} \ZZ{3}{}$ & $\ZZ{2}{} {\oplus} \ZZ{3}{}$
\end{tabular}

      \caption{Die ersten Homotopiegruppen der Sphären. Der stabile Bereich besteht aus allen Gruppen oberhalb der Treppenlinie. Unendliche Gruppen sind grau hinterlegt. Quelle: \cite{toda:composition}}
    \end{table}
  \end{center}
\end{landscape}

% TODO:
% * James F. Davis, Paul Kirk zitieren
% * Bemerkung über Serresche Quotientenkategorien
% * Relativen Hurewicz-Homomorphismus definieren
% * Beweise umformulieren: "man" vermeiden
% * Irgendwo schreiben, dass der Grund für die triviale Wirkung in der ganzen Arbeit der ist, dass die Fundamentalgruppe schon verschwindet.
% * Motivation für Studium der Homotopiegruppen der Sphären beschreiben
% * Switzer/anderen zu Pontryagin-Thom zitieren
% * Abschnitt über das Modulo-C-Hurewicz-Theorem aufteilen?
% * Verwendung von Hurewicz- und Serre-Faserungen überprüfen

\end{document}
