\documentclass[11pt, a4paper, german]{article}

\usepackage[utf8]{inputenc}
\usepackage[ngerman]{babel}
\usepackage{BA_Titelseite}
%\usepackage{amsmath,amsthm,amsfonts,amssymb}
\usepackage{amsmath,amsthm,amssymb}
\usepackage{tikz}
\usepackage{enumitem} % bessere Aufzählungen
\usepackage{mathtools} % \coloneqq
\usetikzlibrary{cd}
\usepackage{geometry}
\geometry{margin=2.5cm}
\usepackage{xfrac}
\usepackage{color} % \red

\author{Tim Baumann}
\geburtsdatum{15. Juni 1994}
\geburtsort{Friedberg}
\date{\today}

\betreuer{Betreuer: Prof. Dr. Bernhard Hanke}
\zweitgutachter{Zweitgutachter: Prof. Dr. X Y}
\institut{Institut für Mathematik}
\title{Spektralsequenzen und der Satz von Serre}
\ausarbeitungstyp{Bachelorarbeit Mathematik}

\theoremstyle{definition}
%\newtheorem*{nota}{Notation}
\newtheorem*{bsp}{Beispiel}
\newtheorem*{satz}{Satz}
\newtheorem*{lem}{Lemma}
\newtheorem*{defn}{Definition}
\newtheorem*{beob}{Beobachtung}

\theoremstyle{remark}
\newtheorem*{bem}{Bemerkung}

\newcommand{\TODO}[1]{\textcolor{red}{TODO: #1}} %\red{#1}} % TODO-Markierungen

% Zahlbereiche
\newcommand{\R}{\mathbb{R}} % Reelle Zahlen
\newcommand{\N}{\mathbb{N}} % Natürliche Zahlen
\newcommand{\Z}{\mathbb{Z}} % Ganze Zahlen
%\newcommand{\C}{\mathbb{C}} % Komplexe Zahlen
%\newcommand{\Q}{\mathbb{Q}} % Rationale Zahlen

% Schöne Fürall- und Existenzquantoren
%\newcommand{\fa}[1]{\forall \, {#1} \,:\,}
%\newcommand{\ex}[1]{\exists \, {#1} \,:\,}

\DeclareMathOperator{\id}{id} % Identität
\DeclareMathOperator{\spann}{spann} % Spann
\DeclareMathOperator{\im}{im} % Image (Bild)
\newcommand{\pt}{\mathrm{pt}}
\newcommand{\blank}{\text{--}} % Platzhalter
\newcommand{\ntimes}{\!\times\!} % schmaleres (narrower) \times
\newcommand{\angles}[1]{{\langle #1 \rangle}}
\newcommand{\const}[1]{\text{konst } #1} % konstante Funktion mit Wert #1
\newcommand{\LH}{\mathcal{H}} % Local homology
\DeclareMathOperator{\colim}{colim} % Kolimes

% Abkürzungen
\newcommand{\ES}{Es sei} % Prof. Hanke schreibt das immer, mag das anscheinend lieber als ein bloßes "Sei"
\newcommand{\ESn}{Es seien} % Prof. Hanke schreibt das immer, mag das anscheinend lieber als ein bloßes "Sei"
\newcommand{\zB}{z.\,B.}
\renewcommand{\dh}{d.\,h.} % das heißt

% Intervalle
\newcommand{\cinterval}[2]{\left[ #1, #2 \right]} % closed interval
\newcommand{\I}{I} % Kompaktes Einheitsinterval [0,1]
%\newcommand{\I}{\cinterval{0}{1}} % Kompaktes Einheitsinterval [0,1] (alternativ)

% Schöne Mengen { #1 | #2 } (benötigt mathtools)
% siehe http://tex.stackexchange.com/questions/13634/define-pretty-sets-in-latex-esp-how-to-do-the-condition-separator
\DeclarePairedDelimiterX\Set[2]{\lbrace}{\rbrace}%
 { #1 \,\delimsize|\, #2 }

% http://tex.stackexchange.com/questions/117732/tikz-and-babel-error
% Es ist schierer Wahnsinn, welche Hacks LaTeX benötigt!
\tikzset{
  every picture/.prefix style={
    execute at begin picture=\shorthandoff{"}
  }
}

% Zentrierte kommutative Diagramme
\newenvironment{centertikzcd}
  {\begin{center}\begin{tikzcd}}
  {\end{tikzcd}\end{center}}

\begin{document}

\maketitle

\section{Spektralsequenzen}

\subsection{Faserungen}

\begin{defn}
  Eine \emph{Serre-Faserung} ist eine stetige Abbildung $p : E \to B$, welche die \emph{Homotopieliftungseigenschaft} (HLE) für die Scheiben $D^n$ besitzt, \dh{}
  für alle $n \geq 0$ und für alle stetigen Abbildungen $H$, $H_0$ wie unten, sodass das äußere Quadrat kommutiert, gibt es eine stetige Abbildung $\tilde{H}$, sodass die beiden Dreiecke kommutieren:
  \begin{centertikzcd}[row sep=1.2cm, column sep=1.4cm]
    D^n \arrow[r, "H_0"] \arrow[d, hook, "i_0", swap] &
    E \arrow[d, "p"] \\
    D^n \times \I \arrow[r, "H"] \arrow[ur, "\exists\, \tilde{H}", dashed] &
    B
  \end{centertikzcd}
  Dabei ist $i_0$ die Inklusion von $D^n$ in $D^n \times \I$ als $D^n \times \{ 0 \}$. \\
  Eindeutigkeit von $\tilde{H}$ wird nicht gefordert.
\end{defn}

\begin{lem}
  \ES{} $p : E \to B$ eine stetige Abbildung. Dann sind äquivalent:
  \begin{enumerate}[label=\alph*)]
    \item $p$ ist eine Serre-Faserung
    \item $p$ besitzt die \emph{relative Homotopieliftungseigenschaft} für CW-Paare, \dh{} für alle CW-Paare $(X, A)$ und für alle $H_0$ und $H$ wie unten, sodass das äußere Quadrat kommutiert, gibt eine stetige Abbildung $\tilde{H}$, sodass die beiden Dreiecke kommutieren:
    \begin{centertikzcd}[row sep=1.2cm, column sep=1.4cm]
      X \!\times\! \{ 0 \} \cup A \!\times\! \I \arrow[r, "H_0"] \arrow[d, hook, swap] &
      E \arrow[d, "p"] \\
      X \times \I \arrow[r, "H"] \arrow[ur, "\exists\, \tilde{H}", dashed] &
      B
    \end{centertikzcd}
  \end{enumerate}
\end{lem}

% vgl. Diskussion im Hatcher auf Seite 376
\begin{proof}
  "`b) $\implies$ a)"' \enspace Folgt sofort mit $(X, A) \coloneqq (D^n, \emptyset)$. \\[2pt]
  "`a) $\implies$ b)"' \enspace Wir behandeln zunächst den Fall $(X, A) = (D^n, S^{n-1})$, $n \in \N$. Dann ist $(D^n \times \I, D^n \times \{ 0 \} \cup S^{n-1} \cup \I) \approx (D^n)$ homöomorph als Raumpaar.
  Somit ist die relative Homotopieliftungseigenschaft in diesem Fall gleichbedeutend zur Homotopieliftungseigenschaft für die Scheibe $D^n$. \\
  \ES{} nun $(X, A)$ ein beliebiges Raumpaar.
  % TODO: Ausführlicher?
  Dann kann man induktiv die Homotopie $H$ auf die $i$-Zellen $e^i_\alpha$ von $X \setminus A$ fortsetzen.
  Dabei ist die Homotopie auf $S^{n-1} = \partial D^n$ durch die Komposition der bisher konstruierten Homotopie mit der anheftenden Abbildung $\phi_\alpha : S^{n-1} \to X^{n-1}$ vorgegeben.
  Man erhält die Fortsetzung durch Anwenden des zuerst bewiesenen Falls.
\end{proof}

% Lemma 4.5 in http://www.math.washington.edu/~mitchell/Notes/serre.pdf
\begin{lem}
  \ESn{} $p : E \to B$ eine Serre-Faserung, $b_0 \in B$, $F \coloneqq p^{-1}(b_0)$ die Faser über $b_0$ und $f_0 \in F$.
  Dann gibt es eine lange exakte Sequenz
  \[ \ldots \to \pi_n(F, f_0) \xrightarrow{i_*} \pi_n(E, f_0) \xrightarrow{p_*} \pi_n(B, b_0) \xrightarrow{\partial} \pi_{n-1}(F, f_0) \to \ldots \to \pi_1(B, b_0) \]
  von Homotopiegruppen.
  Dabei ist $i : F \hookrightarrow E$ die Inklusion.
\end{lem}

\begin{proof}
  Die gesuchte exakte Sequenz ist die lange exakte Homotopiesequenz
  \[ \ldots \to \pi_n(F, f_0) \xrightarrow{i_*} \pi_n(E, f_0) \to \pi_n(E, F, f_0) \xrightarrow{\partial} \pi_{n-1}(F, f_0) \to \ldots \to \pi_1(E, F, f_0) \]
  des Raumpaares $(E, F)$.
  Es bleibt zu zeigen: $\pi_n(E, F, f_0) \cong \pi_n(B, b_0)$ als Gruppe für $n > 1$ und als punktierte Menge für $n = 1$.
  Der Isomorphismus muss außerdem so gewählt werden, dass
  \[ p_* = \left( \pi_n(E, f_0) \to \pi_n(E, F, f_0) \xrightarrow{\cong} \pi_n(B, b_0) \right). \]
  Wir zeigen: $p_* : \pi_n(E, F, f_0) \to \pi_n(B, b_0)$ ist der gesuchte Isomorphismus (damit ist obige Gleichung erfüllt). \\[2pt]
  \emph{Surjektivität}: Sei $[g : (I^{n+1}, \partial I^{n+1}, b_0) \to (B, \{ b_0 \}, b_0)] \in \pi_{n+1}(B, b_0)$, $n \geq 0$.
  Sei $\tilde{g}$ der Lift im folgenden relativen HLE-Diagramm:
  \begin{centertikzcd}[row sep=1.2cm, column sep=1.4cm]
    U \arrow[r, "\const{f_0}"] \arrow[d, hook, swap] &
    E \arrow[d, "p"] \\
    I^n \times I \arrow[r, "g"] \arrow[ur, "\exists\, \tilde{g}", dashed] \arrow[r, "g"] &
    B
  \end{centertikzcd}
  wobei $U \coloneqq I^n \!\times\! \{ 0 \} \cup (\partial I^n) \!\times\! I \subset I^{n+1}$.
  Dann kann man $\tilde{g}$ als eine Abbildung $(I^{n+1}, \partial I^{n+1}, U) \to (E, F, \{ f_0 \})$ von Raumtripeln auffassen, welche ein Element von $\pi_{n+1}(E, F, f_0)$ repräsentiert.
  Es gilt $p_*[\tilde{g}] = [p \circ \tilde{g}] = [g]$. \\[2pt]
  \emph{Injektivität}: Seien $[h_0], [h_1] \in \pi_{n+1}(E, F, f_0)$ mit $p_*[h_0] = p_*[h_1]$.
  Sei
  \[ H : I \times I^{n+1}, \quad (t, x) \mapsto H_t(x) \]
  eine Homotopie mit $H_0 = p \circ h_0$, $H_1 = p \circ h_1$, welche zu jedem Zeitpunkt $t \in \I$ eine Abbildung $H_t : (I^{n+1}, \partial I^{n+1}) \to (B, \{ b_0 \})$ von Raumpaaren ist.
  Betrachte folgendes HLE-Diagramm:
  \begin{centertikzcd}[row sep=1.2cm, column sep=1.8cm]
    V \arrow[r, "h"] \arrow[d, hook, swap] &
    E \arrow[d, "p"] \\
    I^{n+1} \times I \arrow[r, "H"] \arrow[ur, "\exists\, \tilde{H}", dashed] &
    B
  \end{centertikzcd}
  mit $V \coloneqq I^{n+1} \!\times\! \{ 0 \} \cup (\partial I^{n+1}) \!\times\! I \subset I^{n+2}$ und
  \[
    h|_{\{0\} \times I^{n+1}} \coloneqq h_0, \quad
    h|_{\{1\} \times I^{n+1}} \coloneqq h_1, \quad
    h|_{I \times U} \coloneqq \const{f_0}.
  \]
  Nun ist $\tilde{H}$ eine Homotopie von $h_0$ nach $h_1$, welche zu jedem Zeitpunkt $t$ eine Abbildung $\tilde{H}_t : (I^{n+1}, \partial I^{n+1}, U) \to (E, F, \{ b_0 \})$ von Raumtripeln ist.
\end{proof}

\begin{defn}
  \ESn{} $p : E \to B$ und $g : X \to B$ stetig.
  Der \emph{Pullback} von $p$ entlang $g$ ist die Abbildung $g^*(p) : g^*(E) \to X$, wobei $g^*(E) \coloneqq X \times_B E$ das Faserprodukt von $X$ und $E$ über $B$ vermöge $g$ und $p$ ist.
\end{defn}

\begin{bem}
  Pullback ist funktoriell: $(g \circ f)^* = f^* \circ g^*$ und $\id^* = \id$.
\end{bem}

\begin{lem}
  Pullbacks von Serre-Faserungen sind Serre-Faserungen.
\end{lem}

\begin{proof}
  Sei $p : E \to B$ eine Serre-Faserung und $g : X \to B$ stetig.
  Wir müssen die Existenz des Morphismus $\tilde{H}$ im folgenden Diagramm zeigen:
  \begin{centertikzcd}[row sep=1.2cm, column sep=1.4cm]
    D^n \arrow[r, "H_0"] \arrow[d, hook, "i_0", swap] &
    g^*(E) \arrow[d, "g^*(p)"] \arrow[r, "h"] \arrow[dr, phantom, "\ulcorner", very near start] &
    E \arrow[d, "p"] \\
    D^n \times \I \arrow[r, "H"] \arrow[ur, dashed, "\tilde{H}"] &
    X \arrow[r, "g"] &
    B
  \end{centertikzcd}
  Aus der HLE von $p$ erhält wie folgt einen Morphismus $K$:
  \begin{centertikzcd}[row sep=1.2cm, column sep=1.4cm]
    D^n \arrow[r, "H_0"] \arrow[d, hook, "i_0", swap] &
    X \times_B E \arrow[r, "h"] &
    E \arrow[d, "p"] \\
    D^n \times \I \arrow[r, "H"] \arrow[urr, "K", dashed] &
    X \arrow[r, "g"] &
    B
  \end{centertikzcd}
  Nun ist $D^n \times \I$ vermöge $H$ und $K$ ein Kegel über dem Diagramm $(X \xrightarrow{g} B \xleftarrow{p} E)$.
  Die universelle Eigenschaft von $g^*(E)$ induziert einen Morphismus $\tilde{H} : D^n \times \I \to X \times_B E$ mit
  $g^*(p) \circ \tilde{H} = H$ und $h \circ \tilde{H} = K$.
  Aus der univ. Eigenschaft von $g^*(E)$ (Eindeutigkeit) folgt nun $\tilde{H} \circ i_0 = H_0$.
  % weil: $h \circ H_0 = K \circ i_0 = h \circ \tilde{H} \circ i_0$
\end{proof}

\begin{defn}
  Ein Morphismus $(g, \tilde{g}) : p' \to p$ von Serre-Faserungen $p' : E' \to B'$ und $p : E \to B$ ist ein kommutatives Quadrat der Form
  \begin{centertikzcd}
    E' \arrow[r, "\tilde{g}"] \arrow[d, "p'"] &
    E \arrow[d, "p"] \\
    B' \arrow[r, "g"] &
    B
  \end{centertikzcd}
\end{defn}

\begin{bsp}
  Pullback einer Serre-Faserung $p$ entlang einer stetigen Abbildung $g$ induziert einen Morphismus $(g, \tilde{g}) : g^*(p) \to p$ von Serre-Faserungen.
\end{bsp}

\begin{lem}
  Die langen exakten Sequenzen der Homotopiegruppen von Faserungen sind natürlich: \ES{} $(g, \tilde{g}) : p' \to p$ ein Morphismus von Serre-Faserungen $p' : E' \to B'$ und $p : E \to B$, $b_0' \in B'$, $b_0 \coloneqq g(b_0')$, $F' \coloneqq p'^{-1}(b_0')$, $F \coloneqq p^{-1}(b_0)$, $f'_0 \in F'$, $f_0 \coloneqq \tilde{g}(f'_0)$. Dann gibt es eine "`Leiter"' bestehend aus kommutativen Quadraten zwischen den Homotopiesequenzen:
  \begin{centertikzcd}[row sep=0.8cm, column sep=0.6cm]
    \ldots \arrow[r] &
    \pi_n(F', f'_0) \arrow[r, "i'_*"] \arrow[d, "(\tilde{g}|_{F'})_*"] &
    \pi_n(E', f'_0) \arrow[r, "p'_*"] \arrow[d, "\tilde{g}_*"] &
    \pi_n(B', b'_0) \arrow[r, "\partial"] \arrow[d, "g_*"] &
    \pi_{n-1}(F', f'_0) \arrow[r] \arrow[d, "(\tilde{g}|_{F'})_*"] &
    \ldots \\
    \ldots \arrow[r] &
    \pi_n(F, f_0) \arrow[r, "i_*"] &
    \pi_n(E, f_0) \arrow[r, "p_*"] &
    \pi_n(B, b_0) \arrow[r, "\partial"] &
    \pi_{n-1}(F, f_0) \arrow[r] &
    \ldots
  \end{centertikzcd}
\end{lem}

\begin{proof}
  Folgt aus der Natürlichkeit der langen exakten Homotopiesequenz von Raumpaaren.
\end{proof}

\ES{} $p : E \to B$ eine Serre-Faserung, $\gamma : \I \to B$ ein stetiger Weg.
Betrachte die lange exakte Sequenz
% TODO: Aufpunkte, Zusammenhang, H_0?
\[ \ldots \to \pi_n(F_{\gamma(0)}) \to \pi_n(\gamma^*(E)) \to \pi_n(\I) \to \pi_{n-1}(F_{\gamma(0)}) \to \ldots \]
der Homotopiegruppen von $\gamma^*(p) : \gamma^*(E) \to \I$ mit Faser
\[ F_{\gamma(t)} \coloneqq \gamma^*(p)^{-1}(t)  \subset \gamma^*(E) = \Set{(t, e) \in \I \times E}{\gamma(t) = p(e)}. \]
In dieser Sequenz sind die Gruppen $\pi_n(\I)$ trivial.
Folglich sind die Abbildungen $(i_{\gamma(t)})_* : \pi_n(F_{\gamma(t)}, *) \to \pi_n(\gamma^*(E), *)$ Isomorphismen. % TODO: Anderer Name für die Aufpunkte
In anderen Worten: $i_{\gamma(t)}$ ist eine schwache Äquivalenz.
Aus einem Korollar des Whitehead-Theorems folgt nun, dass $i_t$ auch in Homologie und Kohomologie Isomorphismen induziert (vgl. Spanier, AT, S. 406, Cor 7.6.25). % TODO: Besser referenzieren!
Wir untersuchen den Isomorphismus
\[ T_\gamma \coloneqq (i_{\gamma(1)})^* \circ ((i_{\gamma(0)})^*)^{-1} \enspace:\enspace H^*(F_{\gamma(0)}) \xrightarrow{\cong} H^*(F_{\gamma(1)}). \]

\begin{lem}
  $T_\gamma$ hängt lediglich von der Weghomotopieklasse von $\gamma$ ab, \dh{} ist $\eta$ ein zweiter Weg mit $\gamma \simeq \eta$, so gilt $T_\gamma = T_\eta$.
\end{lem}

\begin{proof}
  Sei $H : \I \times \I \to B$ eine Homotopie zw. den Wegen $\gamma$ und $\eta$, \dh{} $H_0 \coloneqq H(0, \blank) = \gamma$, $H_1 = \eta$, $H(\blank, 0) \equiv x$ und $H(\blank, 1) \equiv y$ mit $x \coloneqq \gamma(0) \!=\! \eta(0)$ und $y \coloneqq \gamma(1) \!=\! \eta(1)$.
  Für festes $s \in \I$ sei $i_s : \I \to \I \times \I, \enspace t \mapsto (s, t)$ die Inklusion als $\{ s \} \times I$.
  Betrachte das kommutative Diagramm
  \begin{centertikzcd}[row sep=1.2cm, column sep=1.4cm]
    H_s^*(E) \arrow[r, hook, "\widetilde{i_s}"] \arrow{d}{}[swap]{H_s^*(p)} \arrow[dr, phantom, "\ulcorner", very near start] &
    H^*(E) \arrow[r] \arrow[d, swap, "H^*(p)"] \arrow[dr, phantom, "\ulcorner", very near start] &
    E \arrow[d, "p"] \\
    \I \arrow[r, hook, "i_s"] \arrow[rr, bend right, "H_s"] &
    \I \!\times\! \I \arrow[r, "H"] &
    B
  \end{centertikzcd}
  Sei $t \in \I$ fest.
  Sei $F_{s,t} \coloneqq \left( H_s^*(p) \right)^{-1}(t) = \left( H^*(p) \right)^{-1}((s, t))$ und $f_0 \in F$. % TODO: Warum gibt es solch ein $f_0$?
  Das linke komm. Diagramm induziert einen Morphismus zw. den langen ex. Homotopieseq. von $H_t^*(p)$ und~$H^*(p)$:
  \begin{centertikzcd}[row sep=0.8cm, column sep=0.8cm]
    \ldots \arrow[r] &
    \pi_{n+1}(I, t) \arrow[r, "\partial"] \arrow[d, "i_{s*}"] &
    \pi_n(F_{s,t}, f_0) \arrow[r, "(i'_{s,t})*"] \arrow[d, equal] &
    \pi_n(H_s^*(E), f_0) \arrow{r}{H_s^*(p)_*}{} \arrow{d}{(\widetilde{i_s})_*}{} &
    \pi_n(\I, t) \arrow[r] \arrow{d}{i_{s*}}{} &
    \ldots \\
    \ldots \arrow[r] &
    \pi_{n+1}(I \!\times\! I, (s, t)) \arrow[r, "\partial"] &
    \pi_n(F_{s,t}, f_0) \arrow[r, "(i_{s,t})_*"] &
    \pi_n(H^*(E), f_0) \arrow[r, "H^*(p)_*"] &
    \pi_n(\I \!\times\! \I, (s, t)) \arrow[r] &
    \ldots
  \end{centertikzcd}
  In diesen Sequenzen verschwinden die Gruppen $\pi_n(I, t)$ bzw. $\pi_n(I \ntimes I, (s, t))$.
  Folglich induzieren die Abbildungen $\widetilde{i_s}$ Isomorphismen in Homotopie und in Kohomologie.
  Es gilt nun
  \begin{align*}
    T_\gamma &
    = (i'_{0,1})^* \circ ((i'_{0,0})^*)^{-1}
    = (i'_{0,1})^* \circ (\widetilde{i_0})^* \circ (\widetilde{i_0})^{-1} \circ ((i'_{0,0})^*)^{-1} \\
    & = (i_{0,1})^* \circ ((i_{0,0})^*)^{-1}
    \stackrel{(\star)}{=} (i_{1,1})^* \circ ((i_{1,0})^*)^{-1} \\
    & = (i'_{1,1})^* \circ (\widetilde{i_1})^* \circ (\widetilde{i_1})^{-1} \circ ((i'_{1,0})^*)^{-1}
    = (i'_{1,1})^* \circ ((i'_{1,0})^*)^{-1}
    = T_\eta.
  \end{align*}
  Die Gleichung $(\star)$ gilt wegen $i_{0,1} \simeq i_{1,1}$ und $i_{0,0} \simeq i_{1,0}$.
\end{proof}

Mit ganz ähnlicher Technik kann man zeigen:

\begin{lem}
  Seien $\gamma, \eta : I \to B$ stetige Wege mit $\gamma(1) = \eta(0)$.
  Dann gilt
  \[ T_\eta \circ T_\gamma = T_{\gamma \centerdot \eta} \enspace:\enspace H^*(F_{\gamma(0)}) \xrightarrow{\cong} H^*(F_{\eta(1)}). \]
  Dabei ist die Komposition $\gamma \centerdot \eta$ von $\gamma$ und $\eta$ folgender Weg:
  \[
    \gamma \centerdot \eta : I \to B, \quad
    s \mapsto \begin{cases}
      \gamma(2s), & \text{falls } s \in [0, \tfrac{1}{2}], \\
      \eta(2s - 1), & \text{falls } s \in [\tfrac{1}{2}, 1].
    \end{cases}
  \]
\end{lem}

\begin{proof}
  Betrachte folgendes kommutatives Diagramm:
  \begin{centertikzcd}[row sep=1.2cm, column sep=1.4cm]
    \gamma^*(E) \arrow[r, hook, "\widetilde{j}"] \arrow{d}{}[swap]{\gamma^*(p)} \arrow[dr, phantom, "\ulcorner", very near start] &
    (\gamma \centerdot \eta)^*(E) \arrow[r] \arrow[d, swap, "(\gamma \centerdot \eta)^*(p)"] \arrow[dr, phantom, "\ulcorner", very near start] &
    E \arrow[d, "p"] \\
    \I \arrow[r, hook, "j"] \arrow[rr, bend right, "\gamma"] &
    \I \arrow[r, "\gamma \centerdot \eta"] &
    B
  \end{centertikzcd}
  Dabei ist $j : I \to I$ die Abbildung $s \mapsto \sfrac{s}{2}$.
  Analog zum letzten Lemma sieht man anhand des Leiterdiagramms der langen exakten Sequenzen der Faserungen $\gamma^*(p)$ und $(\gamma \centerdot \eta)^*(p)$, dass $\widetilde{j}$ einen Isomorphismus in Homotopie und Kohomologie induziert.
  Es gibt ein ähnliches Diagramm mit $\eta$ statt $\gamma$ und $k : I \to I, \enspace s \mapsto \sfrac{(1+s)}{2}$ statt $j$.
  Es induziert auch $\widetilde{k}$ einen Isomorphismus in Kohomologie.
  Es gilt nun
  \begin{align*}
    T_\eta \circ T_\gamma
    & = (i_{\eta(1)})^* \circ ((i_{\eta(0)})^*)^{-1} \circ (i_{\gamma(1)})^* \circ ((i_{\gamma(0)})^*)^{-1} \\
    & = (i_{\eta(1)})^* \circ \tilde{k}^* \circ (\tilde{k}^*)^{-1} \circ ((i_{\eta(0)})^*)^{-1} \circ (i_{\gamma(1)})^* \circ \tilde{j}^* \circ (\tilde{j}^*)^{-1} \circ ((i_{\gamma(0)})^*)^{-1} \\
    & = (\tilde{k} \circ i_{\eta(1)})^* \circ ((\tilde{k} \circ i_{\eta(0)})^*)^{-1} \circ (\tilde{j} \circ i_{\gamma(1)})^* \circ ((\tilde{j} \circ i_{\gamma(0)})^*)^{-1} \\
    & = (i_{\gamma \centerdot \eta(1)})^* \circ ((i_{\gamma \centerdot \eta(\sfrac{1}{2})})^*)^{-1} \circ (i_{\gamma \centerdot \eta(\sfrac{1}{2})})^* \circ ((i_{\gamma \centerdot \eta(0)})^*)^{-1} \\
    & = (i_{\gamma \centerdot \eta(1)})^* \circ ((i_{\gamma \centerdot \eta(0)})^*)^{-1} = T_{\gamma \centerdot \eta}. \qedhere
  \end{align*}
\end{proof}

\subsection{Lokale Koeffizienten}

\iffalse
% hier nach Hatcher, AT, Kapitel 3.H

Sei im Folgenden $B$ wegzusammenhängend.
Wir haben gezeigt, dass dann alle Fasern diesselbe Kohomologiegruppen besitzen.
Wir können also von der Kohomologie der Faser $F$ sprechen, wobei $F \coloneqq p^{-1}(b_0)$ für ein beliebiges $b_0 \in B$.
Die Fundamentalgruppe $\pi_1(B, b_0)$ operiert auf $H^*(F)$ durch $[\gamma] \mapsto T_\gamma$.

% TODO: Welche Topologie trägt $G$? Diskret?
\begin{defn}
  Ein \emph{Bündel von Gruppen} auf einem topologischen Raum $B$ besteht aus einer Gruppe $G$, einer Abbildung $q : A \to B$ und Bijektionen $g_b : q^{-1}(b) \xrightarrow{\cong} G$ für alle $b \in B$, welche den Fasern von $q$ eine zu $G$ isomorphe Gruppenstruktur verleihen, sodass gilt:
  Für jeden Punkt $x \in B$ gibt es eine Umgebung $U \subset B$, sodass
  \[
    g_U : q^{-1}(U) \to U \times G, \quad
    y \mapsto (q(y), g_{q(y)}(y))
  \]
  stetig und sogar ein Homöomorphismus ist.
\end{defn}

\begin{bsp}
  Sei $G$ eine Gruppe.
  Die Projektion $B \times G \to B$ ist in kanonischer Weise ein Bündel von Gruppen auf $B$.
\end{bsp}
\fi

% hier nach http://isites.harvard.edu/fs/docs/icb.topic880873.files/local_systems.pdf

\begin{defn}
  Ein \emph{lokales Koeffizientensystem} $\underline{A}$ auf einem topologischen Raum $B$ besteht aus abelschen Gruppen $(A_b)_{b \in B}$ und Isomorphismen $T_\gamma : A_{\gamma(0)} \xrightarrow{\cong} A_{\gamma(1)}$ für jeden stetigen Weg $\gamma : I \to B$, sodass gilt:
  \begin{itemize}
    \item Sind zwei Wege $\gamma, \eta : I \to B$ homotop modulo Endpunkte, so gilt $T_\gamma = T_\eta$.
    \item Für komponierbare Wege $\gamma, \eta : I \to B$ gilt $T_{\gamma \centerdot \eta} = T_\eta \circ T_\gamma$.
    \item Für den konstanten Weg $\gamma \equiv b$ gilt $T_\gamma = \id_{A_b}$.
  \end{itemize}
\end{defn}

\begin{bem}
  % TODO: sagt man "lokales Koeffizientensystem" oder nur "lokales System"?
  Man kann ein lokales Koeffizientensystem auf $B$ auch als Funktor aus dem Fundamentalgruppoid von $B$ in die Kategorie der abelschen Gruppen auffassen.
\end{bem}

\begin{bsp}
  Im letzten Abschnitt wurde gezeigt: Bei einer Serre-Faserung $p : E \to B$ bilden die $q$-ten Kohomologiegruppen $A_b \coloneqq H^q(p^{-1}(b))$ der Fasern ein lokales Koeffizientensystem.
  Wir bezeichnen dieses Koeffizientensystem im Folgenden mit $\LH^q(F_p)$. % XXX: Das ist ne doofe Notation. Ändern?
\end{bsp}

\begin{bsp}
  Für jede abelsche Gruppe $G$ gibt es das konstante Koeffizientensystem $\underline{G}$ mit $G_b \coloneqq G$ für alle $b \in B$ und $T_\gamma = \id_G$ für alle $\gamma : I \to B$.
\end{bsp}

Sei im Folgenden $\Delta_n(B)$ die Menge der $n$-Simplizes in $B$, also die Menge der stetigen Abbildungen $\Delta^n \to B$ mit $\Delta^n \coloneqq \spann \{ e_0, \ldots, e_n \} \subset \R^{n+1}$, und
\[
  d_n : \Delta_n(B) \to \Delta_{n-1}(B) \quad
  \sigma \mapsto \sigma_{\angles{e_0, \ldots, \hat{e_i}, \ldots, e_n}} \qquad (0 \leq i \leq n),
\]
die Abbildung auf die $i$-Seite. % TODO: kann man das "Projektion" nennen?
Für einen $n$-Simplex $\sigma$ bezeichne $\sigma_i \coloneqq \sigma_{\angles{e_i}} \in \Delta_0(B) = B$ die $i$-te Ecke und $\sigma_{ij} \coloneqq \sigma_{\angles{e_i, e_j}} \in \Delta_1(B)$ den Weg von von $\sigma_i$ nach $\sigma_j$ entlang der $ij$-Kante von $\sigma$ ($0 \leq i \leq j \leq n$).

\begin{defn}
  Sei $B$ ein topologischer Raum, $\underline{A}$ ein lokales Koeffizientensystem auf $B$.
  Der \emph{Kokomplex der singulären Koketten auf $B$ mit Koeffizienten in $\underline{A}$} ist folgendermaßen definiert:
  \begin{align*}
    C^n(B; \underline{A}) \coloneqq & \, \{ \text{ Abbildungen $f$, welche einem $n$-Simplex $\sigma \in \Delta_n(B)$} \\
    & \enspace\,\, \text{ ein Element $f(\sigma) \in A_{\sigma_0}$ zuordnen } \} \\[2pt]
    \delta^n(f) \coloneqq & \, (\sigma \in \Delta_{n+1}(B)) \mapsto T_{\sigma_{01}}^{-1}(f(d_0(\sigma))) + \sum_{i=1}^{n+1} (-1)^i f(d_i(\sigma)).
  \end{align*}
  Man überprüft leicht, dass $\delta^{n+1} \circ \delta^n = 0$ gilt.
  Die Kohomologie $H^*(B; \underline{A}) \coloneqq H^*(C^*(B; \underline{A}))$ dieses Kettenkomplexes heißt \emph{singuläre Kohomologie von $B$ mit Koeffizienten in $\underline{A}$}.
\end{defn}

\begin{beob}
  Für das konstante Koeffizientensystem $\underline{G}$ gilt $H^*(B; \underline{G}) \cong H^*(B; G)$.
  Gewöhnliche Kohomologie mit Koeffizienten ist also ein Spezialfall von Kohomologie mit Koeffizienten in einem lokalen System.
\end{beob}

\subsection{Spektralsequenzen}

% TODO: Einleitendes Gefasel zu Spektralsequenzen

\ES{} $A$ im Folgenden ein kommutativer Ring mit Eins.

\begin{defn}
  Eine (kohomologische) \emph{Spektralsequenz} besteht aus
  \begin{itemize}
    \item $A$-Moduln $E_r^{p,q}$ für alle $p, q \in \Z$ und $r \geq 1$,
    \item $A$-Modul-Homomorphismen $d_r^{p,q} : E_r^{p,q} \to E_r^{p+r,q-r+1}$ mit $d_r^{p+r,q-r+1} \circ d_r^{p,q} = 0$
    \item und Isomorphismen $\alpha_r^{p,q} : H^{p,q}(E_r) \!\coloneqq\! \ker(d_r^{p,q}) / \im(d_r^{p-r,q+r-1}) \xrightarrow{\cong} E_{r+1}^{p,q}$.
  \end{itemize}
\end{defn}


\begin{bem}
  \begin{itemize}
    \item Die Homomorphismen $d^r_{p,q}$ heißen \emph{Differentiale}.
    \item Die Gesamtheit der Module $E^r_{p,q}$ und Differentiale $d_r^{pq}$ mit $r \in \N$ fest heißt $r$-te \emph{Seite} $E^r$.
    \item Man stellt Seiten für gewöhnlich in einem 2-dimensionalen Raster dar:
  \end{itemize}
  \begin{center}
    \begin{tikzpicture}[x=12pt,y=12pt]
      \begin{scope}[shift={(0,0)}]
        \foreach \x in {-1,...,4}{
          \foreach \y in {-2,-1,...,3}{
            \node[draw,circle,inner sep=0.5pt,fill] at (\x,\y) {};
          }
        }
        \foreach \x in {-1,...,5}{
          \foreach \y in {-2,-1,...,3}{
            \draw[->,gray] (\x-0.8,\y) -- (\x-0.2,\y);
          }
        }
        \draw[->] (-0.35,-0.35) -- (5.35,-0.35) node[below] {p};
        \draw[->] (-0.35,-0.35) -- (-0.35,4) node[left] {q};
        \node at (5,4.5) {$E_1$};
      \end{scope}
      \begin{scope}[shift={(10,0)}]
        \foreach \x in {-1,...,4}{
          \foreach \y in {-2,-1,...,3}{
            \node[draw,circle,inner sep=0.5pt,fill] at (\x,\y) {};
          }
        }
        \foreach \x in {0,...,5}{
          \foreach \y in {-2,-1,...,3}{
            \draw[->,gray] (\x-1.8,\y+0.9) -- (\x-0.2,\y+0.1);
          }
        }
        \draw[->] (-0.35,-0.35) -- (5.35,-0.35) node[below] {p};
        \draw[->] (-0.35,-0.35) -- (-0.35,4) node[left] {q};
        \node at (5,4.5) {$E_2$};
      \end{scope}
      \begin{scope}[shift={(20,0)}]
        \foreach \x in {-1,...,4}{
          \foreach \y in {-2,-1,...,3}{
            \node[draw,circle,inner sep=0.5pt,fill] at (\x,\y) {};
          }
        }
        \foreach \x in {1,...,5}{
          \foreach \y in {-2,-1,...,2}{
            \draw[->,gray] (\x-2.8,\y+1.9) -- (\x-0.2,\y+0.1);
          }
        }
        \draw[->] (-0.35,-0.35) -- (5.35,-0.35) node[below] {p};
        \draw[->] (-0.35,-0.35) -- (-0.35,4) node[left] {q};
        \node at (5,4.5) {$E_3$};
      \end{scope}
    \end{tikzpicture}
  \end{center}
\end{bem}

\begin{defn}
  Eine Spektralsequenz \emph{konvergiert}, falls für alle $p, q \in \Z$ ein $R \in \N$ existiert, sodass für alle $r \geq R$ die Differentiale von und nach $E_r^{p,q}$ null sind und damit $E^\infty_{p,q} \coloneqq E_R^{p,q} \cong E_{R+1}^{p,q} \cong \ldots$ \\[2pt]
  Der Grenzwert der SS ist die Unendlich-Seite $E_\infty \coloneqq \{ E_\infty^{p,q} \}_{p,q}$.
\end{defn}

\begin{bem}
  Viele Spektralsequenzen sind im ersten Quadranten konzentriert, \dh{} $E_r^{p,q}$ ist nur für $p, q \geq 0$ ungleich Null.
  Solche Spektralsequenzen konvergieren immer, denn für alle $p, q \in \Z$ führen für $r \geq \max(p+1, q+2)$ alle Differentiale von $E_r^{p,q}$ aus dem ersten Quadranten heraus und alle dort eintreffenden Differentiale kommen von außerhalb des ersten Quadranten und sind daher Null.
\end{bem}

\begin{defn}
  Eine \emph{Filtrierung} eines $A$-Moduls $M$ ist eine absteigende Folge
  \[ M \ldots \supseteq F^{p-1} M \supseteq F^p M \supseteq F^{p+1} M \supseteq \ldots \]
  von Untermoduln von $M$, $p \!\in\! \Z$.
  Eine Filtrierung heißt
  \begin{itemize}
    \item \emph{ausschöpfend}, falls $M = \cup_p F^p$,
    \item \emph{Hausdorffsch}, wenn $0 = \cap_p F^p M$ und
    \item \emph{regulär}, wenn sie ausschöpfend und Hausdorffsch ist.
  \end{itemize}
\end{defn}

\begin{defn}
  Eine Spektralsequenz $E$ \emph{konvergiert gegen} einen graduierten $A$-Modul $M = \oplus_{n \in \Z} M_n$ (notiert $E_r^{p,q} \Rightarrow M^{p+q}$), falls $E$ überhaupt konvergiert und reguläre Filtrierungen
  \[ M_n \supseteq \ldots \supseteq F^{p-1} M_n \supseteq F^p M_n \supseteq F^{p+1} M_n \supseteq \ldots \]
  existieren, sodass $E_\infty^{pq} \cong F^p M_{p+q} / F^{p+1} M_{p+q}$ für alle $p, q \in \Z$.
\end{defn}

\subsection{Die Spektralsequenz eines filtrierten Komplexes}

% Hier nach http://people.mpim-bonn.mpg.de/matschke/SpectralSequences.pdf

\begin{defn}
  Eine \emph{Filtrierung eines Kokettenkomplexes} $C^\bullet$ ist eine absteigende Folge
  \[ C^\bullet \supseteq \ldots \supseteq F^{p-1} C^\bullet \supseteq F^p C^\bullet \supseteq F^{p+1} C^\bullet \supseteq \ldots \]
 von Unterkomplexen.
\end{defn}

% TODO: Morphismus von filtrierten Komplexen?

\begin{lem}
  \ES{} $C^\bullet$ ein filtrierter Kokettenkomplex.
  Es gibt eine Spektralsequenz mit
  \[ E_1^{pq} = H^{p+q}(F^p C^\bullet / F^{p+1} C^\bullet). \]
  Angenommen, die Filtrierung ist
  \begin{enumerate}[label=\alph*)]
    \item \emph{gradweise nach unten beschränkt}, \dh{} für alle $q \in \Z$ gibt es ein $p \in \Z$ mit $F^p C^q = 0$,
    \item \emph{ausschöpfend}, \dh{} für alle $q \in \Z$ ist $\cup_p F^p C^q = C^q$ und %(aus der ersten Bedingung folgt schon $\cap_p F^p C^q = 0$).
    \item für alle $q \in \Z$ gibt es ein $P \in \Z$, sodass für alle $p \leq P$ gilt: Die Inklusion $F^p C^\bullet \hookrightarrow C^\bullet$ induziert einen Isomorphismus $H^q (F^p C^\bullet) \cong H^q(C^\bullet)$ in Kohomologie.
  \end{enumerate}
  Dann konvergiert die Spektralsequenz gegen $H^*(C^\bullet)$.
\end{lem}

Wir führen zunächst etwas neue Notation ein.
Diese hilft, den Beweis verständlicher zu formulieren.
Wir fassen im Folgenden den Kettenkomplex als ein einziges Modul $C \coloneqq \oplus_{n \in \Z} C^n$ anstatt als Folge von Modulen auf.
Dieses Modul ist filtriert durch die Untermodule $F^p \coloneqq \oplus_{n \in \Z} F^p C^n$.
Wir setzen $F^{-\infty} \coloneqq C$ und $F^{\infty} \coloneqq 0$.
Die Korandabbildung fassen wir als Homomorphismen $d : C \to C$ mit $d \circ d = 0$ auf, der die Filtrierung von $C$ respektiert.

Wir sind interessiert an der Kohomologie von $C^\bullet$, also an $H^*(C) \coloneqq \ker(d) / \im(d)$ und an der Kohomologie von $F^p / F^{p+1}$, also $H^*(F^p / F^{p+1}) \cong (d|_{F^p})^{-1}(F^{p+1}) / d(F^p)$.
Wir geben nun eine Verallgemeinerung der Definition der Kohomologie von $C^\bullet$ und der Kohomologie des Quotientenkomplexes $F^p / F^q$: Statt Zykeln (\dh{} Elementen $c \in C$ mit $d(c) = 0$) betrachten wir \emph{$z$-Zykel}, das sind Elemente $c \in C$ mit $d(c) \in F^z$. Wir teilen diese durch die Menge $d(F^b)$ der \emph{$b$-Ränder} anstatt durch die Menge $d(C)$ der Ränder. Wir setzen

\[ S[z, q, p, b] \coloneqq \frac{F^p \cap d^{-1}(F^z)}{(F^p \cap d^{-1}(F^z)) \cap (F^q + d(F ^b))}. \]

% TODO: Bemerkung zur grafischen Notation von Matschke

Wir haben als Spezialfälle
\[
  S[p,q,p,q] \cong F^p / F^q
  \quad \text{und} \quad
  S[q,q,p,p] \cong H^*(F^p / F^q).
\]

\begin{lem}
  \ES{} $z_1 \geq q_1 \geq p_1 = z_2 \geq b_1 = q_2 \geq p_2 \geq b_2$.
  Dann ist folgende Abbildung ein wohldefinierter Homomorphismus:
  \[
    d^* : S[z_2, q_2, p_2, b_2] \to S[z_1, q_1, p_1, b_1], \quad
    [c] \mapsto [d(c)].
  \]
\end{lem}

\begin{proof}
  Falls $[c] = 0$ in $S[z_2, q_2, p_2, b_2]$, so existieren $x \in F^{q_2}$ und $y \in F^{b_2}$ mit $c = x + d(y)$. Somit gilt $d^*[c] = [dc] = [d(x) + d^2(y)] = [d(x)] = 0$ in $S[z_1, q_1, p_1, b_1]$, da $F^{b_1} = F^{q_2}$.
\end{proof}

\begin{lem}
  Es seien Filtrierungsindizes wie folgt gegeben:
  \begin{centertikzcd}[row sep=0.2cm, column sep=0.5cm]
    &&&& z_3 \arrow[r, "\geq" description, phantom] \arrow[d, equal] &
    q_3 \arrow[r, "\geq" description, phantom] \arrow[d, equal] &
    p_3 \arrow[r, "\geq" description, phantom] &
    b_3 \\

    && z_2 \arrow[r, "\geq" description, phantom] \arrow[d, equal] &
    q_2 \arrow[r, "\geq" description, phantom] \arrow[d, equal] &
    p_2 \arrow[r, "\geq" description, phantom] &
    b_2 \\

    z_1 \arrow[r, "\geq" description, phantom] &
    q_1 \arrow[r, "\geq" description, phantom] &
    p_1 \arrow[r, "\geq" description, phantom] &
    b_1
  \end{centertikzcd}
  Dann ist
  \[
    \alpha : S[q_1, q_2, p_2, p_3] \to
    \frac{\ker(d^* : S[z_2, q_2, p_2, b_2] \to S[z_1, q_1, p_1, b_1])}{\im(d^* : S[z_3, q_3, p_3, b_3] \to S[z_2, q_2, p_2, b_2])}, \quad
    [c] \mapsto [c]
  \]
  ein wohldefinierter Isomorphismus.
\end{lem}

\begin{proof}
  Sei $A$ der Quotient auf der rechten Seite. \\[2pt]
  \emph{Wohldefiniertheit}: Sei $[c] = 0$ in $S[q_1, q_2, p_2, p_3]$, \dh{} es gibt $e \in F^{q_2} = F^{b_1}$ und $f \in F^{p_1}$ mit $c = e + d(f)$. Dann ist $d^*[c] = [d(c)] = [d(e)] = 0$ in $S[z_1, q_1, p_1, b_1]$, also $c \in \ker(d^* : S[z_2, q_2, p_2, b_2] \to S[z_1, q_1, p_1, b_1])$.
  Nun ist $f \in d^{-1}(F^{z_3})$, da $d(f) = c - e \in F^{p_2} = F^{z_3}$.
  Es gilt $[c] = [e + d(f)] = [d(f)] = d^*[f] = 0$ in $A$. \\[2pt]
  \emph{Injektivität}: Sei $c \in F^{p_2} \cap d^{-1}(F^{q_1})$ mit $[c] = 0$ in $A$.
  Das heißt, es gibt $e \in F^{q_2}$, $f \in F^{b_2}$ und $g \in F^{p_3} \cap d^{-1}(F^{z_3})$ mit $c = e + d(f) + d(g)$.
  Dann ist $[c] = [e + d(f+g)] = 0$ in $S[q_1, q_2, p_2, p_3]$, da $f+g \in F^{p_3}$. \\[2pt]
  \emph{Surjektivität}: Sei $\tilde{c} \in F^{p_2} \cap d^{-1}(F^{z_2})$ mit $[\tilde{c}] \in \ker(d^* : S[z_2, q_2, p_2, b_2] \to S[z_1, q_1, p_1, b_1])$.
  Das heißt, es gibt $e \in F^{q_1}$ und $f \in F^{b_1} = F^{q_2}$ mit $d(\tilde{c}) = e + d(f)$.
  Dann ist $[\tilde{c}] = [\tilde{c} - f]$ in $S[q_1, q_2, p_2, p_3]$ mit $\tilde{c} - f \in F^{p_2} \cap d^{-1}(F^{q_1})$, da $d(\tilde{c} - f) = e \in F^{q_1}$.
\end{proof}

\begin{proof}[Beweis des Lemmas über Existenz der Spektralsequenz]
  Wir beachten jetzt wieder, dass $C$ und damit $S[z, q, p, b]$ graduiert und $d$ ein Differential vom Grad $+1$ ist.
  Es sei $S[z, q, p, b]^n$ die $n$-te Komponente.
  Setze
  \[ E_r^{pq} \coloneqq S[p\!+\!r, p\!+\!1, p, p\!-\!r\!+\!1]^{p+q}. \]
  Die Differentiale sind
  \[
    d_r^{pq}
    \enspace : \enspace
    \underbrace{S[p\!+\!r, p\!+\!1, p, p\!-\!r\!+\!1]^{p+q}}_{= E_r^{p,q}} \to \underbrace{S[p\!+\!2r, p\!+\!r\!+\!1, p\!+\!r, p\!+\!1]^{p+q+1}}_{= E_r^{p+r,q-r+1}}, \quad
    [c] \mapsto [d(c)].
  \]
  Sie sind wohldefiniert nach Lemma \TODO{Nr}.
  Wegen Lemma \TODO{Nr} ist
  \[
    \alpha_r^{pq} : H^{p,q}(E_r) = \ker(d^{pq}_r) / \im(d^{p-r,q+r-1}_r) \to E_{r+1}^{pq}, \quad
    [c] \mapsto [c]
  \]
  ein wohldefinierter Isomorphismus.

  \emph{Beweis der Konvergenz}:
  Es seien $p, q \in \Z$.
  Wegen Bedingung a) gibt es ein $R_1 \geq 0$, sodass $F^{p+R_1} C^{p+q+1} = 0$.
  Für $r \geq R_1$ ist damit $E^{p+r,q-r+1}_r$ als Subquotient (\dh{} Quotient eines Untermoduls) von $F^{p+R_1} C^{p+q+1}$ Null.
  Folglich verschwindet auch das Differential $d_r^{pq}$.
  Wegen Bedingung c) gibt es ein $S \in \Z$, sodass $F^s C^\bullet \hookrightarrow C^\bullet$ und somit auch $F^s C^\bullet \hookrightarrow F^{s-1} C^\bullet$ für $s \leq S$ einen Isomorphismus in $H^{p+q-1}$ und $H^{p+q}$ induziert.
  Anhand der langen exakten Sequenz zu $0 \to F^s C^\bullet \to F^{s-1} C^\bullet \to F^{s-1} C^\bullet / F^s C^\bullet \to 0$ sieht man, dass $H^{p+q-1}(F^{s-1} C^\bullet / F^s C^\bullet) = 0$.
  Somit ist $E_r^{p-r,q+r-1}$ für $r \geq R_2 \coloneqq p - s + 1$ als Submodul von $H^{p+q-1}(F^{p-r} C^\bullet / F^{p-r+1} C^\bullet)$ Null.
  Folglich verschwindet auch $d_r^{p-r,q+r-1}$.
  Mit $R \coloneqq \max(R_1, R_2)$ gilt dann $E^{pq}_R \cong E^{pq}_{R+1} \cong \ldots \cong E^{pq}_\infty$.

  Sei $H^n(C^\bullet)$ absteigend filtriert durch $F^p H^n(C^\bullet) \coloneqq \im(i^* : H^n(F^p C^\bullet) \to H^n(C^\bullet))$.
  Für $r \geq R$ ist
  \[ E^{pq}_\infty \cong E^{pq}_r = \frac{F^p C^{p+q} \cap d^{-1}(0)}{(F^p C^{p+q} \cap d^{-1}(0)) \cap (F^{p+1} C^{p+q} + d(F^{p-r+1} C^{p+q-1}))} = S[\infty,p+1,p,p-r+1]^{p+q}. \]
  Es ist daher $F^p H^{p+q}(C^\bullet) / F^{p+1} H^{p+q}(C^\bullet) \cong S[\infty, p+1, p, -\infty]^{p+q}$ ein Quotient von $E^{pq}_\infty$.
  Tatsächlich gilt $S[\infty, p+1, p, -\infty]^{p+q} \cong E^{pq}_\infty$, denn:
  Sei $c \in F^p C^{p+q} \cap d^{-1}(0)$ mit $[c] = 0$ in $S[\infty, p+1, p, -\infty]^{p+q}$.
  Dann gibt es ein $e \in F^{p+1} C^{p+q}$ und ein $f \in C^{p+q-1}$ mit $c = e + d(f)$.
  Wegen Bedingung b) gibt es ein $\tilde{p} \in \Z$ mit $f \in F^{\tilde{p}} C^{p+q+1}$.
  Wähle $r$ so, dass $r \geq R$ und $p-r+1 \leq \tilde{p}$.
  Dann ist $[c] = [e]+[d(f)] = 0$ in $E^{pq}_r \cong E^{pq}_\infty$.
\end{proof}


\subsection{Die Serre-Spektralsequenz}

\begin{satz}[Jean-Pierre Serre]
  Für jede Serre-Faserung $p : E \to B$ existiert eine Spektralsequenz mit
  \[ E_2^{p,q} = H^p(B; \LH^q(F_p)), \]
  welche gegen $H^*(E)$ konvergiert.
\end{satz}

% Eilenberg-MacLane-Räume

\end{document}
