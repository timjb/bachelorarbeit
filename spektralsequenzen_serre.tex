\documentclass[11pt, a4paper, german]{article}

\usepackage[utf8]{inputenc}
\usepackage[ngerman]{babel}
\usepackage{BA_Titelseite}
\usepackage{amsmath,amsthm,amssymb}
\usepackage{tikz}
\usepackage{enumitem} % bessere Aufzählungen
\usepackage{mathtools} % \coloneqq
\usepackage{stmaryrd} % \mapsfrom
\usetikzlibrary{cd}
\usepackage{geometry}
\geometry{margin=2.5cm}
\usepackage{xfrac}
\usepackage{color} % \red
%\usepackage{wrapfig} % Bild neben Text
\usepackage{cutwin} % Bild neben Text
\usepackage{mathabx} % \divides

\author{Tim Baumann}
\geburtsdatum{15. Juni 1994}
\geburtsort{Friedberg}
\date{\today}

\betreuer{Betreuer: Prof. Dr. Bernhard Hanke}
\zweitgutachter{Zweitgutachter: Prof. Dr. X Y}
\institut{Institut für Mathematik}
\title{Spektralsequenzen und der Satz von Serre}
\ausarbeitungstyp{Bachelorarbeit Mathematik}

\theoremstyle{definition}
%\newtheorem*{nota}{Notation}
\newtheorem{bsp}{Beispiel}
\newtheorem{satz}{Satz}
\newtheorem{lem}{Lemma}
\newtheorem{defn}{Definition}
\newtheorem{beob}{Beobachtung}
\newtheorem{bspe}{Beispiele}
\newtheorem{kor}{Korollar}

\theoremstyle{remark}
\newtheorem*{bem}{Bemerkung}

\newcommand{\TODO}[1]{\textcolor{red}{TODO: #1}} %\red{#1}} % TODO-Markierungen

% Zahlbereiche
\newcommand{\R}{\mathbb{R}} % Reelle Zahlen
\newcommand{\N}{\mathbb{N}} % Natürliche Zahlen
\newcommand{\Z}{\mathbb{Z}} % Ganze Zahlen
%\newcommand{\C}{\mathbb{C}} % Komplexe Zahlen
\newcommand{\Q}{\mathbb{Q}} % Rationale Zahlen

% Schöne Fürall- und Existenzquantoren
\newcommand{\fa}[1]{\forall \, {#1} \,:\,}
%\newcommand{\ex}[1]{\exists \, {#1} \,:\,}

\DeclareMathOperator{\Hom}{Hom} % Homomorphisms
\DeclareMathOperator{\id}{id} % Identität
\DeclareMathOperator{\spann}{spann} % Spann
\DeclareMathOperator{\im}{im} % Image (Bild)
\DeclareMathOperator{\coker}{coker} % Kokern
\newcommand{\pt}{\mathrm{pt}}
\newcommand{\blank}{\text{--}} % Platzhalter
\newcommand{\ntimes}{\!\times\!} % schmaleres (narrower) \times
\newcommand{\angles}[1]{{\langle #1 \rangle}}
\newcommand{\const}[1]{\text{konst } #1} % konstante Funktion mit Wert #1
\newcommand{\LH}{\mathcal{H}} % Local homology
\DeclareMathOperator{\colim}{colim} % Kolimes
\newcommand{\nspace}[1]{\foreach \i in {1,...,#1}{ \! }} % Negativer Abstand
\DeclareMathOperator{\Tor}{Tor} % Tor-Funktor
\newcommand{\SC}{\mathcal{C}} % Serre-Klasse
\newcommand{\FG}{\mathcal{FG}} % Serre-Klasse der endlich erzeugten Gruppen
\newcommand{\F}{\mathcal{F}} % Serre-Klasse der endlichen Gruppen

% Abkürzungen
\newcommand{\ES}{Es sei} % Prof. Hanke schreibt das immer, mag das anscheinend lieber als ein bloßes "Sei"
\newcommand{\ESn}{Es seien} % Prof. Hanke schreibt das immer, mag das anscheinend lieber als ein bloßes "Sei"
\newcommand{\zB}{z.\,B.}
\renewcommand{\dh}{d.\,h.} % das heißt

% Intervalle
\newcommand{\cinterval}[2]{\left[ #1, #2 \right]} % closed interval
\newcommand{\I}{I} % Kompaktes Einheitsinterval [0,1]
%\newcommand{\I}{\cinterval{0}{1}} % Kompaktes Einheitsinterval [0,1] (alternativ)

% Schöne Mengen { #1 | #2 } (benötigt mathtools)
% siehe http://tex.stackexchange.com/questions/13634/define-pretty-sets-in-latex-esp-how-to-do-the-condition-separator
\DeclarePairedDelimiterX\Set[2]{\lbrace}{\rbrace}%
 { #1 \,\delimsize|\, #2 }

% http://tex.stackexchange.com/questions/117732/tikz-and-babel-error
% Es ist schierer Wahnsinn, welche Hacks LaTeX benötigt!
\tikzset{
  every picture/.prefix style={
    execute at begin picture=\shorthandoff{"}
  }
}

% Zentrierte kommutative Diagramme
\newenvironment{centertikzcd}
  {\begin{center}\begin{tikzcd}}
  {\end{tikzcd}\end{center}}

% TikZ-Makros
\newcommand{\zeroDot}[2]{\node[draw,circle,inner sep=0.4pt,fill] at (#1,#2) {};} % Punkt, der in Spektralsequenz für eine Null steht

\begin{document}

\maketitle

\section{Spektralsequenzen}

\subsection{Faserungen}

\begin{defn}
  Eine \emph{Serre-Faserung} ist eine stetige Abbildung $p : E \to B$, welche die \emph{Homotopieliftungseigenschaft} (HLE) für die Scheiben $D^n$ besitzt, \dh{}
  für alle $n \geq 0$ und für alle stetigen Abbildungen $H$, $H_0$ wie unten, sodass das äußere Quadrat kommutiert, gibt es eine stetige Abbildung $\tilde{H}$, sodass die beiden Dreiecke kommutieren:
  \begin{centertikzcd}[row sep=1.2cm, column sep=1.4cm]
    D^n \arrow[r, "H_0"] \arrow[d, hook, "i_0", swap] &
    E \arrow[d, "p"] \\
    D^n \times \I \arrow[r, "H"] \arrow[ur, "\exists\, \tilde{H}", dashed] &
    B
  \end{centertikzcd}
  Dabei ist $i_0$ die Inklusion von $D^n$ in $D^n \times \I$ als $D^n \times \{ 0 \}$. \\
  Eindeutigkeit von $\tilde{H}$ wird nicht gefordert.
\end{defn}

\begin{lem}
  \ES{} $p : E \to B$ eine stetige Abbildung. Dann sind äquivalent:
  \begin{enumerate}[label=\alph*)]
    \item $p$ ist eine Serre-Faserung
    \item $p$ besitzt die \emph{relative Homotopieliftungseigenschaft} für CW-Paare, \dh{} für alle CW-Paare $(X, A)$ und für alle $H_0$ und $H$ wie unten, sodass das äußere Quadrat kommutiert, gibt eine stetige Abbildung $\tilde{H}$, sodass die beiden Dreiecke kommutieren:
    \begin{centertikzcd}[row sep=1.2cm, column sep=1.4cm]
      X \!\times\! \{ 0 \} \cup A \!\times\! \I \arrow[r, "H_0"] \arrow[d, hook, swap] &
      E \arrow[d, "p"] \\
      X \times \I \arrow[r, "H"] \arrow[ur, "\exists\, \tilde{H}", dashed] &
      B
    \end{centertikzcd}
  \end{enumerate}
\end{lem}

\begin{bem}
  Eine \emph{Hurewicz-Faserung} ist eine Serre-Faserung, welche die Homotopieliftungseigenschaft sogar für alle topologischen Räume besitzt.
\end{bem}

% vgl. Diskussion im Hatcher auf Seite 376
\begin{proof}
  "`b) $\implies$ a)"' \enspace Folgt sofort mit $(X, A) \coloneqq (D^n, \emptyset)$. \\[2pt]
  "`a) $\implies$ b)"' \enspace Wir behandeln zunächst den Fall $(X, A) = (D^n, S^{n-1})$, $n \in \N$. Dann ist $(D^n \times \I, D^n \times \{ 0 \} \cup S^{n-1} \cup \I) \approx (D^n)$ homöomorph als Raumpaar.
  Somit ist die relative Homotopieliftungseigenschaft in diesem Fall gleichbedeutend zur Homotopieliftungseigenschaft für die Scheibe $D^n$. \\
  \ES{} nun $(X, A)$ ein beliebiges Raumpaar.
  % TODO: Ausführlicher?
  Dann kann man induktiv die Homotopie $H$ auf die $i$-Zellen $e^i_\alpha$ von $X \setminus A$ fortsetzen.
  Dabei ist die Homotopie auf $S^{n-1} = \partial D^n$ durch die Komposition der bisher konstruierten Homotopie mit der anheftenden Abbildung $\phi_\alpha : S^{n-1} \to X^{n-1}$ vorgegeben.
  Man erhält die Fortsetzung durch Anwenden des zuerst bewiesenen Falls.
\end{proof}

% Lemma 4.5 in http://www.math.washington.edu/~mitchell/Notes/serre.pdf
\begin{lem}
  \ESn{} $p : E \to B$ eine Serre-Faserung, $b_0 \in B$, $F \coloneqq p^{-1}(b_0)$ die Faser über $b_0$ und $f_0 \in F$.
  Dann gibt es eine lange exakte Sequenz
  \[ \ldots \to \pi_n(F, f_0) \xrightarrow{i_*} \pi_n(E, f_0) \xrightarrow{p_*} \pi_n(B, b_0) \xrightarrow{\partial} \pi_{n-1}(F, f_0) \to \ldots \to \pi_1(B, b_0) \]
  von Homotopiegruppen.
  Dabei ist $i : F \hookrightarrow E$ die Inklusion.
\end{lem}

\begin{proof}
  Die gesuchte exakte Sequenz ist die lange exakte Homotopiesequenz
  \[ \ldots \to \pi_n(F, f_0) \xrightarrow{i_*} \pi_n(E, f_0) \to \pi_n(E, F, f_0) \xrightarrow{\partial} \pi_{n-1}(F, f_0) \to \ldots \to \pi_1(E, F, f_0) \]
  des Raumpaares $(E, F)$.
  Es bleibt zu zeigen: $\pi_n(E, F, f_0) \cong \pi_n(B, b_0)$ als Gruppe für $n > 1$ und als punktierte Menge für $n = 1$.
  Der Isomorphismus muss außerdem so gewählt werden, dass
  \[ p_* = \left( \pi_n(E, f_0) \to \pi_n(E, F, f_0) \xrightarrow{\cong} \pi_n(B, b_0) \right). \]
  Wir zeigen: $p_* : \pi_n(E, F, f_0) \to \pi_n(B, b_0)$ ist der gesuchte Isomorphismus (damit ist obige Gleichung erfüllt). \\[2pt]
  \emph{Surjektivität}: Sei $[g : (I^{n+1}, \partial I^{n+1}, b_0) \to (B, \{ b_0 \}, b_0)] \in \pi_{n+1}(B, b_0)$, $n \geq 0$.
  Sei $\tilde{g}$ der Lift im folgenden relativen HLE-Diagramm:
  \begin{centertikzcd}[row sep=1.2cm, column sep=1.4cm]
    U \arrow[r, "\const{f_0}"] \arrow[d, hook, swap] &
    E \arrow[d, "p"] \\
    I^n \times I \arrow[r, "g"] \arrow[ur, "\exists\, \tilde{g}", dashed] \arrow[r, "g"] &
    B
  \end{centertikzcd}
  wobei $U \coloneqq I^n \!\times\! \{ 0 \} \cup (\partial I^n) \!\times\! I \subset I^{n+1}$.
  Dann kann man $\tilde{g}$ als eine Abbildung $(I^{n+1}, \partial I^{n+1}, U) \to (E, F, \{ f_0 \})$ von Raumtripeln auffassen, welche ein Element von $\pi_{n+1}(E, F, f_0)$ repräsentiert.
  Es gilt $p_*[\tilde{g}] = [p \circ \tilde{g}] = [g]$. \\[2pt]
  \emph{Injektivität}: Seien $[h_0], [h_1] \in \pi_{n+1}(E, F, f_0)$ mit $p_*[h_0] = p_*[h_1]$.
  Sei
  \[ H : I \times I^{n+1}, \quad (t, x) \mapsto H_t(x) \]
  eine Homotopie mit $H_0 = p \circ h_0$, $H_1 = p \circ h_1$, welche zu jedem Zeitpunkt $t \in \I$ eine Abbildung $H_t : (I^{n+1}, \partial I^{n+1}) \to (B, \{ b_0 \})$ von Raumpaaren ist.
  Betrachte folgendes HLE-Diagramm:
  \begin{centertikzcd}[row sep=1.2cm, column sep=1.8cm]
    V \arrow[r, "h"] \arrow[d, hook, swap] &
    E \arrow[d, "p"] \\
    I^{n+1} \times I \arrow[r, "H"] \arrow[ur, "\exists\, \tilde{H}", dashed] &
    B
  \end{centertikzcd}
  mit $V \coloneqq I^{n+1} \!\times\! \{ 0 \} \cup (\partial I^{n+1}) \!\times\! I \subset I^{n+2}$ und
  \[
    h|_{\{0\} \times I^{n+1}} \coloneqq h_0, \quad
    h|_{\{1\} \times I^{n+1}} \coloneqq h_1, \quad
    h|_{I \times U} \coloneqq \const{f_0}.
  \]
  Nun ist $\tilde{H}$ eine Homotopie von $h_0$ nach $h_1$, welche zu jedem Zeitpunkt $t$ eine Abbildung $\tilde{H}_t : (I^{n+1}, \partial I^{n+1}, U) \to (E, F, \{ b_0 \})$ von Raumtripeln ist.
\end{proof}

\begin{defn}
  \ESn{} $p : E \to B$ und $g : X \to B$ stetig.
  Der \emph{Pullback} von $p$ entlang $g$ ist die Abbildung $g^*(p) : g^*(E) \to X$, wobei $g^*(E) \coloneqq X \times_B E$ das Faserprodukt von $X$ und $E$ über $B$ vermöge $g$ und $p$ ist.
\end{defn}

\begin{bem}
  Pullback ist funktoriell: $(g \circ f)^* = f^* \circ g^*$ und $\id^* = \id$.
\end{bem}

\begin{lem}
  Pullbacks von Serre-Faserungen sind Serre-Faserungen.
\end{lem}

\begin{proof}
  Sei $p : E \to B$ eine Serre-Faserung und $g : X \to B$ stetig.
  Wir müssen die Existenz des Morphismus $\tilde{H}$ im folgenden Diagramm zeigen:
  \begin{centertikzcd}[row sep=1.2cm, column sep=1.4cm]
    D^n \arrow[r, "H_0"] \arrow[d, hook, "i_0", swap] &
    g^*(E) \arrow[d, "g^*(p)"] \arrow[r, "h"] \arrow[dr, phantom, "\ulcorner", very near start] &
    E \arrow[d, "p"] \\
    D^n \times \I \arrow[r, "H"] \arrow[ur, dashed, "\tilde{H}"] &
    X \arrow[r, "g"] &
    B
  \end{centertikzcd}
  Aus der HLE von $p$ erhält wie folgt einen Morphismus $K$:
  \begin{centertikzcd}[row sep=1.2cm, column sep=1.4cm]
    D^n \arrow[r, "H_0"] \arrow[d, hook, "i_0", swap] &
    X \times_B E \arrow[r, "h"] &
    E \arrow[d, "p"] \\
    D^n \times \I \arrow[r, "H"] \arrow[urr, "K", dashed] &
    X \arrow[r, "g"] &
    B
  \end{centertikzcd}
  Nun ist $D^n \times \I$ vermöge $H$ und $K$ ein Kegel über dem Diagramm $(X \xrightarrow{g} B \xleftarrow{p} E)$.
  Die universelle Eigenschaft von $g^*(E)$ induziert einen Morphismus $\tilde{H} : D^n \times \I \to X \times_B E$ mit
  $g^*(p) \circ \tilde{H} = H$ und $h \circ \tilde{H} = K$.
  Aus der univ. Eigenschaft von $g^*(E)$ (Eindeutigkeit) folgt nun $\tilde{H} \circ i_0 = H_0$.
  % weil: $h \circ H_0 = K \circ i_0 = h \circ \tilde{H} \circ i_0$
\end{proof}

\begin{defn}
  Ein Morphismus $(g, \tilde{g}) : p' \to p$ von Serre-Faserungen $p' : E' \to B'$ und $p : E \to B$ ist ein kommutatives Quadrat der Form
  \begin{centertikzcd}
    E' \arrow[r, "\tilde{g}"] \arrow[d, "p'"] &
    E \arrow[d, "p"] \\
    B' \arrow[r, "g"] &
    B
  \end{centertikzcd}
\end{defn}

\begin{bsp}
  Pullback einer Serre-Faserung $p$ entlang einer stetigen Abbildung $g$ induziert einen Morphismus $(g, \tilde{g}) : g^*(p) \to p$ von Serre-Faserungen.
\end{bsp}

\begin{lem}
  Die langen exakten Sequenzen der Homotopiegruppen von Faserungen sind natürlich: \ES{} $(g, \tilde{g}) : p' \to p$ ein Morphismus von Serre-Faserungen $p' : E' \to B'$ und $p : E \to B$, $b_0' \in B'$, $b_0 \coloneqq g(b_0')$, $F' \coloneqq p'^{-1}(b_0')$, $F \coloneqq p^{-1}(b_0)$, $f'_0 \in F'$, $f_0 \coloneqq \tilde{g}(f'_0)$. Dann gibt es eine "`Leiter"' bestehend aus kommutativen Quadraten zwischen den Homotopiesequenzen:
  \begin{centertikzcd}[row sep=0.8cm, column sep=0.6cm]
    \ldots \arrow[r] &
    \pi_n(F', f'_0) \arrow[r, "i'_*"] \arrow[d, "(\tilde{g}|_{F'})_*"] &
    \pi_n(E', f'_0) \arrow[r, "p'_*"] \arrow[d, "\tilde{g}_*"] &
    \pi_n(B', b'_0) \arrow[r, "\partial"] \arrow[d, "g_*"] &
    \pi_{n-1}(F', f'_0) \arrow[r] \arrow[d, "(\tilde{g}|_{F'})_*"] &
    \ldots \\
    \ldots \arrow[r] &
    \pi_n(F, f_0) \arrow[r, "i_*"] &
    \pi_n(E, f_0) \arrow[r, "p_*"] &
    \pi_n(B, b_0) \arrow[r, "\partial"] &
    \pi_{n-1}(F, f_0) \arrow[r] &
    \ldots
  \end{centertikzcd}
\end{lem}

\begin{proof}
  Folgt aus der Natürlichkeit der langen exakten Homotopiesequenz von Raumpaaren.
\end{proof}

\ES{} $p : E \to B$ eine Serre-Faserung, $\gamma : \I \to B$ ein stetiger Weg.
Betrachte die lange exakte Sequenz
% TODO: Aufpunkte, Zusammenhang, H_0?
\[ \ldots \to \pi_n(F_{\gamma(0)}) \to \pi_n(\gamma^*(E)) \to \pi_n(\I) \to \pi_{n-1}(F_{\gamma(0)}) \to \ldots \]
der Homotopiegruppen von $\gamma^*(p) : \gamma^*(E) \to \I$ mit Faser
\[ F_{\gamma(t)} \coloneqq \gamma^*(p)^{-1}(t)  \subset \gamma^*(E) = \Set{(t, e) \in \I \times E}{\gamma(t) = p(e)}. \]
In dieser Sequenz sind die Gruppen $\pi_n(\I)$ trivial.
Folglich sind die Abbildungen $(i_{\gamma(t)})_* : \pi_n(F_{\gamma(t)}, *) \to \pi_n(\gamma^*(E), *)$ Isomorphismen. % TODO: Anderer Name für die Aufpunkte
In anderen Worten: $i_{\gamma(t)}$ ist eine schwache Äquivalenz.
Aus einem Korollar des Whitehead-Theorems folgt nun, dass $i_t$ auch in Homologie und Kohomologie Isomorphismen induziert (vgl. Spanier, AT, S. 406, Cor 7.6.25). % TODO: Besser referenzieren!
Wir untersuchen den Isomorphismus
\[ T_\gamma \coloneqq (i_{\gamma(1)})^* \circ ((i_{\gamma(0)})^*)^{-1} \enspace:\enspace H^*(F_{\gamma(0)}) \xrightarrow{\cong} H^*(F_{\gamma(1)}). \]

\begin{lem}
  $T_\gamma$ hängt lediglich von der Weghomotopieklasse von $\gamma$ ab, \dh{} ist $\eta$ ein zweiter Weg mit $\gamma \simeq \eta$, so gilt $T_\gamma = T_\eta$.
\end{lem}

\begin{proof}
  Sei $H : \I \times \I \to B$ eine Homotopie zw. den Wegen $\gamma$ und $\eta$, \dh{} $H_0 \coloneqq H(0, \blank) = \gamma$, $H_1 = \eta$, $H(\blank, 0) \equiv x$ und $H(\blank, 1) \equiv y$ mit $x \coloneqq \gamma(0) \!=\! \eta(0)$ und $y \coloneqq \gamma(1) \!=\! \eta(1)$.
  Für festes $s \in \I$ sei $i_s : \I \to \I \times \I, \enspace t \mapsto (s, t)$ die Inklusion als $\{ s \} \times I$.
  Betrachte das kommutative Diagramm
  \begin{centertikzcd}[row sep=1.2cm, column sep=1.4cm]
    H_s^*(E) \arrow[r, hook, "\widetilde{i_s}"] \arrow{d}{}[swap]{H_s^*(p)} \arrow[dr, phantom, "\ulcorner", very near start] &
    H^*(E) \arrow[r] \arrow[d, swap, "H^*(p)"] \arrow[dr, phantom, "\ulcorner", very near start] &
    E \arrow[d, "p"] \\
    \I \arrow[r, hook, "i_s"] \arrow[rr, bend right, "H_s"] &
    \I \!\times\! \I \arrow[r, "H"] &
    B
  \end{centertikzcd}
  Sei $t \in \I$ fest.
  Sei $F_{s,t} \coloneqq \left( H_s^*(p) \right)^{-1}(t) = \left( H^*(p) \right)^{-1}((s, t))$ und $f_0 \in F$. % TODO: Warum gibt es solch ein $f_0$?
  Das linke komm. Diagramm induziert einen Morphismus zw. den langen ex. Homotopieseq. von $H_t^*(p)$ und~$H^*(p)$:
  \begin{centertikzcd}[row sep=0.8cm, column sep=0.8cm]
    \ldots \arrow[r] &
    \pi_{n+1}(I, t) \arrow[r, "\partial"] \arrow[d, "i_{s*}"] &
    \pi_n(F_{s,t}, f_0) \arrow[r, "(i'_{s,t})*"] \arrow[d, equal] &
    \pi_n(H_s^*(E), f_0) \arrow{r}{H_s^*(p)_*}{} \arrow{d}{(\widetilde{i_s})_*}{} &
    \pi_n(\I, t) \arrow[r] \arrow{d}{i_{s*}}{} &
    \ldots \\
    \ldots \arrow[r] &
    \pi_{n+1}(I \!\times\! I, (s, t)) \arrow[r, "\partial"] &
    \pi_n(F_{s,t}, f_0) \arrow[r, "(i_{s,t})_*"] &
    \pi_n(H^*(E), f_0) \arrow[r, "H^*(p)_*"] &
    \pi_n(\I \!\times\! \I, (s, t)) \arrow[r] &
    \ldots
  \end{centertikzcd}
  In diesen Sequenzen verschwinden die Gruppen $\pi_n(I, t)$ bzw. $\pi_n(I \ntimes I, (s, t))$.
  Folglich induzieren die Abbildungen $\widetilde{i_s}$ Isomorphismen in Homotopie und in Kohomologie.
  Es gilt nun
  \begin{align*}
    T_\gamma &
    = (i'_{0,1})^* \circ ((i'_{0,0})^*)^{-1}
    = (i'_{0,1})^* \circ (\widetilde{i_0})^* \circ (\widetilde{i_0})^{-1} \circ ((i'_{0,0})^*)^{-1} \\
    & = (i_{0,1})^* \circ ((i_{0,0})^*)^{-1}
    \stackrel{(\star)}{=} (i_{1,1})^* \circ ((i_{1,0})^*)^{-1} \\
    & = (i'_{1,1})^* \circ (\widetilde{i_1})^* \circ (\widetilde{i_1})^{-1} \circ ((i'_{1,0})^*)^{-1}
    = (i'_{1,1})^* \circ ((i'_{1,0})^*)^{-1}
    = T_\eta.
  \end{align*}
  Die Gleichung $(\star)$ gilt wegen $i_{0,1} \simeq i_{1,1}$ und $i_{0,0} \simeq i_{1,0}$.
\end{proof}

Mit ganz ähnlicher Technik kann man zeigen:

\begin{lem}
  Seien $\gamma, \eta : I \to B$ stetige Wege mit $\gamma(1) = \eta(0)$.
  Dann gilt
  \[ T_\eta \circ T_\gamma = T_{\gamma \centerdot \eta} \enspace:\enspace H^*(F_{\gamma(0)}) \xrightarrow{\cong} H^*(F_{\eta(1)}). \]
  Dabei ist die Komposition $\gamma \centerdot \eta$ von $\gamma$ und $\eta$ folgender Weg:
  \[
    \gamma \centerdot \eta : I \to B, \quad
    s \mapsto \begin{cases}
      \gamma(2s), & \text{falls } s \in [0, \tfrac{1}{2}], \\
      \eta(2s - 1), & \text{falls } s \in [\tfrac{1}{2}, 1].
    \end{cases}
  \]
\end{lem}

\begin{proof}
  Betrachte folgendes kommutatives Diagramm:
  \begin{centertikzcd}[row sep=1.2cm, column sep=1.4cm]
    \gamma^*(E) \arrow[r, hook, "\widetilde{j}"] \arrow{d}{}[swap]{\gamma^*(p)} \arrow[dr, phantom, "\ulcorner", very near start] &
    (\gamma \centerdot \eta)^*(E) \arrow[r] \arrow[d, swap, "(\gamma \centerdot \eta)^*(p)"] \arrow[dr, phantom, "\ulcorner", very near start] &
    E \arrow[d, "p"] \\
    \I \arrow[r, hook, "j"] \arrow[rr, bend right, "\gamma"] &
    \I \arrow[r, "\gamma \centerdot \eta"] &
    B
  \end{centertikzcd}
  Dabei ist $j : I \to I$ die Abbildung $s \mapsto \sfrac{s}{2}$.
  Analog zum letzten Lemma sieht man anhand des Leiterdiagramms der langen exakten Sequenzen der Faserungen $\gamma^*(p)$ und $(\gamma \centerdot \eta)^*(p)$, dass $\widetilde{j}$ einen Isomorphismus in Homotopie und Kohomologie induziert.
  Es gibt ein ähnliches Diagramm mit $\eta$ statt $\gamma$ und $k : I \to I, \enspace s \mapsto \sfrac{(1+s)}{2}$ statt $j$.
  Es induziert auch $\widetilde{k}$ einen Isomorphismus in Kohomologie.
  Es gilt nun
  \begin{align*}
    T_\eta \circ T_\gamma
    & = (i_{\eta(1)})^* \circ ((i_{\eta(0)})^*)^{-1} \circ (i_{\gamma(1)})^* \circ ((i_{\gamma(0)})^*)^{-1} \\
    & = (i_{\eta(1)})^* \circ \tilde{k}^* \circ (\tilde{k}^*)^{-1} \circ ((i_{\eta(0)})^*)^{-1} \circ (i_{\gamma(1)})^* \circ \tilde{j}^* \circ (\tilde{j}^*)^{-1} \circ ((i_{\gamma(0)})^*)^{-1} \\
    & = (\tilde{k} \circ i_{\eta(1)})^* \circ ((\tilde{k} \circ i_{\eta(0)})^*)^{-1} \circ (\tilde{j} \circ i_{\gamma(1)})^* \circ ((\tilde{j} \circ i_{\gamma(0)})^*)^{-1} \\
    & = (i_{\gamma \centerdot \eta(1)})^* \circ ((i_{\gamma \centerdot \eta(\sfrac{1}{2})})^*)^{-1} \circ (i_{\gamma \centerdot \eta(\sfrac{1}{2})})^* \circ ((i_{\gamma \centerdot \eta(0)})^*)^{-1} \\
    & = (i_{\gamma \centerdot \eta(1)})^* \circ ((i_{\gamma \centerdot \eta(0)})^*)^{-1} = T_{\gamma \centerdot \eta}. \qedhere
  \end{align*}
\end{proof}

\subsection{Lokale Koeffizienten}

\iffalse
% hier nach Hatcher, AT, Kapitel 3.H

Sei im Folgenden $B$ wegzusammenhängend.
Wir haben gezeigt, dass dann alle Fasern diesselbe Kohomologiegruppen besitzen.
Wir können also von der Kohomologie der Faser $F$ sprechen, wobei $F \coloneqq p^{-1}(b_0)$ für ein beliebiges $b_0 \in B$.
Die Fundamentalgruppe $\pi_1(B, b_0)$ operiert auf $H^*(F)$ durch $[\gamma] \mapsto T_\gamma$.

% TODO: Welche Topologie trägt $G$? Diskret?
\begin{defn}
  Ein \emph{Bündel von Gruppen} auf einem topologischen Raum $B$ besteht aus einer Gruppe $G$, einer Abbildung $q : A \to B$ und Bijektionen $g_b : q^{-1}(b) \xrightarrow{\cong} G$ für alle $b \in B$, welche den Fasern von $q$ eine zu $G$ isomorphe Gruppenstruktur verleihen, sodass gilt:
  Für jeden Punkt $x \in B$ gibt es eine Umgebung $U \subset B$, sodass
  \[
    g_U : q^{-1}(U) \to U \times G, \quad
    y \mapsto (q(y), g_{q(y)}(y))
  \]
  stetig und sogar ein Homöomorphismus ist.
\end{defn}

\begin{bsp}
  Sei $G$ eine Gruppe.
  Die Projektion $B \times G \to B$ ist in kanonischer Weise ein Bündel von Gruppen auf $B$.
\end{bsp}
\fi

% hier nach http://isites.harvard.edu/fs/docs/icb.topic880873.files/local_systems.pdf

\begin{defn}
  Ein \emph{lokales Koeffizientensystem} $\underline{A}$ auf einem topologischen Raum $B$ besteht aus abelschen Gruppen $(A_b)_{b \in B}$ und Isomorphismen $T_\gamma : A_{\gamma(0)} \xrightarrow{\cong} A_{\gamma(1)}$ für jeden stetigen Weg $\gamma : I \to B$, sodass gilt:
  \begin{itemize}
    \item Sind zwei Wege $\gamma, \eta : I \to B$ homotop modulo Endpunkte, so gilt $T_\gamma = T_\eta$.
    \item Für komponierbare Wege $\gamma, \eta : I \to B$ gilt $T_{\gamma \centerdot \eta} = T_\eta \circ T_\gamma$.
    \item Für den konstanten Weg $\gamma \equiv b$ gilt $T_\gamma = \id_{A_b}$.
  \end{itemize}
\end{defn}

\begin{bem}
  % TODO: sagt man "lokales Koeffizientensystem" oder nur "lokales System"?
  Man kann ein lokales Koeffizientensystem auf $B$ auch als Funktor von dem Fundamentalgruppoid von $B$ in die Kategorie der abelschen Gruppen auffassen.
\end{bem}

\begin{bsp}
  Im letzten Abschnitt wurde gezeigt: Bei einer Serre-Faserung $p : E \to B$ bilden die $q$-ten Kohomologiegruppen $A_b \coloneqq H^q(p^{-1}(b))$ der Fasern ein lokales Koeffizientensystem.
  Mit dem universellen Koeffizententheorem sieht man, dass gleiches auch für die Homologiegruppen $A_b \coloneqq H^q(p^{-1}(b); G)$ mit Koeffizienten in einer abelschen Gruppe $G$ gilt.
  Wir bezeichnen dieses Koeffizientensystem im Folgenden mit $\LH^q(F_p; G)$. % XXX: Das ist ne doofe Notation. Ändern?
\end{bsp}

\begin{bsp}
  Für jede abelsche Gruppe $G$ gibt es das konstante Koeffizientensystem $\underline{G}$ mit $G_b \coloneqq G$ für alle $b \in B$ und $T_\gamma = \id_G$ für alle $\gamma : I \to B$.
\end{bsp}

Sei im Folgenden $\Delta_n(B)$ die Menge der $n$-Simplizes in $B$, also die Menge der stetigen Abbildungen $\Delta^n \to B$ mit $\Delta^n \coloneqq \spann \{ e_0, \ldots, e_n \} \subset \R^{n+1}$, und
\[
  d_n : \Delta_n(B) \to \Delta_{n-1}(B) \quad
  \sigma \mapsto \sigma_{\angles{e_0, \ldots, \hat{e_i}, \ldots, e_n}} \qquad (0 \leq i \leq n),
\]
die Abbildung auf die $i$-Seite. % TODO: kann man das "Projektion" nennen?
Für einen $n$-Simplex $\sigma$ bezeichne $\sigma_i \coloneqq \sigma_{\angles{e_i}} \in \Delta_0(B) = B$ die $i$-te Ecke und $\sigma_{ij} \coloneqq \sigma_{\angles{e_i, e_j}} \in \Delta_1(B)$ den Weg von von $\sigma_i$ nach $\sigma_j$ entlang der $ij$-Kante von $\sigma$ ($0 \leq i \leq j \leq n$).

\begin{defn}
  Sei $B$ ein topologischer Raum, $\underline{A}$ ein lokales Koeffizientensystem auf $B$.
  Der \emph{Kokomplex der singulären Koketten auf $B$ mit Koeffizienten in $\underline{A}$} ist folgendermaßen definiert:
  \[
    C^n(B; \underline{A}) \coloneqq \prod_{\mathclap{\sigma \in \Delta_n(B)}} A_{\sigma_0}, \quad
    \delta^n\left((a_\tau)_{\tau \in \Delta_n(B)}\right)_{\sigma \in \Delta_{n+1}(B)} \coloneqq T_{\sigma_{01}}^{-1}(a_{d_0(\sigma)}) + \sum_{i=1}^{n+1} (-1)^i a_{d_i(\sigma)}.
  \]
  Man überprüft leicht, dass $\delta^{n+1} \circ \delta^n = 0$ gilt.
  Die Kohomologie $H^*(B; \underline{A}) \coloneqq H^*(C^*(B; \underline{A}))$ dieses Kettenkomplexes heißt \emph{singuläre Kohomologie von $B$ mit Koeffizienten in $\underline{A}$}.
\end{defn}

\begin{beob}
  Für das konstante Koeffizientensystem $\underline{G}$ gilt $H^*(B; \underline{G}) \cong H^*(B; G)$.
  Gewöhnliche Kohomologie mit Koeffizienten ist also ein Spezialfall von Kohomologie mit Koeffizienten in einem lokalen System.
\end{beob}

\begin{defn}
  Es sei $\underline{R}$ ein lokales Koeffizientensystem, in dem die Gruppen $R_b$ sogar Ringe und die Abbildungen $T_\gamma$ Ringisomorphismen sind. Dann definiert
  \begin{align*}
    & \cup : H^m(B; \underline{R}) \times H^n(B; \underline{R}) \to H^{m+n}(B; \underline{R}), \\
    & ([(a_\sigma)_{\sigma \in \Delta_m(B)}] \cup [(b_\sigma)_{\sigma \in \Delta_n(B)}])_{\tau \in \Delta_{m+n}(B)} \coloneqq a_{\sigma_{\angles{e_1, \ldots, e_m}}} \cdot T_{\sigma_{0m}}^{-1}(b_{\sigma_{e_m, \ldots, e_{m+n}}})
  \end{align*}
  ein Produkt, das sogenannte \emph{Cup-Produkt}.
\end{defn}

\subsection{Spektralsequenzen}

% TODO: Einleitendes Gefasel zu Spektralsequenzen

\ES{} $A$ im Folgenden ein kommutativer Ring mit Eins.

\begin{defn}
  Eine (kohomologische) \emph{Spektralsequenz} besteht aus
  \begin{itemize}
    \item $A$-Moduln $E_r^{p,q}$ für alle $p, q \in \Z$ und $r \geq 1$,
    \item $A$-Modul-Homomorphismen $d_r^{p,q} : E_r^{p,q} \to E_r^{p+r,q-r+1}$ mit $d_r^{p+r,q-r+1} \circ d_r^{p,q} = 0$
    \item und Isomorphismen $\alpha_r^{p,q} : H^{p,q}(E_r) \!\coloneqq\! \ker(d_r^{p,q}) / \im(d_r^{p-r,q+r-1}) \xrightarrow{\cong} E_{r+1}^{p,q}$.
  \end{itemize}
\end{defn}


\begin{bem}
  \begin{itemize}
    \item Die Homomorphismen $d^r_{p,q}$ heißen \emph{Differentiale}.
    \item Die Gesamtheit der Module $E^r_{p,q}$ und Differentiale $d_r^{pq}$ mit $r \in \N$ fest heißt $r$-te \emph{Seite} $E^r$.
    \item Man stellt Seiten für gewöhnlich in einem 2-dimensionalen Raster dar:
  \end{itemize}
  \begin{center}
    \begin{tikzpicture}[x=12pt,y=12pt]
      \begin{scope}[shift={(0,0)}]
        \foreach \x in {-1,...,4}{
          \foreach \y in {-2,-1,...,3}{
            \zeroDot{\x}{\y}
          }
        }
        \foreach \x in {-1,...,5}{
          \foreach \y in {-2,-1,...,3}{
            \draw[->,gray] (\x-0.8,\y) -- (\x-0.2,\y);
          }
        }
        \draw[->] (-0.35,-0.35) -- (5.35,-0.35) node[below] {p};
        \draw[->] (-0.35,-0.35) -- (-0.35,4) node[left] {q};
        \node at (5,4.5) {$E_1$};
      \end{scope}
      \begin{scope}[shift={(10,0)}]
        \foreach \x in {-1,...,4}{
          \foreach \y in {-2,-1,...,3}{
            \zeroDot{\x}{\y}
          }
        }
        \foreach \x in {0,...,5}{
          \foreach \y in {-2,-1,...,3}{
            \draw[->,gray] (\x-1.8,\y+0.9) -- (\x-0.2,\y+0.1);
          }
        }
        \draw[->] (-0.35,-0.35) -- (5.35,-0.35) node[below] {p};
        \draw[->] (-0.35,-0.35) -- (-0.35,4) node[left] {q};
        \node at (5,4.5) {$E_2$};
      \end{scope}
      \begin{scope}[shift={(20,0)}]
        \foreach \x in {-1,...,4}{
          \foreach \y in {-2,-1,...,3}{
            \node[draw,circle,inner sep=0.5pt,fill] at (\x,\y) {};
          }
        }
        \foreach \x in {1,...,5}{
          \foreach \y in {-2,-1,...,2}{
            \draw[->,gray] (\x-2.8,\y+1.9) -- (\x-0.2,\y+0.1);
          }
        }
        \draw[->] (-0.35,-0.35) -- (5.35,-0.35) node[below] {p};
        \draw[->] (-0.35,-0.35) -- (-0.35,4) node[left] {q};
        \node at (5,4.5) {$E_3$};
      \end{scope}
    \end{tikzpicture}
  \end{center}
\end{bem}

\begin{defn}
  Eine Spektralsequenz \emph{konvergiert}, falls für alle $p, q \in \Z$ ein $R \in \N$ existiert, sodass für alle $r \geq R$ die Differentiale von und nach $E_r^{p,q}$ null sind und damit $E^\infty_{p,q} \coloneqq E_R^{p,q} \cong E_{R+1}^{p,q} \cong \ldots$ \\[2pt]
  Der Grenzwert der SS ist die Unendlich-Seite $E_\infty \coloneqq \{ E_\infty^{p,q} \}_{p,q}$.
\end{defn}

\begin{bem}
  Viele Spektralsequenzen sind im ersten Quadranten konzentriert, \dh{} $E_r^{p,q}$ ist nur für $p, q \geq 0$ ungleich Null.
  Solche Spektralsequenzen konvergieren immer, denn für alle $p, q \in \Z$ führen für $r \geq \max(p+1, q+2)$ alle Differentiale von $E_r^{p,q}$ aus dem ersten Quadranten heraus und alle dort eintreffenden Differentiale kommen von außerhalb des ersten Quadranten und sind daher Null.
\end{bem}

\begin{defn}
  Eine \emph{Filtrierung} eines $A$-Moduls $M$ ist eine absteigende Folge
  \[ M \supseteq \ldots \supseteq F^{p-1} M \supseteq F^p M \supseteq F^{p+1} M \supseteq \ldots \]
  von Untermoduln von $M$, $p \!\in\! \Z$.
  Eine Filtrierung heißt
  \begin{itemize}
    \item \emph{ausschöpfend}, falls $M = \cup_p F^p$,
    \item \emph{Hausdorffsch}, wenn $0 = \cap_p F^p M$ und
    \item \emph{regulär}, wenn sie ausschöpfend und Hausdorffsch ist.
  \end{itemize}
\end{defn}

\begin{defn}
  Eine Spektralsequenz $E$ \emph{konvergiert gegen} einen graduierten $A$-Modul $M = \oplus_{n \in \Z} M^n$ (notiert $E_r^{p,q} \Rightarrow M^{p+q}$), falls $E$ überhaupt konvergiert und reguläre Filtrierungen
  \[ M^n \supseteq \ldots \supseteq F^{p-1} M^n \supseteq F^p M^n \supseteq F^{p+1} M^n \supseteq \ldots \]
  existieren, sodass $E_\infty^{pq} \cong F^p M^{p+q} / F^{p+1} M^{p+q}$ für alle $p, q \in \Z$.
\end{defn}

\subsection{Die Spektralsequenz eines filtrierten Komplexes}

% Hier nach http://people.mpim-bonn.mpg.de/matschke/SpectralSequences.pdf

\begin{defn}
  Eine \emph{Filtrierung eines Kokettenkomplexes} $C^\bullet$ ist eine absteigende Folge
  \[ C^\bullet \supseteq \ldots \supseteq F^{p-1} C^\bullet \supseteq F^p C^\bullet \supseteq F^{p+1} C^\bullet \supseteq \ldots \]
 von Unterkomplexen.
\end{defn}

% TODO: Morphismus von filtrierten Komplexen?

\begin{lem}
  \ES{} $C^\bullet$ ein filtrierter Kokettenkomplex.
  Es gibt eine Spektralsequenz mit
  \[ E_1^{pq} = H^{p+q}(F^p C^\bullet / F^{p+1} C^\bullet). \]
  Angenommen, die Filtrierung ist
  \begin{enumerate}[label=\alph*)]
    \item \emph{gradweise nach unten beschränkt}, \dh{} für alle $q \in \Z$ gibt es ein $p \in \Z$ mit $F^p C^q = 0$,
    \item \emph{ausschöpfend}, \dh{} für alle $q \in \Z$ ist $\cup_p F^p C^q = C^q$ und %(aus der ersten Bedingung folgt schon $\cap_p F^p C^q = 0$).
    \item für alle $q \in \Z$ gibt es ein $P \in \Z$, sodass für alle $p \leq P$ gilt: Die Inklusion $F^p C^\bullet \hookrightarrow C^\bullet$ induziert einen Isomorphismus $H^q (F^p C^\bullet) \cong H^q(C^\bullet)$ in Kohomologie.
  \end{enumerate}
  Dann konvergiert die Spektralsequenz gegen $H^*(C^\bullet)$.
\end{lem}

Wir führen zunächst etwas neue Notation ein.
Diese hilft, den Beweis verständlicher zu formulieren.
Wir fassen im Folgenden den Kettenkomplex als ein einziges Modul $C \coloneqq \oplus_{n \in \Z} C^n$ anstatt als Folge von Modulen auf.
Dieses Modul ist filtriert durch die Untermodule $F^p \coloneqq \oplus_{n \in \Z} F^p C^n$.
Wir setzen $F^{-\infty} \coloneqq C$ und $F^{\infty} \coloneqq 0$.
Die Korandabbildung fassen wir als Homomorphismen $d : C \to C$ mit $d \circ d = 0$ auf, der die Filtrierung von $C$ respektiert.

Wir sind interessiert an der Kohomologie von $C^\bullet$, also an $H^*(C) \coloneqq \ker(d) / \im(d)$ und an der Kohomologie von $F^p / F^{p+1}$, also $H^*(F^p / F^{p+1}) \cong (d|_{F^p})^{-1}(F^{p+1}) / d(F^p)$.
Wir geben nun eine Verallgemeinerung der Definition der Kohomologie von $C^\bullet$ und der Kohomologie des Quotientenkomplexes $F^p / F^q$: Statt Zykeln (\dh{} Elementen $c \in C$ mit $d(c) = 0$) betrachten wir \emph{$z$-Zykel}, das sind Elemente $c \in C$ mit $d(c) \in F^z$. Wir teilen diese durch die Menge $d(F^b)$ der \emph{$b$-Ränder} anstatt durch die Menge $d(C)$ der Ränder. Wir setzen

\[ S[z, q, p, b] \coloneqq \frac{F^p \cap d^{-1}(F^z)}{(F^p \cap d^{-1}(F^z)) \cap (F^q + d(F ^b))}. \]

% TODO: Bemerkung zur grafischen Notation von Matschke

Wir haben als Spezialfälle
\[
  S[p,q,p,q] \cong F^p / F^q
  \quad \text{und} \quad
  S[q,q,p,p] \cong H^*(F^p / F^q).
\]

\begin{lem}
  \ES{} $z_1 \geq q_1 \geq p_1 = z_2 \geq b_1 = q_2 \geq p_2 \geq b_2$.
  Dann ist folgende Abbildung ein wohldefinierter Homomorphismus:
  \[
    d^* : S[z_2, q_2, p_2, b_2] \to S[z_1, q_1, p_1, b_1], \quad
    [c] \mapsto [d(c)].
  \]
\end{lem}

\begin{proof}
  Falls $[c] = 0$ in $S[z_2, q_2, p_2, b_2]$, so existieren $x \in F^{q_2}$ und $y \in F^{b_2}$ mit $c = x + d(y)$. Somit gilt $d^*[c] = [dc] = [d(x) + d^2(y)] = [d(x)] = 0$ in $S[z_1, q_1, p_1, b_1]$, da $F^{b_1} = F^{q_2}$.
\end{proof}

\begin{lem}
  Es seien Filtrierungsindizes wie folgt gegeben:
  \begin{centertikzcd}[row sep=0.2cm, column sep=0.5cm]
    &&&& z_3 \arrow[r, "\geq" description, phantom] \arrow[d, equal] &
    q_3 \arrow[r, "\geq" description, phantom] \arrow[d, equal] &
    p_3 \arrow[r, "\geq" description, phantom] &
    b_3 \\

    && z_2 \arrow[r, "\geq" description, phantom] \arrow[d, equal] &
    q_2 \arrow[r, "\geq" description, phantom] \arrow[d, equal] &
    p_2 \arrow[r, "\geq" description, phantom] &
    b_2 \\

    z_1 \arrow[r, "\geq" description, phantom] &
    q_1 \arrow[r, "\geq" description, phantom] &
    p_1 \arrow[r, "\geq" description, phantom] &
    b_1
  \end{centertikzcd}
  Dann ist
  \[
    \alpha : S[q_1, q_2, p_2, p_3] \to
    \frac{\ker(d^* : S[z_2, q_2, p_2, b_2] \to S[z_1, q_1, p_1, b_1])}{\im(d^* : S[z_3, q_3, p_3, b_3] \to S[z_2, q_2, p_2, b_2])}, \quad
    [c] \mapsto [c]
  \]
  ein wohldefinierter Isomorphismus.
\end{lem}

\begin{proof}
  Sei $A$ der Quotient auf der rechten Seite. \\[2pt]
  \emph{Wohldefiniertheit}: Sei $[c] = 0$ in $S[q_1, q_2, p_2, p_3]$, \dh{} es gibt $e \in F^{q_2} = F^{b_1}$ und $f \in F^{p_1}$ mit $c = e + d(f)$. Dann ist $d^*[c] = [d(c)] = [d(e)] = 0$ in $S[z_1, q_1, p_1, b_1]$, also $c \in \ker(d^* : S[z_2, q_2, p_2, b_2] \to S[z_1, q_1, p_1, b_1])$.
  Nun ist $f \in d^{-1}(F^{z_3})$, da $d(f) = c - e \in F^{p_2} = F^{z_3}$.
  Es gilt $[c] = [e + d(f)] = [d(f)] = d^*[f] = 0$ in $A$. \\[2pt]
  \emph{Injektivität}: Sei $c \in F^{p_2} \cap d^{-1}(F^{q_1})$ mit $[c] = 0$ in $A$.
  Das heißt, es gibt $e \in F^{q_2}$, $f \in F^{b_2}$ und $g \in F^{p_3} \cap d^{-1}(F^{z_3})$ mit $c = e + d(f) + d(g)$.
  Dann ist $[c] = [e + d(f+g)] = 0$ in $S[q_1, q_2, p_2, p_3]$, da $f+g \in F^{p_3}$. \\[2pt]
  \emph{Surjektivität}: Sei $\tilde{c} \in F^{p_2} \cap d^{-1}(F^{z_2})$ mit $[\tilde{c}] \in \ker(d^* : S[z_2, q_2, p_2, b_2] \to S[z_1, q_1, p_1, b_1])$.
  Das heißt, es gibt $e \in F^{q_1}$ und $f \in F^{b_1} = F^{q_2}$ mit $d(\tilde{c}) = e + d(f)$.
  Dann ist $[\tilde{c}] = [\tilde{c} - f]$ in $S[q_1, q_2, p_2, p_3]$ mit $\tilde{c} - f \in F^{p_2} \cap d^{-1}(F^{q_1})$, da $d(\tilde{c} - f) = e \in F^{q_1}$.
\end{proof}

\begin{proof}[Beweis des Lemmas über Existenz der Spektralsequenz]
  Wir beachten jetzt wieder, dass $C$ und damit $S[z, q, p, b]$ graduiert und $d$ ein Differential vom Grad $+1$ ist.
  Es sei $S[z, q, p, b]^n$ die $n$-te Komponente.
  Setze
  \[ E_r^{pq} \coloneqq S[p\!+\!r, p\!+\!1, p, p\!-\!r\!+\!1]^{p+q}. \]
  Die Differentiale sind
  \[
    d_r^{pq}
    \enspace : \enspace
    \underbrace{S[p\!+\!r, p\!+\!1, p, p\!-\!r\!+\!1]^{p+q}}_{= E_r^{p,q}} \to \underbrace{S[p\!+\!2r, p\!+\!r\!+\!1, p\!+\!r, p\!+\!1]^{p+q+1}}_{= E_r^{p+r,q-r+1}}, \quad
    [c] \mapsto [d(c)].
  \]
  Sie sind wohldefiniert nach Lemma \TODO{Nr}.
  Wegen Lemma \TODO{Nr} ist
  \[
    \alpha_r^{pq} : H^{p,q}(E_r) = \ker(d^{pq}_r) / \im(d^{p-r,q+r-1}_r) \to E_{r+1}^{pq}, \quad
    [c] \mapsto [c]
  \]
  ein wohldefinierter Isomorphismus.

  \emph{Beweis der Konvergenz}:
  Es seien $p, q \in \Z$.
  Wegen Bedingung a) gibt es ein $R_1 \geq 0$, sodass $F^{p+R_1} C^{p+q+1} = 0$.
  Für $r \geq R_1$ ist damit $E^{p+r,q-r+1}_r$ als Subquotient (\dh{} Quotient eines Untermoduls) von $F^{p+R_1} C^{p+q+1}$ Null.
  Folglich verschwindet auch das Differential $d_r^{pq}$.
  Wegen Bedingung c) gibt es ein $S \in \Z$, sodass $F^s C^\bullet \hookrightarrow C^\bullet$ und somit auch $F^s C^\bullet \hookrightarrow F^{s-1} C^\bullet$ für $s \leq S$ einen Isomorphismus in $H^{p+q-1}$ und $H^{p+q}$ induziert.
  Anhand der langen exakten Sequenz zu $0 \to F^s C^\bullet \to F^{s-1} C^\bullet \to F^{s-1} C^\bullet / F^s C^\bullet \to 0$ sieht man, dass $H^{p+q-1}(F^{s-1} C^\bullet / F^s C^\bullet) = 0$.
  Somit ist $E_r^{p-r,q+r-1}$ für $r \geq R_2 \coloneqq p - s + 1$ als Submodul von $H^{p+q-1}(F^{p-r} C^\bullet / F^{p-r+1} C^\bullet)$ Null.
  Folglich verschwindet auch $d_r^{p-r,q+r-1}$.
  Mit $R \coloneqq \max(R_1, R_2)$ gilt dann $E^{pq}_R \cong E^{pq}_{R+1} \cong \ldots \cong E^{pq}_\infty$.

  Sei $H^n(C^\bullet)$ absteigend filtriert durch $F^p H^n(C^\bullet) \coloneqq \im(i^* : H^n(F^p C^\bullet) \to H^n(C^\bullet))$.
  Für $r \geq R$ ist
  \[ E^{pq}_\infty \cong E^{pq}_r = \frac{F^p C^{p+q} \cap d^{-1}(0)}{(F^p C^{p+q} \cap d^{-1}(0)) \cap (F^{p+1} C^{p+q} + d(F^{p-r+1} C^{p+q-1}))} = S[\infty,p+1,p,p-r+1]^{p+q}. \]
  Es ist daher $F^p H^{p+q}(C^\bullet) / F^{p+1} H^{p+q}(C^\bullet) \cong S[\infty, p+1, p, -\infty]^{p+q}$ ein Quotient von $E^{pq}_\infty$.
  Tatsächlich gilt $S[\infty, p+1, p, -\infty]^{p+q} \cong E^{pq}_\infty$, denn:
  Sei $c \in F^p C^{p+q} \cap d^{-1}(0)$ mit $[c] = 0$ in $S[\infty, p+1, p, -\infty]^{p+q}$.
  Dann gibt es ein $e \in F^{p+1} C^{p+q}$ und ein $f \in C^{p+q-1}$ mit $c = e + d(f)$.
  Wegen Bedingung b) gibt es ein $\tilde{p} \in \Z$ mit $f \in F^{\tilde{p}} C^{p+q+1}$.
  Wähle $r$ so, dass $r \geq R$ und $p-r+1 \leq \tilde{p}$.
  Dann ist $[c] = [e]+[d(f)] = 0$ in $E^{pq}_r \cong E^{pq}_\infty$.
\end{proof}


\subsection{Die Serre-Spektralsequenz}

\begin{satz}[Jean-Pierre Serre]
  Es sei $G$ eine abelsche Gruppe.
  Für jede Serre-Faserung $p : E \to B$ existiert eine Spektralsequenz mit
  \[ E_2^{p,q} = H^p(B; \LH^q(F_p; G)), \]
  welche gegen $H^*(E)$ konvergiert.
\end{satz}

\begin{proof}
  \TODO{}
\end{proof}

\subsection{Multiplikative Struktur der Serre-Spektralsequenz}

\begin{satz}
  Es sei $R$ ein Ring, $p : E \to B$ eine Serre-Faserung.
  Dann gibt es bilineare Abbildungen
  \[ m_r : E^{p,q}_r \times E^{s,t}_r \to E^{p+s,q+t}_r, \enspace (x, y) \mapsto m_r(x, y) =: xy \]
  mit folgenden Eigenschaften:
  \begin{enumerate}[label=(\roman*)]
    \item $d_r$ ist derivativ: $d_r^{p+s,q+t}(xy) = (d_r^{p,q} x) y + (-1)^{p+q} x (d_r^{s,t} y)$
    \item Es gilt $m_{r+1}([x], [y]) = [m_r(x, y)]$ für alle $x \in \ker(d_r^{p,q})$, $y \in \ker(d_r^{s,t})$.
    \item $m_2 : E_2^{p,q} \!\times\! E_2^{r,s} \!\to\! E_2^{p+s,q+t}$ ist das $(-1)^{qs}$-fache des Cup-Produkts
    \[
      H^p(B; \LH^q(F; R)) \times H^s(B; \LH^t(F; R)) \to H^{p+s}(B; \LH^{q+t}(F; R)),
    \]
    welches für $a = [(a_\sigma)_{\sigma \in \Delta_m(B)}] \in H^p(B; \LH^q(F; R))$ und $b = [(b_\sigma)_{\sigma \in \Delta_n(B)}] \in H^s(B; \LH^t(F; R))$ definiert ist durch
    \[
      (a \cup b)_{\sigma \in \Delta_{p+s}(B)} \coloneqq
      a_{\sigma_{\angles{e_1, \ldots, e_m}}} \cup T_{\sigma_{0m}}^{-1}(b_{\sigma_{\angles{e_m, \ldots, e_{m+n}}}}).
    \]
    \item Das Cup-Produkt auf $H^*(B; R)$ respektiert die Filtrierungen von $H^n(B; R)$ und schränkt daher ein zu Abbildungen $F_p^m \times F_s^n \to F_{p+s}^{m+n}$.
    Die induzierte Abbildung auf dem Quotienten $F^m_p/F^m_{p+1} \times F^n_s/F^n_{s+1} \to F^{m+n}_{p+s} / F^{m+n}_{p+s+1}$ entspricht dem Grenzwert $m_\infty : E_\infty^{p,m-p} \times E_\infty^{s,n-s} \to E_\infty^{p+s,m+n-p-s}$ der Multiplikationen $m_r$.
  \end{enumerate}
\end{satz}

\begin{proof}
  \TODO{}
\end{proof}

% Spanier, Algebraic Topology: SS zum filtrierten Komplex; nur triviale Wirkung; etwas genauer als Hatcher
% Hatcher: nur triviale Wirkung
% Switzer, Homology and Homotopy: nur triviale Wirkung, p. 350
% May/Ponto, More Concise AT: Allgemeine Serre-Spektralsequenz (mit lokalen Koeff'en); nur ein "sketch proof"
% McCleary, User's Guide to SS: Lokale Koeffizienten, statement: p. 139, proof: p. 167ff

\subsection{Die Serre-Spektralsequenz für Homologie}

Es gibt eine Version des Satzes für Serre für Homologie anstatt Kohomologie.
In Homologie wird eine andere Indizierung für Spektralsequenzen verwendet:

\begin{defn}
  Eine homologische \emph{Spektralsequenz} besteht aus
  \begin{itemize}
    \item $A$-Moduln $E^r_{p,q}$ für alle $p, q \in \Z$ und $r \geq 1$,
    \item $A$-Modul-Homomorphismen $d^r_{p,q} : E^r_{p,q} \to E^r_{p-r,q+r-1}$ mit $d^r_{p-r,q+r-1} \circ d^r_{p,q} = 0$
    \item und Isomorphismen $\alpha^r_{p,q} : H_{p,q}(E^r) \!\coloneqq\! \ker(d ^r_{p,q}) / \im(d^r_{p+r,q-r+1}) \xrightarrow{\cong} E^{r+1}_{p,q}$.
  \end{itemize}
\end{defn}

Jede homologische Spektralsequenz $E$ liefert eine kohomologische Spektralsequenz, wenn man $E_r^{p,q} \coloneqq E^r_{-p,-q}$ setzt.

\begin{defn}
  Eine homologische Spektralsequenz $E$ \emph{konvergiert gegen} einen graduierten $A$-Modul $M = \oplus_{n \in \Z} M_n$ (notiert $E^r_{p,q} \Rightarrow M_{p+q}$), falls $E$ überhaupt konvergiert und reguläre Filtrierungen
  \[ 0 \subseteq \ldots \subseteq F^{p-1} M_n \subseteq F^p M_n \subseteq F^{p+1} M_n \subseteq \ldots \]
  existieren, sodass $E^\infty_{pq} \cong F^p M_{p+q} / F^{p-1} M_{p+q}$ für alle $p, q \in \Z$.
\end{defn}

\begin{satz}[Jean-Pierre Serre]
  Es sei $G$ eine abelsche Gruppe.
  Für jede Serre-Faserung $p : E \to B$ existiert eine (homologische) Spektralsequenz mit
  \[ E^2_{p,q} = H_p(B; \LH_q(F_p; G)), \]
  welche gegen $H_*(E)$ konvergiert.
\end{satz}

% TODO: Verweis auf Beweis in irgendeinem Buch.

Dabei bilden die $q$-ten Homologiegruppen der Fasern ein lokales Koeffizientensystem mit $\LH_q(F_p; G)_b = H_q(p^{-1}(b))$.
Homologie mit einem Koeffizientensystem $\underline{A}$ ist ähnlich definiert wie Homologie.
Falls $B$ einfach zusammenhängend ist oder allgemeiner $\pi_1(B)$ trivial auf den Homologiegruppen der Faser wirkt, so gilt $H_p(B; \LH_q(F_p; G)) \cong H_p(B; H_q(F_p; G))$.

\subsection{Die Pfadfaserung}

% TODO: wohin?
\begin{defn}
  Der \emph{Pfadraum} eines punktierten topologischer Raum $(X, x_0)$ ist
  \[ PX \coloneqq \Set{\gamma \in X^\I}{\gamma(0) = x_0} \subset X^I \]
  mit der Unterraumtopologie des Raumes $X^I$, welcher die Kompakt-Offen-Topologie besitzt.
  Der Basispunkt von $PX$ ist der konstante Weg $y_0 : \I \to X, \enspace t \mapsto x_0$.
\end{defn}

\begin{bem}
  Der Raum $PX$ ist zusammenziehbar: Die Abbildung
  \[
    H : \I \times PX \to PX, \quad
    (t, \gamma) \mapsto \gamma(t \cdot \blank)
  \]
  ist eine Homotopie zwischen der konstanten Abbildung mit Wert $\gamma_0$ und $\id_{PX}$.
\end{bem}

\begin{lem}
  Die Abbildung
  $
    p : PX \to X, \enspace
    \gamma \mapsto \gamma(1)
  $
  ist eine Hurewicz-Faserung.
\end{lem}

\begin{proof}
  Es sei ein topologischer Raum $A$ und stetige Abbildungen $H_0 : A \to PX$, $H : \I \times A \to X$ mit $H \circ i_0 = p \circ H_0$ gegeben.
  %Definiere $\tilde{H}$ durch
  Dann ist eine Homotopieliftung gegeben durch
  \begin{align*}
    \tilde{H} : \I \times A & \to PX, \\
    (s, a) & \mapsto \gamma_{s, a}, \quad
    \gamma_{s, a}(t) \coloneqq \begin{cases}
      H_0(a)(t \cdot (1+s)) & \text{falls $t \cdot (1+s) \leq 1$,} \\
      H(t \cdot (1+s) - 1, a) & \text{falls $t \cdot (1+s) \geq 1$.}
      \qedhere
    \end{cases}
  \end{align*}
\end{proof}

Die Faserung $p : PX \to X$ wird \emph{Pfadfaserung} genannt.

\begin{bem}
  Die Faser von $p$ über $x_0$ ist
  \[ \Omega X \coloneqq \Set{\gamma \in X^\I}{\gamma(0) = \gamma(1) = x_0}. \]
  Der Raum $\Omega X$ heißt \emph{Schleifenraum} von $X$.
\end{bem}

% XXX: $\Sigma X$ definieren?

\begin{bem}
  Seien $(X, x_0)$ und $(Y, y_0)$ punktierte Räume.
  Es gibt eine in $X$ und $Y$ natürliche Bijektion
  % XXX: Einschränkung auf Unterkategorie der topologischen Räume nötig?
  \begin{alignat*}{4}
    \Hom((\Sigma X, x_0), (Y, y_0)) & \enspace\cong\enspace && \Hom((X, x_0), (\Omega Y, \gamma_0)), \\
    f & \enspace\mapsto\enspace && (x \mapsto t \mapsto f([(x, t)])), \\
    ([(x, t)] \mapsto g(x)(t)) & \enspace\mapsfrom\enspace && g.
  \end{alignat*}
\end{bem}

\begin{lem}\label{convert-to-fibration}
  Man kann jede stetige Abbildung $f : X \to Y$ schreiben als Komposition
  \[ X \xrightarrow{i} E_f \xrightarrow{p} Y \]
  einer Homotopieäquivalenz $i$ und einer Hurewicz-Faserung $p$.
  Genauer gilt
  \begin{align*}
    E_f & \coloneqq \Set{(x, \gamma) \in X \times X^\I}{f(x) = \gamma(0)} \subset X \times Y^\I, \\
    i(x) & \coloneqq (x, t \mapsto f(x)), \\
    p(x, \gamma) & \coloneqq \gamma(1).
  \end{align*}
\end{lem}

% Notation wie in https://en.wikipedia.org/wiki/Homotopy_fiber
\begin{proof}
  Offensichtlich sind $i$ und $p$ stetig und es gilt $p \circ i = f$.
  Das Homotopie-Inverse von $i$ ist $j : E_f \to X, \enspace (x, \gamma) \mapsto x$.
  Es gilt $j \circ i = \id_X$ und eine Homotopie zwischen $i \circ j$ und $id_{E_f}$ ist gegeben durch
  \[
    H : \I \times E_f \to E_f, \quad
    (s, (x, \gamma)) \mapsto (x, \gamma(s \cdot \blank)).
  \]
  Es bleibt zu zeigen, dass $p$ eine Faserung ist.
  Es sei dazu ein topologischer Raum $A$ und Abbildungen $H_0 : A \to E_f$ und $H : \I \times A \to Y$ mit $H \circ i_0 = p \circ H_0$ gegeben.
  Dann ist folgende Abbildung eine Homotopieliftung:
  \begin{align*}
    \tilde{H} : \I \times A & \to E_f, \\
    (s, a) & \mapsto \gamma_{s, a}, \quad
    \gamma_{s, a}(t) \coloneqq \begin{cases}
      H_0(a)(t \cdot (1+s)) & \text{falls $t \cdot (1+s) \leq 1$,} \\
      H(t \cdot (1+s) - 1, a) & \text{falls $t \cdot (1+s) \geq 1$.}
      \qedhere
    \end{cases}
  \end{align*}
\end{proof}
% TODO: Die Homotopieliftung ist zweimal genau gleich als Abbildung. Kann man die Pfadfaserung als Spezialfall vom Lemma sehen und sich das einmal sparen?

\subsection{Eilenberg-MacLane-Räume}

\begin{defn}
  Sei $G$ eine Gruppe, $n \geq 1$.
  Ein \emph{Eilenberg-MacLane-Raum} vom Typ $K(G, n)$ ist ein punktierter, zusammenhängender topologischer Raum $(X, x_0)$ mit
  \[
    \pi_q(X, x_0) = \begin{cases}
      G & \text{falls $q = n$,} \\
      0 & \text{falls $q \neq n$.}
    \end{cases}
  \]
\end{defn}

\begin{lem}
  Sei $G$ eine abelsche Gruppe, $n \geq 2$.
  Dann existiert ein CW-Komplex $(X, x_0)$ vom Typ $K(G, n)$.
\end{lem}

\begin{proof}
  \TODO{}
\end{proof}

% TODO: Eindeutigkeit von K(G, n)'s?

\begin{bem}
  Sei $(X, x_0)$ ein $K(G, n)$. Dann ist $\Omega X$ ein $K(G, n-1)$, denn

  \begin{align*}
    \pi_q(\Omega X, \gamma_0) & \cong \Hom((S^q, *), (\Omega X, \gamma_0))
    \cong \Hom((\Sigma S^q, *), (X, x_0))
    \cong \pi_{q+1}(X, x_0) \\
    & \cong \begin{cases}
      G & \text{falls $q+1 = n \iff q = n-1$} \\
      0 & \text{falls $q+1 \neq n$.}
    \end{cases}
  \end{align*}
  % XXX: Verträglichkeit mit Gruppenstruktur???
\end{bem}

\subsection{Das Hurewicz-Mod-$\SC$-Theorem}

Sei $(X, x_0)$ ein punktierter topologischer Raum.
Für $n \geq 1$ liefert der Hurewicz-Homomorphismus $h_n : \pi_n(X, x_0) \to H_n(X; \Z)$ einen Zusammenhang zwischen der $n$-ten Homotopiegruppe und der $n$-ten Homologiegruppe von $X$.
Er ist definiert durch $h_n([f]) \coloneqq H_n(f)(\alpha)$ für einen fest gewählten Erzeuger $\alpha \in H_n(S^n; \Z)$.

\begin{satz}[Hurewicz]
  Sei $(X, x_0)$ ein $(n{-}1)$-zusammenhängender topologischer Raum, \dh{} $\pi_i(X, x_0) = 0$ für $i < n$.
  Dann ist $h_i : \pi_i(X, x_0) \to H_i(X; \Z)$ ein Isomorphismus für $0 < i \leq n$.
  Insbesondere gilt $H_i(X; \Z) = 0$ für $0 < i < n$.
\end{satz}

% TODO: Auf Beweis im Hatcher verweisen

\begin{defn}
  Eine Klasse $\SC$ von abelschen Gruppen heißt \emph{Serre-Klasse}, falls
  \begin{enumerate}
    \item Für jede kurze exakte Sequenz $0 \to A \to B \to C \to 0$ von abelschen Gruppen gilt: $B \in \SC \iff A, B \in \SC$.
    \item Für $A, B \in \SC$, sind auch $A \otimes B \in \SC$ und $\Tor(A, B) \in \SC$.
  \end{enumerate}
\end{defn}

\begin{bem}
  Aus der ersten Eigenschaft folgt, dass Bilder, Untergruppen und Quotienten einer Gruppe aus $\SC$ wieder in $\SC$ sind.
  Genauer gilt für eine ab. Gruppe $B$ und eine Untergruppe $A < B$: $B \in \SC \iff A, B/A \in \SC$.
  Durch Induktion kann man zeigen, dass für eine Gruppe $A$ mit endlicher Filtrierung
  $A = F^0 A \supseteq F^1 A \supseteq \ldots \supseteq F^k A = 0$
  gilt: $A \in \SC \iff F^0 A / F^1 A, \ldots, F^{k-1} A / F^k A \in \SC$.
  Außerdem ist die direkte Summe zweier Gruppen aus $\SC$ wieder in $\SC$.
\end{bem}

\begin{defn}
  Es sei $\SC$ eine Serre-Klasse.
  Ein Morphismus $f : A \to B$ zwischen abelschen Gruppen heißt \emph{Isomorphimus modulo $\SC$}, falls $\ker(f), \coker(f) \in \SC$. \\
\end{defn}

\begin{bem}
  Dies ist äquivalent zur Existenz einer exakten Sequenz $K \to A \xrightarrow{f} B \to C$ mit $K, C \in \SC$.
  Eine Gruppe, welche modulo-$\SC$-isomorph zu einer Gruppe aus $\SC$ ist, ist selbst in $\SC$.
\end{bem}

\begin{bspe}
  Man kann leicht zeigen, dass folgende Klassen die Definition erfüllen:
  % XXX: Nummer der Definition
  \begin{itemize}
    \item $\FG \coloneqq \{\, \text{endlich erzeugte Gruppen} \,\}$
    \item $\F \coloneqq \{\, \text{endliche Gruppen} \,\}$
    % TODO: ist das tatsächlich leicht zu zeigen?
    % XXX: weitere Beispiele?
  \end{itemize}
\end{bspe}


% Hatcher, 1.8
\begin{satz}["`Hurewicz-mod-$\SC$-Theorem"']\label{hurewicz-mod-c}
  Es sei $\SC \in \{ \FG, \F \}$ und
  $(X, x_0)$ ein einfach zusammenhängender topologischer Raum.
  Angenommen, $\pi_i(X, x_0) \in \SC$ für $0 < i < n$.
  Dann ist $h_i : \pi_i(X, x_0) \to H_i(X; \Z)$ ein Isomorphismus modulo $\SC$ für $i \leq n$.
  Insbesondere gilt $H_i(X; \Z) \in \SC$ für $0 < i < n$.
\end{satz}

%Mit $\SC \coloneqq \{ 0 \}$ folgt das gewöhnliche Hurewicz-Theorem.

% Hatcher, 1.7 (etwas allgemeiner)
\begin{kor}\label{homotopy-in-c-iff-homology-in-c}
  Es sei $\SC \in \{ \FG, \F \}$ und
  $(X, x_0)$ ein einfach zusammenhängender topologischer Raum.
  Dann gilt für alle $N \in \N \cup \{ \infty \}$:
  \[
    \fa{0 \leq n < N} \pi_n(X, x_0) \in \SC
    \quad \iff \quad
    \fa{1 \leq n < N} H_n(X; \Z) \in \SC.
  \]
\end{kor}

\begin{proof}
  Die Aussage folgt aus Satz \ref{hurewicz-mod-c} durch Induktion über $N$.
\end{proof}

Aus dem Korollar folgt, dass die Homotopiegruppen der Sphären alle endlich erzeugt sind, da die Homologiegruppen der Sphären endlich erzeugt sind.

\begin{defn}
  Es sei $\SC$ eine Serre-Klasse.
  Wir nennen einen topologischen Raum~$X$ \emph{$\SC$-azyklisch}, falls $\widetilde{H}_i(X) \in \SC$ für alle $i \geq 0$.
\end{defn}

\begin{lem}\label{two-of-three}
  Es sei $F \to X \to B$ eine Faserung und die Räume $F$, $X$ und $B$ wegzusammenhängend.
  % TODO: Wirkung beschreiben (bisher nur für Kohomologie getan)
  Wirke $\pi_1(B)$ trivial auf $H_*(F)$.
  Dann gilt folgende $2$-aus-$3$-Eigenschaft: Falls zwei der Räume $F$, $X$ und $B$ $\SC$-azyklisch sind, so auch der dritte.
\end{lem}

\begin{proof}
  Wir betrachten die Serre-Spektralsequenz zu der Faserung mit Koeffizienten in $\Z$.
  Die Aussage, dass die Homologiegruppe $H_n(X)$ in $\SC$ liegt, ist äquivalent dazu, dass die Gruppen $E^\infty_{i,n-i}$ für $i = 0, \ldots, n$ in $\SC$ liegen, denn diese Gruppen sind die Quotienten einer endlichen Filtrierung von $H_n(X)$.

  \emph{Fall 1: $H_n(F), H_n(B) \in \SC$ für alle $n > 0$}: \enspace
  Die universelle Koeffizientenformel liefert
  \[ E^2_{pq} \cong H_p(B; H_q(F; \Z)) \cong \left( H_p(B) \otimes H_q(F) \right) \oplus \Tor(H_{p-1}(B), H_q). \]
  Man sieht durch Unterscheidung der Fälle $p=0$, $p=1$ und $p>1$ sowie $q=0$ und $q > 0$, dass $E^2_{pq} \in \SC$ für $(p, q) \neq (0, 0)$.
  Als Subquotient von $E^2_{pq}$, einer Gruppe aus $\SC$, ist dann auch $E^\infty_{pq} \in \SC$ für $(p, q) \neq (0, 0)$.
  Dies zeigt die Behauptung nach der Bemerkung am Anfang des Beweises.

  \emph{Fall 2: $H_n(F), H_n(X) \in \SC$ für alle $n > 0$}: \enspace
  Wir zeigen nun durch Induktion über $k$, dass $H_p(B) \in \SC$ für $0 < p < k$.
  %Dies gilt trivialerweise für $k = 1$.
  Gelte dies für $k \geq 1$.
  Wir wollen zeigen, dass dann auch $H_k(B)$ in $\SC$ liegt.
  %Zunächst sieht man wie im Fall 1, dass $E^2_{pq}$ und somit auch $E^r_{pq}$ in $\SC$ für $(p, q) \neq (0, 0)$ und $p < k$.
  Für alle $r \geq 2$ gibt es eine kurze exakte Sequenz
  \begin{centertikzcd}[column sep=0.4cm, row sep=0.35cm]
    0 \arrow[r] &
    \ker(d^r_{k,0}) \arrow[d, equal] \arrow[r] &
    E^r_{k,0} \arrow[r, "d^r_{k,0}"] &
    \im(d^r_{k,0}) \arrow[d, "\subseteq"] \arrow[r] &
    0 \\
    & E^{r+1}_{k,0} &&
    E^r_{k-r,r-1}
  \end{centertikzcd}
  Man sieht unter Verwendung der Induktionsannahme und der universellen Koeffizientenformel, dass $E^2_{k-r,r-1}$ und somit auch $E^r_{k-r,r-1}$ in $\SC$ liegen.
  Folglich gilt auch $\im(d^r_{k,0}) \in \SC$, also $E^r_{k,0} \in \SC \iff E^{r+1}_{k,0} \in \SC$.
  Da aber $E^R_{k,0} \cong E^\infty_{k,0} \in \SC$ für $R$ groß genug, gilt $E^r_{k,0} \in \SC$ für alle $r \geq 2$.
  Insbesondere $H_k(B; \Z) \cong H_k(B; H_0(F; \Z)) \cong E^2_{k,0} \in \SC$.

  \emph{Fall 3: $H_n(X), H_n(B) \in \SC$ für alle $n > 0$}: \enspace
  Analog zum vorherigen Fall zeigt man induktiv, dass $H_q(F) \in \SC$ für $0 < q < k$.
  Dazu verwendet man die kurze exakte Sequenz $0 \to \im(d^r_{r,k-r+1}) \hookrightarrow E^r_{0,k} \to E^{r+1}_{0,k}$.
\end{proof}

% 1.10 im Hatcher
\begin{lem}\label{homology-kgn-in-c}
  Es sei $G \in \SC \in \{ \FG, \F \}$.
  Dann ist $K(G, n)$ $\SC$-azyklisch für alle $n \geq 1$.
\end{lem}

\begin{proof}
  Sei zunächst $n=1$.
  \begin{itemize}
    \item Falls $G = \Z$, so stimmt die Aussage, denn der Kreis $S^1$ ist ein $K(\Z, 1)$ und $H_*(S^1; \Z) \in \FG$.
    \item Falls $G = \Z_m$, \TODO{begründen damit, dass der "`unendliche Linsenraum"' ein $K(\Z_m, 1)$ ist}
    \item Falls $G = G_1 \oplus G_2$, dann ist $K(G_1, 1) \times K(G_2, 1)$ ein $K(G, 1)$.
    Wenn die Aussage für $G_1$ und $G_2$ stimmt, so folgt aus dem letzten Lemma, angewendet auf die Produktfaserung $K(G_1, 1) \to K(G_1, 1) \times K(G_2, 1) \to K(G_2, 1)$, dass sie auch für $G$ gilt.
  \end{itemize}
  Da man jede endlich erzeugte abelsche Gruppe als direkte Summe von endlich vielen Summanden der Form $\Z$ und $\Z_m$ schreiben kann, gilt die Aussage für $n=1$.

  Induktiv zeigen wir nun, dass die Aussage für $n$ beliebig gilt.
  Dazu verwenden wir die Pfadraumfaserung $K(G, n) \to P \to K(G, n{+}1)$.
  Es gilt $H_k(P) = 0 \in \SC$ und $H_k(K(G, n)) \in \SC$ für $k \geq 1$ nach Induktionshypothese, also $H_k(K(G, n{+}1)) \in \SC$ für alle $k \geq 1$ nach dem vorherigen Lemma.
\end{proof}


\renewcommand\windowpagestuff{
  % Vertikal:
  \iffalse
  \begin{centertikzcd}[ampersand replacement=\&, column sep=0.5cm, row sep=0.2cm]
    \& \vdots \arrow[d] \\
    \& X_3 \arrow[d] \\
    \& X_2 \arrow[d] \\
    X \arrow[r] \arrow[ru] \arrow[ruu] \& X_1
  \end{centertikzcd}
  \fi
  \vspace{1cm}
  \begin{centertikzcd}[ampersand replacement=\&, column sep=0.2cm, row sep=0.5cm]
    \&\&\& X \arrow[dll] \arrow[dl] \arrow[d] \\
    \dots \arrow[r] \&
    X_3 \arrow[r] \&
    X_2 \arrow[r] \&
    X_1
  \end{centertikzcd}
}
\opencutright
\begin{cutout}{0}{\dimexpr\linewidth-4.5cm\relax}{0pt}{2}
  \begin{defn}
    Ein \emph{Postnikov-Turm} eines wegzusammenhängenden Raumes $X$ ist ein kommutatives Diagramm wie rechts, für das gilt: \\[3pt]
    %\begin{itemize}
    %\item
    \quad$\bullet$\enspace $\pi_i(X \to X_n)$ ist ein Isomorphismus für $i \leq n$ und \\
    \quad$\bullet$\enspace $\pi_i(X_n) = 0$ für $i > n$.
    %\end{itemize}
  \end{defn}
\end{cutout}

\TODO{Bemerkung zur Konstruktion von Postnikov-Türmen}

\begin{bem}
  Es sei ein Postnikov-Turm $\ldots \to X_2 \to X_1$ gegeben.
  Dann kann man durch wiederholtes Anwenden der in \TODO{Ref} beschriebenen Konstruktion einen neuen Postnikovturm $\ldots \to X_2' \to X_1'$ und Homotopieäquivalenzen $X_i \simeq X_i'$ konstruieren, sodass die Abbildungen $X_{i+1}' \to X_i'$ Hurewicz-Faserungen sind.
  Man sieht anhand der langen exakten Sequenz von Homotopiegruppen, dass die Faser von $X_{i+1}' \to X_i'$ ein $K(\pi_{n+1}(X), n{+}1)$ ist.
\end{bem}

\begin{lem}\label{homotopy-in-c-implies-homology-in-c}
  Es sei $\SC \in \{ \FG, \F \}$ und
  $X$ einfach zusammenhängend mit $\pi_i(X, x_0) \in \SC$ für alle $i \geq 0$.
  Dann ist $X$ $\SC$-azyklisch, \dh{} es gilt $H_i(X) \in \SC$ für $i \geq 1$.
\end{lem}

\begin{proof}
  Es sei $\ldots \to X_{i+1} \to X_i \to \ldots \to X_1$ ein Postnikov-Turm von $X$, dessen Abbildungen $X_{i+1} \to X_i$ Hurewicz-Faserungen sind.
  Wir zeigen, dass $H_i(X_k; \Z) \in \SC$ für alle $i, k > 0$.
  Die Aussage stimmt für $k = 1$, da alle Homotopiegruppen und somit auch Homologiegruppen von $X_1$ gleich Null sind.
  %Beachte, dass $X_2$ ein $K(\pi_2(X); 2)$, denn $\pi_1(X_2) \cong \pi_1(X) = 0$.
  %Da $X_2$ ist ein  ist, stimmt die Aussage nach Lemma \ref{homology-kgn-in-c} für $k=2$.
  Gelte die Aussage nun für ein $k \geq 1$.
  Wir verwenden die Faserung $K(\pi_{k+1}(X), k{+}1) \to X_{k+1} \to X_k$.
  Nach Lemma \ref{homology-kgn-in-c} sind die Homologiegruppen der Faser in $\SC$.
  Gleiches gilt für den Basisraum nach Induktionsvoraussetzung,
  und somit auch für $X_k$ nach Lemma \ref{two-of-three}.

  Es gilt $H_i(X; \Z) \cong H_i(X_k; \Z) \in \SC$ für $k \geq i$, da $\pi_i(X \to X_k)$ und nach einem Korollar der relativen Version des Hurewicz-Theorems damit auch $H_i(X \to X_k)$ ein Isomorphismus für $i \leq k$ ist.
  % TODO: Fußnote zu dem letzten Satz?
\end{proof}


\begin{proof}[Beweis von Satz \ref{hurewicz-mod-c}]
  Aufgrund der Natürlichkeit des Hurewicz-Homomorphismus kommutiert folgendes Diagramm:
  \begin{centertikzcd}
    \pi_n(X) \arrow[r, "\cong"] \arrow[d, "h_n"] &
    \pi_n(X_n) \arrow[d, "h_n"] \\
    H_n(X) \arrow[r, "\cong"] &
    H_n(X_n)
  \end{centertikzcd}
  Es genügt daher zu zeigen, dass $h_n : \pi_n(X_n) \to H_n(X_n; \Z)$ ein Isomorphismus modulo $\SC$ ist.
  Wir betrachten die Serre-Spektralsequenz zur Faserung $F_n \to X_n \to X_{n-1}$.
  \[ E^2_{pq} \cong H_p(X_{n-1}; \underbrace{H_q(K(\pi_n(X), n); \Z)}_{= 0 \text{ für } 0 < q < n}), \]
  % TODO: Begründung für das Verschwinden dieser Homologiegruppen
  Es verschwinden also alle Einträge zwischen der 0-ten und der $n$-ten Zeile. Für diese gilt
  Wir betrachten die exakte Sequenz
  % Alternativ:
  \iffalse
    \usepackage{stackengine} % Für Zeilen unter mathematischem Symbol
    \def\useanchorwidth{T}\stackMath
    \stackunder{E^n_{n+1,0}}{
      \def\stackalignment{l}
      \stackunder{= E^2_{n+1,0}}{
        = H_{n+1}(X_{n-1})
      }
    } \arrow[r] &
  \fi
  \begin{centertikzcd}[row sep=0.1cm, column sep=0.5cm]
    H_{n+1}(X_{n-1}) \arrow[d, equal] &&
    H_n(F_n) \arrow[d, equal] &
    0 \arrow[rd] &&
    0 &&&
    H_n(X_{n-1}) \arrow[d, equal] \\
    E^2_{n+1,0} \arrow[d, equal] && E^2_{0,n} \arrow[d, equal] && E^\infty_{0,n} \arrow[rrd] \arrow[ru] &&&&
    E^2_{n,0} \arrow[d, equal] \\
    %\underbrace{E^n_{n+1,0}}_{\mathrlap{\substack{= E^2_{n+1,0} \\ = H_{n+1}(X_{n-1})}}} \arrow[r, "d^n_{n+1,0}"] &
    E^n_{n+1,0} \arrow[rr, "d^r_{n+1,0}"] &&
    E^n_{0,n} \arrow[rrrr] \arrow[rru] &&&&
    H_n(X_n) \arrow[rr] &&
    E^\infty_{n,0} \arrow[r] &
    0
  \end{centertikzcd}
  welche sich aus zwei kurzen Sequenzen zusammensetzt:
  \begin{itemize}
    \item Die linke Sequenz ist exakt, da $E^\infty_{0,n} \cong E^{n+1}_{0,n} \cong \ker(d^r_{0,n}) / \im(d^r_{n+1,0}) = E^n_{0,n} / \im(d^r_{n+1,0})$.
    \item Die rechte Sequenz ist exakt, da $E^\infty_{0,n} = F^0 H_n(X_n)$ und $E^\infty_{n,0} = F^n H_n(X_n) / F^{n-1} H_n(X_n)$ für eine Filtrierung $0 = F^{-1} H_n(X_n) \subseteq F^0 H_n(X_n) \subseteq \ldots \subseteq F^n H_n(X_n) = H_n(X_n)$.
    Da $F^p H_n(X_n) / F^{p-1} H_n(X_n) = E^\infty_{p,n-p} = 0$ für $p = 1, \ldots, n-1$, gilt $F^0 H_n(X_n) = \ldots = F^{n-1} H_n(X_n)$.
  \end{itemize}
  Aus Lemma \label{homotopy-in-c-implies-homology-in-c} folgt, dass $E^n_{n+1,0}, E^\infty_{n,0} \in \SC$.
  Somit ist der mittlere Morphismus $H_n(F_n) = E^n_{0,n} \to H_n(X_n)$ ein Isomorphismus modulo $\SC$.
  \TODO{Begründen, warum dieser Morphismus genau der von der Inklusion $F_n \hookrightarrow X_n$ induzierte Morphismus ist.}
  Betrachte nun das kommutative Diagramm
  \begin{centertikzcd}
    \pi_n(F_n) \arrow{d}{h_n}[swap]{\cong} \arrow[r, "\cong"] &
    \pi_n(X_n) \arrow[d, "h_n"] \\
    H_n(F_n) \arrow[r] &
    H_n(X_n)
  \end{centertikzcd}
  Dabei sind die vertikalen Morphismen die Hurewicz-Homomorphismen und die horizontalen Morphismen werden durch die Inklusion induziert.
  Der linke Hurewicz-Homomorphismus ist nach dem Hurewicz-Theorem ein Isomorphismus, da $F_n$ $(n{-}1)$-zusammenhängend ist.
  Der obere Morphismus ist ein Isomorphismus, wie man anhand der langen exakten Sequenz der Faserung $F_n \to X_n \to X_{n-1}$ sehen kann.
  Der untere Morphismus ist ein Isomorphismus modulo $\SC$, wie wir gerade eben gesehen haben.
  Somit ist auch der rechte Morphismus im Diagramm ein Isomorphismus modulo $\SC$.
\end{proof}

\begin{bem}
  Man kann in diesem Kapitel die Voraussetzung, dass $X$ einfach zusammenhängend ist, ersetzen durch die Forderung, dass $X$ wegzusammenhängend und abelsch ist, \dh{} die Wirkung der Fundamentalgruppe $\pi_1(X)$ auf den höheren Homotopiegruppen $\pi_n(X)$ trivial ist.
\end{bem}

\subsection{Rationale Kohomologie von Räumen vom Typ $K(\Z, n)$}

% TODO: Etwas schreiben über rationale Homotopiegruppen

% Hatcher, 1.20
\begin{satz}\label{rational-homology-kzn}
  Für $n \geq 1$ gilt
  \[
    H^*(K(\Z, n); \Q) \cong \begin{cases}
      \Q[x], & \text{falls $n$ gerade}, \\
      \Lambda_\Q[x], & \text{falls $n$ ungerade}
    \end{cases}
  \]
  als graduierte Ringe mit Erzeuger $x \in H^n(K(\Z, n); \Q)$.
  Dabei bezeichnet $\Lambda_\Q[x]$ die äußere Algebra mit Erzeuger $x$.
\end{satz}

\begin{proof}
  Durch Induktion über $n$.
  Der Satz gilt für $n = 1$, denn der Kreis $S^1$ ist ein $K(\Z, 1)$ und es gilt bekanntermaßen $H^*(S^1; R) \cong \Lambda_R[x]$ für $R = \Z$ und somit auch für $R = \Q$. \\
  Im Induktionsschritt nutzen wir die Pfadraumfaserung $F \coloneqq K(\Z, n{-}1) \to P \to B \coloneqq K(\Z, n)$. Da $K(\Z, n)$ für $n \geq 2$ einfach zusammenhängend ist, gilt für deren zugehörige Serre-Spektralsequenz
  $E_2^{p,q} \cong H^p(B; H^q(F))$.

  \vspace{0.5cm}

  \renewcommand\windowpagestuff{
    \begin{center}\begin{tikzpicture}[x=16pt,y=16pt]\begin{scope}[shift={(0,0)}]
      \foreach \x in {1,2,3,5,6,7}{
        \foreach \y in {0,1,...,3}{
          \zeroDot{\x}{\y}
        }
      }
      \foreach \x in {0,...,8}{
        \zeroDot{\x}{4}
      }
      \foreach \y in {0,...,4}{
        \node at (9.5,\y) {$\cdots$};
      }
      \node at (0,0) {$\Q$\small $1$};
      \zeroDot{0}{1}
      \zeroDot{0}{2}
      \node at (0,3) {$\Q$\small $a$};
      \node at (4,0) {$\Q$\small $x$};
      \zeroDot{4}{1}
      \zeroDot{4}{2}
      \node at (4,3) {$\Q$\small $ax$};
      \draw[->,gray] (0.6,2.6) -- (3.4,0.4);
      \node at (8,0) {$\Q$\small $x^2$};
      \zeroDot{8}{1}
      \zeroDot{8}{2}
      \node at (8,3) {$\Q$\small $ax^2$};
      \draw[->,gray] (4.6,2.6) -- (7.4,0.4);
      \draw[->] (-0.7,-0.6) -- (10.3,-0.6); %node[below] {p};
      \draw[->] (-0.7,-0.6) -- (-0.7,4.5); %node[left] {q};
      \node[left] at (-0.7,0) {$0$};
      \node[left] at (-0.9,1.7) {\vdots};
      \node[left] at (-0.7,3) {$n{-}1$};
      \node[below] at (0,-0.6) {$0$};
      \node[below] at (2,-0.7) {$\cdots$};
      \node[below] at (4,-0.7) {$n$};
      \node[below] at (6,-0.7) {$\cdots$};
      \node[below] at (8,-0.6) {$2n$};
      \node[below] at (9.5,-0.7) {$\cdots$};
    \end{scope}\end{tikzpicture}\end{center}
  }
  \opencutright
  \begin{cutout}{0}{\dimexpr\linewidth-8cm\relax}{0pt}{8}
    \emph{Falls $n$ gerade:} \enspace
    Dann sieht die Spektralsequenz auf der Seite $E_r$, $r \leq n$ aus wie rechts skizziert (dabei stehen die Gitterpunkte für die Nullgruppe).
    Das sieht man folgendermaßen: Zunächst ist $E_2^{pq} = 0$ und somit $E_r^{pq} = 0$ außer für $q \in \{ 0, n{-}1 \}$, denn nach Induktionsvoraussetzung gilt $H^*(F; \Q) \cong \Lambda_\Q[x]$.
    Es folgt, dass nur auf der $n$-ten Seite $E_n$ nicht verschwindende Differentiale existieren können und $E_2 \cong E_n$ und $E_{n+1} \cong E_\infty$ gilt.
    Außerdem ist $E_n^{0,0} \cong E_2^{0,0} \cong \Q$ und $E_n^{0,n-1} \cong E_2^{0,n-1} \cong \Q$, da $B$ zusammenhängend ist.
    Die Spektralsequenz konvergiert gegen $H^*(P; \Q)$.
    Da $P$ zusammenziehbar ist, gilt $H^0(P; \Q) = \Q$ und $H^n(P; \Q) = 0$ für $n > 0$.
    Folglich ist $E_{n+1}^{pq} \cong E_\infty^{pq} = 0$ außer für $p = q = 0$.
    Insbesondere gilt $E_{n+1}^{0,n-1} \cong E_\infty^{0,n-1} = 0$.
    %Es gilt $E_n^{0,n-1} \cong E_2^{0,n-1} \cong \Q$, da die von $E_r^{0,n-1}$ ausgehenden Differentiale $d_r^{0,n-1}$ für $r < n$ Null und die dort eintreffenden Differentiale für alle $r$ Null sind.
    Das eingezeichnete Differential $d_n^{0,n-1} : E_n^{0,n-1} \to E_n^{n,0}$ ist nun injektiv, denn $\ker(d_n^{0,n-1}) \cong E_{n+1}^{0,n-1} = 0$.
    Dieses Differential ist auch surjektiv, denn $\coker(d_n^{0,n-1}) \cong E_{n+1}^{n,0} \cong E_\infty^{n,0} = 0$, also ein Isomorphismus.
    Somit $H^n(B; \Q) \cong E_2^{n,0} \cong E_n^{n,0} \cong E_n^{0,n-1} \cong \Q$.
    Der zweite Isomorphismus kommt daher, da für $r \leq n-1$ alle Differentiale von und nach $E_r^{n,0}$ Null sind. % XXX: besser formulieren?
    Damit ist $E_2^{n,n-1} \cong H^{n}(B; H^{n-1}(F; \Q)) \cong H^{n}(B; \Q) \cong \Q$.
    Induktiv sieht man nun, dass die Abbildungen $d_n^{kn,n-1}$ Isomorphismen sind und dass $H^{kn}(B; \Q) \cong E_2^{kn,0} \cong  E_2^{kn,n-1} \cong \Q$ für alle $k \geq 0$.
    Damit haben wir gezeigt, dass die graduierte, additive Struktur von $H^*(B; \Q)$ wie behauptet ist.
  \end{cutout}
  Es sei nun $a \in E_n^{0,n-1} \cong H^0(B; H^{n-1}(F; \Q))$ ungleich Null und $x \coloneqq d_n^{0,n-1}(a) \in E_n^{n,0} \cong H^n(B; H^0(F; \Q)) \cong H^n(B; \Q)$.
  Dann gilt auch $x \neq 0$ und $ax \coloneqq m_r(a, x) \neq 0$, da wegen (ii) und (iii) das Produkt $m_r$ gerade dem kanonischen Produkt $H^n(B; H^0(F; \Q)) \times H^0(B; H^{n-1}(F; \Q)) \to H^n(B; H^{n-1}(F; \Q))$ entspricht. % XXX: kommutatives Diagramm dazu?
  Es gilt
  $0 \neq d_n^{n,n-1}(ax) = d_n^{0,n-1}(a)x - a d_n^{n,0}(x) = xx$.
  Da das Produkt $xx \in E_n^{2n,0}$ gerade dem Cup-Produkt $x \cup x \in H^{2n}(B; \Q)$ entspricht, ist $x \cup x \neq 0$, also ein Erzeuger von $H^{2n}(B; \Q)$.
  Induktiv ist nun $0 \neq d_n^{kn,n-1}(a x^k) = x^{k+1} \in E_n^{kn,0}$ da ja $0 \neq a x^k$.
  Somit ist für alle $k$ das $k$-fache Cup-Produkt $x^k \in H^{kn}(B; \Q)$ ein Erzeuger.

  \vspace{1.5cm}

  \renewcommand\windowpagestuff{
    \begin{center}\begin{tikzpicture}[x=16pt,y=16pt]\begin{scope}[shift={(0,0)}]
      \foreach \x in {1,2,3}{
        \foreach \y in {0,1,...,6}{
          \zeroDot{\x}{\y}
        }
      }
      \foreach \y in {0,...,6}{
        \zeroDot{5}{\y}
      }
      \foreach \x in {0,...,5}{
        \node at (\x,7.3) {\vdots};
      }
      \node at (0,0) {$\Q$\small $1$};
      \zeroDot{0}{1}
      \zeroDot{0}{2}
      \node at (0,3) {$\Q$\small $a$};
      \zeroDot{0}{4}
      \zeroDot{0}{5}
      \node at (0,6) {$\Q$\small $a^2$};
      \node at (4,0) {$\Q$\small $x$};
      \zeroDot{4}{1}
      \zeroDot{4}{2}
      \node at (4,3) {$\Q$\small $ax$};
      \zeroDot{4}{4}
      \zeroDot{4}{5}
      \node at (4,6) {$\Q$\small $a^2 x$};
      \draw[->,gray] (0.6,2.6) -- (3.4,0.4);
      \draw[->,gray] (0.6,5.6) -- (3.4,3.4);
      \draw[->] (-0.7,-0.6) -- (5.6,-0.6); %node[below] {p};
      \draw[->] (-0.7,-0.6) -- (-0.7,7.8); %node[left] {q};
      \node[left] at (-0.7,0) {$0$};
      \node[left] at (-0.9,1.7) {\vdots};
      \node[left] at (-0.7,3) {$n{-}1$};
      \node[left] at (-0.9,4.7) {\vdots};
      \node[left] at (-0.7,6) {$2n{-}2$};
      \node[below] at (0,-0.6) {$0$};
      \node[below] at (2,-0.7) {$\cdots$};
      \node[below] at (4,-0.7) {$n$};
    \end{scope}\end{tikzpicture}\end{center}
  }
  \opencutleft
  \begin{cutout}{0}{0pt}{\dimexpr\linewidth-5.5cm\relax}{11}
    \emph{Falls $n$ ungerade:} \enspace
    Dann ist $E_r^{p,q} = 0$ für alle $q$, die kein Vielfaches von $n-1$ sind.
    Somit verschwinden alle Differentiale auf $E_r$ für $r < n$ und $E_2 \cong E_n$.
    Für $0 < m < n$ verschwinden alle Differentiale von und nach $E_r^{m,0}$ und daher ist $H^m(B; \Q) \cong E_2^{m,0} \cong E_\infty^{m,0} = 0$ und  folglich $E_2^{m,k} = 0$ für alle $k \geq 0$.
    Selbiges gilt folglich auch für $n < m < 2n$ und allgemeiner für solche $m$, die kein Vielfaches von $n$ sind.
    Analog wie im vorherigen Fall sieht man, dass das eingezeichnete Differential $d_n^{0,n-1}$ ein Isomorphismus ist.
    Somit $H^n(B; \Q) \cong H^n(B; H^0(F; \Q)) \cong \Q$ und $E_2^{n,k(n-1)} \cong \Q$ für alle $k \geq 0$.
    Sei $a \in H^0(B; H^{n-1}(F; \Q))$ ungleich Null und $x \coloneqq d_n^{0,n-1}$.
    Dann ist auch $a^2 \neq 0 \in E_n^{0,2n-2}$ und $d_n^{0,2n-2}(a^2) = d_n^{0,n-1}(a)a + d_n^{0,n-1}(a)a = xa + ax = (-1)^{0 \cdot n + (n-1) \cdot 0} ax + ax = 2ax \neq 0$.
    Also ist $d_n^{0,2n-2}$ ein Isomorphismus.
    Analog sieht man, dass $d_n^{0,k(n-1)}$ für alle $k \geq 1$ ein Isomorphismus ist.
    Es bleibt zu zeigen, dass $H^{kn}(B; \Q) = 0$ für $k > 1$.
    Das einzige potentiell nichttriviale Differential, das bei $E_r^{2n,0}$ ankommt, ist $d_n^{n,n-1}$.
    Dieses ist aber Null, da $\ker(d_n^{n,n-1}) = \im(d_n^{0,2n-2}) = E_n^{n,n-1}$.
    Also $H^{2n}(B; \Q) \cong E_2^{2n,0} \cong E_\infty^{2n,0} = 0$ und $E_2^{2n,k} = 0$ für alle $k \geq 0$.
    Für $k > 2$ sieht man durch Induktion, dass alle Differentiale von und nach $E_r^{2n,0}$ verschwinden und daher $H^{kn}(B; \Q) = 0$.
  \end{cutout}
\end{proof}

\subsection{Satz von Serre}

\begin{lem}
  Es sei $n \geq 2$ ungerade und $X$ ein $(n{-}1)$-zusammenhängender topologischer Raum.
  Angenommen, $H^k(X; \Z) = 0$ für $k > n$ und $H_n(X) \cong \pi_n(X)$ ist die direkte Summe von $\Z$ und einer endlichen Gruppe.
  Dann sind die Homotopiegruppen $\pi_k(X, x_0)$, $k > n$ endlich.
\end{lem}

\begin{proof}
  Durch Töten der höheren Homotopiegruppen bekommen wir eine Abbildung $f : X \to K(\pi_n(X), n) \approx K(\Z, n) \times K(G, n)$, wobei $G$ eine endliche Gruppe ist.
  Wir führen die in Lemma \ref{convert-to-fibration} beschriebene Konstruktion durch und erhalten einen zu $X$ homotopieäquivalenten Raum $X'$ und eine Hurewicz-Faserung $p : X' \to K(\pi_n(X), n)$ mit Faser $F$.
  Anhand der langen exakten Sequenz dieser Faserung sehen wir, dass
  \[
    \pi_i(F) \cong \begin{cases}
      0, & \text{für $i \leq n$}, \\
      \pi_i(X') \cong \pi_i(X), & \text{für $i > n$.}
    \end{cases}
  \]
  Wir wenden diesselbe Konstruktion auf die Inklusion $F \hookrightarrow X'$ an und bekommen einen zu $F$ homotopieäquivalenten Raum $F'$ und eine Faserung $q : F' \to X'$.
  Aus der langen exakten Homotopiesequenz ergibt sich, dass die Faser $\tilde{F}$ dieser Faserung ein $K(\pi_n(X), n{-}1)$ ist.

  Wir wollen die Serre-Spektralsequenz der Faserung $\tilde{F} \to F' \to X'$ mit Koeffizienten in $\Q$ verwenden.
  Dazu untersuchen wir zunächst die rationale Kohomologie von Faser und Basisraum.
  Lemma \ref{homology-kgn-in-c} impliziert, dass die Homotopiegruppen und wegen Korollar \ref{homotopy-in-c-iff-homology-in-c} auch die reduzierten Homologiegruppen von $K(G, n{-}1)$ endlich sind.
  Nach der universellen Koeffizientenformel verschwinden somit alle reduzierten Kohomologiegruppen von $K(G, n{-}1)$ mit Koeffizienten in $\Q$.
  Wir sehen nun an der Serre-Spektralsequenz der Produktfaserung $K(G, n{-}1) \to K(\pi_n(X), n{-}1) \to K(\Z, n{-}1)$ und Lemma \ref{rational-homology-kzn}, dass $H^*(K(\pi_n(X), n{-}1); Q) \cong H^*(K(\Z, n{-}1); \Q) \cong \Q[x]$ mit Erzeuger $a \in H^{n-1}(K(\pi_n(X), n{-}1); \Q)$.
  Aus der universellen Koeffizientenformel folgt, dass die rationale Kohomologie von $X'$ gleich $\Q$ ist in Grad 0 und in Grad $n$, und Null sonst.

  Wir wissen also, dass $E_2^{pq} = 0$ außer falls $p \in \{ 0, n \}$ und $n{-}1 \divides q$ gilt.
  Die Spektralsequenz besitzt also auf der $E_2$-Seite und damit auch auf der $E_n$-Seite die gleichen Einträge wie die Spektralsequenz aus dem zweiten Fall ($n$ ungerade) des vorhergehenden Lemmas.
  Genau wie dort schließen wir, dass $d_n^{0,n-1}$ ein Isomorphismus ist (denn $H^n(X'; \Q) = 0$) und dass aufgrund der multiplikativen Struktur der Spektralsequenz auch die Differentiale $d_n^{0,k(n-1)}$ für $k \geq 1$ Isomorphismen sind.
  Somit ist $E_{n+1}^{pq} = 0$ außer für $p = q = 0$.
  Es folgt, dass $H^n(F'; \Q) = 0$ für $n > 0$.
\end{proof}

% Hatcher, 1.21
\begin{satz}
  Die Homotopiegruppen $\pi_i(S^n)$ sind endlich bis auf die Gruppen $\pi_{4k-1}(S^{2k})$, $k \geq 1$, welche jeweils isomorph zu einer direkten Summe von $\Z$ mit einer endlichen Gruppe sind.
\end{satz}

\begin{proof}
  \TODO{}
\end{proof}

% TODO:
% * Homologische Spektralsequenzen einführen
% * Eilenberg-MacLane-Räume
% * Abbildungen in Faserungen konvertieren
% "pathspace fibration"
% simpliziale Approximation
% pi_q(S^n) wobei q < n
% pi_n(S^n) = Z (mit Freudenthal-Suspensionssatz?)
% pi_n(S^1)
% * Satz von Serre (1.21)
% * Was hat die Hopf-Invariante mit dem ganzen Zeugs zu tun?
% * Existenz der Homologie-Spektralsequenz

% Theoreme, die benutzt werden, die man beweisen müsste:
% * Zelluläre Approximation

\end{document}
